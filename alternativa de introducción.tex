\documentclass{article}
\usepackage{graphicx} % Required for inserting images

\title{ideas}
\author{r.reyes }
\date{December 2023}

\begin{document}

\maketitle

\section{Introducción}

En el medio interestelar hay muchas hay grandes regiones de nubes moleculares frías en las cuales se pueden formar estrellas a partir del colapso gravitacional y estas estrellas interactúan con el medio que la rodea. Cuando hay grandes concentraciones de gas y polvo en el medio se forman los \textit{glóbulos}, que se cree que se forman por inhomogeneidades en el medio (Dibai ?). 

Estos glóbulos interactúan con la radiación UV de estrellas jóvenes masivas en regiones de formación estelar o en nebulosas alrededor de estrellas. Durante esta interacción se forma un frente de ionización, el cuál podemos ver en la mayoría de los casos. Dependiendo de que tan intenso sea el flujo radiativo por parte de la o las estrellas, en algunas ocasiones podemos ver un flujo fotoevaporativo por parte del glóbulo que es causado por la radiación incidente.

Los primeros glóbulos fueron observados por Bart Bok en 1940, estos glóbulos son nubes oscuras, relativamente pequeños comparados con otras regiones de formación estelar, que tienen gran cantidad de gas y polvo. Estos glóbulos contienen principalmente hidrógeno molecular en su interior, así como también pueden tener otras moléculas, metales e incluso algunos silicatos. Si bien puede tener formación estelar en su interior no podemos ver la radiación UV ya que es absorbida por el hidrógeno atómico y el polvo, por eso que se ven oscuras. Sin embargo, estos pueden ser radiados externamente, en regiones de formación estelar, por estrellas jóvenes masivas que se están formando cerca, y en algunos casos podemos ver el frente de ionización.

Esta interacción entre estrellas y glóbulos se puede dar a diferentes escalas, lo que nos da una gran variedad de estructuras. Entre las de mayor tamaño se encuentran lo que parecen ser columnas, pilares o trompas de elefantes, como se les conoce en la literatura, que llegan a tener un tamaño de $\sim0.6$pc y una densidad del orden de $10^3cm^{-3}$. 

\end{document}
