\documentclass{book}
\usepackage{graphicx} % Required for inserting images
\usepackage{subcaption}

\usepackage[utf8]{inputenc}
\usepackage[T1]{fontenc}
\usepackage{listings}
\usepackage{appendix}
\usepackage{siunitx}
\usepackage{multirow}
\usepackage[version=4]{mhchem}
\usepackage{natbib}
\usepackage{booktabs}
\usepackage{bachelorstitlepageUNAM}
\usepackage[spanish,es-nodecimaldot,es-tabla]{babel}
\usepackage{tikz} 
\usepackage{tocloft}
\graphicspath{{./figs/}}
\usepackage{setspace}
\usepackage{hyperref}
\usepackage{float}              % Para usar [H] en las figures
\usepackage{pdfpages}

\usepackage{tikz}
\usepackage{lipsum}
\usepackage{caption}
\usepackage{array}   % for \newcolumntype macro
\newcolumntype{L}{>{$}l<{$}} % math-mode version of lrc column types
\newcolumntype{R}{>{$}r<{$}} 
\newcolumntype{C}{>{$}c<{$}}

\DeclareSIUnit\msun{\text{M\ensuremath{_\odot}}}
\DeclareSIUnit\mum{\text{\ensuremath{\mu}m}}
\usepackage{astrojournals}
\bibliographystyle{aasjournal}
\renewcommand{\lstlistingname}{Algoritmus}

\title{Avances de tesis}
\author{r.reyes }
\date{December 2023}

\begin{document}

\begin{titlepage}
\includepdf[scale=1,pagecommand={}]{Portada_final.pdf}
\end{titlepage}


\chapter*{Abstract}

The circumstellar nebula M1-67 around the Wolf–Rayet star WR~124
contains hundreds of small neutral globules, as revealed by recent
images from the \textit{James Webb Space Telescope} (JWST). The
ionized emission of the nebula displays an intricate pattern of shells
and filaments, many of which appear associated with the globules but
displaced toward the central star. We propose a simple model for the
nebula in which photoevaporative flows from the irradiated surfaces of
the globules interact with the stellar wind of the Wolf–Rayet star to
form hemispherical emission shells. We test this model against JWST
and H$\alpha$ images of the nebula obtained with the \textit{Hubble Space
  Telescope} (HST), finding good agreement for the best-observed and
most isolated globules. The model provides a physical explanation for
the observed morphology of the nebula and globules, and suggests that
the globules are hydrodynamically shielded from the stellar wind by
the photoevaporative flows. We derive an independent estimate of the
stellar wind strength, which is consistent with values previously
obtained from stellar atmosphere modeling. We are also able to
constrain the three-dimensional distribution of the globules.

% \chapter*{Agradecimientos}

% A la persona que más me gustaría darle las gracias es a mi asesor,
% el Dr. Will. Él me supo guiar durante toda la maestría y estuvo ahí
% para mí siempre, sobre todo en los momentos más difíciles. Gracias a
% él, mi aventura en el mundo de la astronomía ha sido la más
% maravillosa que he tenido. Gracias doctor por toda su paciencia,
% apoyo y conocimiento que me compartió.

% También quisiera agradecer a la UNAM en general. Por aceptarme en el
% posgrado, lo cual para mí fue todo un reto. Por darme todas las
% comodidades para llevar a cabo mis estudios y compartirlo con la
% gente. Al campus Morelia porque ahí conocí a mi maestra de baile
% Lupita y a todos mis amiguitos, a quienes aprecio con toda mi alma.
% Con ustedes aprendí que no es necesario voltear al cielo para ver
% las estrellas, y menos si es para verlas bailar. Al IRyA por darme
% todo su apoyo incondicional, y sobre todo a los de divulgación, que
% aunque no lo crean, en sus eventos yo siempre me emocionaba tanto
% como los niños. A mamá Karin por toda su paciencia y cariño. También
% a mis compañeros de clase quienes siempre estuvieron ahí,
% literalmente.

% Quiero agradecer a mis sinodales por su tiempo, sugerencias y
% comentarios. Con su ayuda mejoré mucho mi trabajo. Gracias Toala! ya
% que por tu culpa me divertí mucho con este proyecto que tanto me
% fascinó.

% Agradezco a CONAHCYT por el apoyo financiero brindado para mis
% estudios e investigaciones. También al proyecto PAPIIT IN109823 de
% DGAPA por el apoyo financiero para hacer posible este trabajo.

% Finalmente, Gracias Dra. Gloria por coincidir en esta vida y
% mostrarme lo maravilloso que es este universo.

\chapter*{Acknowledgements}

The person I most wish to thank is my advisor, Dr.~Will. He guided me
throughout the entire master’s program and was always there for me,
especially in the most difficult moments. Thanks to him, my adventure
in the world of astronomy has been the most wonderful I have ever
experienced. Thank you, Doctor, for all the patience, support, and
knowledge you shared with me.

I would also like to thank UNAM in general: for accepting me into the
graduate program, which for me was quite a challenge; for providing
all the facilities I needed to carry out my studies; and for giving me
the chance to share them with others. To the Morelia campus, because
there I met my dance teacher Lupita and all my dear friends, whom I
cherish with all my heart. With you I learned that it is not necessary
to look up at the sky to see the stars—especially if it is to watch
them dance. To IRyA for giving me its unconditional support, and
especially to the outreach team, because although you may not believe
it, at your events I was always as excited as the children. To mamá
Karin for all her patience and affection. Also to my classmates, who
were always there—literally.

I want to thank my committee members for their time, suggestions, and
comments. With your help I greatly improved my work. Thanks,
Toalá!—since it was your fault that I had so much fun with this
project that fascinated me so much.

I am grateful to CONAHCYT for the financial support provided for my
studies and research. Also to the PAPIIT project IN109823 of DGAPA for
the financial support that made this work possible.

Finally, thank you, Dra.~Gloria, for crossing paths with me in this
life and showing me how wonderful this universe is.

\newpage

\tableofcontents

\newpage

\chapter{Introduction}\label{Capitulo 1:introduccion}

\textit{Globules} are dense concentrations of gas and dust in the
interstellar medium that are thought to form through thermal
instabilities, gravitational collapse, or turbulence
\citep{Ballesteros:2011,Padoan:2002}. These globules can arise in
regions of massive star formation or in nebulae around evolved stars,
such as planetary nebulae \citep{O'Dell:2007}.

In general, globules show a wide range of sizes. For example, when we
refer to globules in regions of massive star formation, they are
commonly large, $\sim$\SI{0.1}{pc} \citep{Schenider:2016}, whereas in
nebulae around evolved stars they are smaller, $\sim$\SI{e-2}{pc}
\citep{GFGahm:2013}.

The first globules were observed by Bart Bok in 1940. As we can see in
Figure \ref{fig:Banard}, because the background stars are reddened by
dust, these globules appear as dark clouds, given their large amounts
of neutral gas and dust. The globules contain primarily molecular
hydrogen in their interiors, and may also harbor other molecules
\citep{Amin:2005, DFrancesco:2002}. Although star formation can occur
inside them, the ionizing radiation from such stars cannot be observed
because it is absorbed by the neutral hydrogen (both molecular and
atomic) and the dust between the stars and the observer. For this
reason, they appear dark.

When globules are found in regions of massive star formation, they can
interact with the ultraviolet (UV) radiation from nearby young massive
stars, or with the radiation of the central star if the globules are
in a circumstellar nebula. In such cases the ionization front can be
seen as a bright rim of emission (see Figures \ref{fig:Pillars} and
\ref{fig:nudos}).

\begin{figure}[htb]
    \centering
    \includegraphics[width=0.85\textwidth]{images Chapter 1/C1_Bok_globule.jpg}
    \caption{Example of a Bok globule. Image of Barnard 68 taken with
      the Very Large Telescope FORS1 at 440 nm, 557 nm, and 768 nm,
      with an angular size of \ang{;6.83;}$\times$\ang{;6.83;}. A dark
      region can be seen, which is the globule itself, together with
      the apparent reddening of stars caused by dust on the globule’s
      surface. In this image there is no evidence of external
      photoevaporation from nearby stars \citep{Alves:2001}.}
    \label{fig:Banard}
\end{figure}

This interaction between stars and globules can occur on different
scales, giving rise to a wide variety of structures. Among the largest
are those that resemble columns, pillars, or “elephant trunks,” as
they are known in the literature. These can reach sizes of
$\sim$\SI{1}{pc} and densities of order \SI{e3}{cm^{-3}}. Such
interactions can also occur within H\,{\sc ii} regions, as shown in
Figure \ref{fig:Pillars}.

\begin{figure}[htb]
    \centering
    \includegraphics[width=1 \textwidth]{images Chapter 1/C1_Pillars.jpg}
    \caption{\textbf{A:} Two examples of pillars. In each case, the
      right-hand image is observed at \SI{2.12}{\mu m} (\SI{}{H_2})
      and the left-hand image shows \SI{}{H_2-Br_{\gamma}}
      \citep{Hartigan:2015}. \textbf{B:} An example of an elephant
      trunk. This is an image of M16 taken with WFPC2 using the F656N
      filter; the \SI{30}{\arcsecond} bar corresponds to
      \SI{9e17}{cm} (\SI{0.29}{pc}) \citep{JJHester:1996}. \textbf{C:}
      The outflow of the Tadpole globule, consisting of the HH900
      jet+outflow system. The lower panel shows the object in
      \SI{}{H\alpha} with the continuum \citep{MeganReiter:2019}.}
    \label{fig:Pillars}
\end{figure}

On smaller scales are the so-called EGGs (Evaporating Gaseous
Globules), which have sizes of $\sim$\SI{e-2}{pc}, and the proplyds,
which are $\le\SI{e-2}{pc}$. These globules are found not only in star
forming regions, but also in nebulae around evolved stars, where they
are more commonly known as \textit{knots}. An example of this is panel
\textbf{D} in Figure \ref{fig:nudos}. In this work we will study in
greater detail the knots present in a nebula surrounding a particular
evolved star.

\begin{figure}[htb]
    \centering
    \includegraphics[width=1 \textwidth]{images Chapter 1/C1_Globulettes.jpg}
    \caption{\textbf{A:} Proplyds with their bow shocks in Orion,
      observed with the HST Planetary Camera. The black bar indicates
      a length of \SI{1}{\arcsecond}, corresponding to 430 AU
      (\SI{2e-3}{pc}) \citep{Garcia-Arredondo:2001}. \textbf{B:}
      Examples of EGGs in Carina, observed with WFC3, ACS, and WFPC2.
      The white bars of \SI{1}{\arcsecond} correspond to a physical
      size of \SI{e-2}{pc} \citep{Mesa-Delgado:2016}. \textbf{C:} The
      dense globulette RN88 seen in \SI{}{H\alpha} with a diameter of
      \SI{6}{\arcsecond} (\SI{4e-2}{pc}) in the Rosette Nebula
      \citep{GFGahm:2013}. \textbf{D:} Examples of knots in the Helix
      Nebula. The mosaics cover \SI{47.5}{\arcsecond}$\times$
      \SI{44.8}{\arcsecond} (\SI{4.76e-2}{pc}$\times$\SI{4.49e-2}{pc})
      \citep{O'Dell:2007}.}
    \label{fig:nudos}
\end{figure}

% \section{Flujos de fotoevaporación ionizada} \label{Sec:fluijos fotoevaporativos}

% Todos los ejemplos de las Figuras \ref{fig:Banard}, \ref{fig:Pillars}
% y \ref{fig:nudos} se encuentran ya sea en regiones de formación
% estelar o en nebulosas alrededor de estrellas evolucionadas. Lo
% interesante en todos estos ejemplos es la forma que toman al
% interaccionar con las estrellas más masivas que se encuentran cerca,
% esto para los glóbulos que se encuentran en regiones de formación
% estelar. Mientras que los que se encuentran en nebulosas planetarias
% interactúan con la estrella evolucionada. Durante estas interacciones,
% en algunos casos podemos ver lo que se conoce como \textit{flujos
%   fotoevaporativos}, los cuales explicaremos mejor a continuación.

% Cuando la radiación ionizante incide en la superficie del glóbulo,
% este comienza a ionizar el gas neutro. A este flujo de gas ionizado
% que sale de la base del glóbulo y que viaja en dirección a la fuente
% ionizante se le conoce como flujo fotoevaporativo.

% En el caso de las regiones de formación estelar podemos considerar una
% estrella masiva y una nube densa de gas neutro. Por lo que para poder
% ver el flujo fotoevaporativo es necesario que la estrella sea masiva,
% o que tenga un gran flujo ionizante como para poder ionizar el gas
% neutro, de lo contrario no podremos ver el flujo fotoevaporativo.
% Recordemos que en las regiones de formación estelar hay muchas
% estrellas nuevas de baja masa que emiten principalmente en radio o
% infrarrojo, por lo que no todas las estrellas nuevas pueden ionizar el
% gas neutro.

% \cite{OortySpitzer_1955} explican de manera detallada como es la
% interacción entre una estrella tipo O y una nube interestelar de gas
% neutro. La cual se puede observar en regiones de formación estelar
% masiva. Ellos consideran tres elementos importantes para esto: La
% estrella ionizante, la nube interestelar de gas neutro y la región que
% hay entre la estrella y la nube interestelar. La nube interestelar
% debe ser mucho más densa y fría que la región que hay entre la
% estrella y la nube como vemos en la Figura \ref{kahn_zones}.

% \begin{figure}[htb]
%     \centering
%     \includegraphics[width= \textwidth]{artesanales/ImgFi01-5.pdf}
%     \caption{Esquema inicial utilizado en \cite{OortySpitzer_1955},
%       donde podemos apreciar que la nube es más fría y densa que la
%       región que hay entre la nube y la estrella.}
%     \label{kahn_zones}
% \end{figure}

% Cuando la radiación UV comienza a calentar el gas de la nube, el gas
% ionizado comienza a expandirse en dirección a la estrella, esto ya que
% en esta dirección la densidad es menor que la de la nube y puede
% expandirse libremente (Figura \ref{fig:evolucion de la nube}).

% \begin{figure}[htb]
%     \centering
%     \includegraphics[width= \textwidth]{ultimos/implosion_inicial.pdf}
%     \caption{Esquema de la implosión inicial. Cuando la radiación
%       ionizante (flechas azules) inciden en el glóbulo (color verde)
%       produce que el gas ionizado (color naranja) viaje en dirección a
%       la estrella como lo muestran las flechas negras. En esta fase se
%       produce un choque interno que hará que la nube se comprima.
%       Tanto el choque como el frente de ionización (color rojo) viajan
%       hacia el centro del glóbulo como lo muestran las flechas
%       amarillas.}
%     \label{fig:evolucion de la nube}
% \end{figure}

% En un inicio esta radiación ioniza el gas neutro de la nube a una tasa
% muy rápida. Esto causa que una gran cantidad partículas ionizadas,
% provenientes de la nube, viajen en dirección a la estrella. Conforme
% esto va evolucionando se va formando una capa aislante alrededor de la
% nube \citep{OortySpitzer_1955}. Esta capa aislante esta conformada por
% el gas ionizado, y puede proteger a la nube de flujos o vientos
% externos, así como de la radiación.

% Durante esta interacción tenemos tanto un frente de ionización como un
% choque interno que viajan a través de la nube a la parte trasera (ver
% Figura \ref{fig:evolucion de la nube}). Al inicio estos dos tienen una
% velocidad similar de $\sim\SI{10}{km.s^{-1}}$, pero una vez que las
% recombinaciones se vuelven importantes en la capa aislante, el frente
% de ionización comienza a desacelerar. Mientras que el choque interno
% hace que la nube se comprima \citep{Bertoldi_1989}.

% No siempre podemos ver un flujo fotoevaporativo por parte de las nubes
% en este tipo de interacción, para esto \cite{Bertoldi_1989} nos dice
% que si el parámetro de ionización, definido como
% \begin{equation}
%     \Gamma  \equiv \frac{F_\mathrm{i}}{n_0 c}
% \end{equation}
% donde $F_\mathrm{i}$ es el flujo incidente del continuo de Lyman,
% $n_0$ la densidad del gas neutro, y $c$ la velocidad de la luz, es
% menor que \SI{e-7}{}, entonces la radiación ionizante incidente no
% tendrá un efecto dinámico sobre la nube por lo que no tendremos un
% flujo fotoevaporativo por parte de la nube. Por otro lado, si
% \begin{equation}
%     \delta'\equiv\frac{F_\mathrm{i}}{2\alpha_\mathrm{i} r_0 n_0^2}>1
% \end{equation}
% donde $\alpha_\mathrm{i}$ es el coeficiente de recombinación a todos los
% estados, excepto al nivel base y $r_0$ el radio de la nube, entonces
% la nube se ionizará por completo, esto ya que el flujo ionizante es
% mayor que las recombinaciones.

\section{Ionized photoevaporative flows} 
\label{Sec:fluijos fotoevaporativos}

All the examples in Figures \ref{fig:Banard}, \ref{fig:Pillars}, and
\ref{fig:nudos} occur either in regions of star formation or in
nebulae around evolved stars. What is interesting in all these cases
is the way they interact with the most massive stars in their
vicinity, for globules found in star-forming regions, or with the
evolved central star in planetary nebulae. During these interactions,
in some cases we can see what are known as \textit{photoevaporative
  flows}, which we explain in more detail below.

When ionizing radiation strikes the surface of a globule, it begins to
ionize the neutral gas. The resulting flow of ionized gas that streams
away from the base of the globule and travels toward the ionizing
source is known as a photoevaporative flow.

In the case of star-forming regions, we can consider a massive star
and a dense cloud of neutral gas. In order to observe a
photoevaporative flow, the star must be massive, or must have a large
ionizing flux capable of ionizing the neutral gas. Otherwise, the
photoevaporative flow will not be visible. Recall that in star-forming
regions there are many new low-mass stars that emit mainly in radio or
infrared, so not all young stars are capable of ionizing the neutral
gas.

\cite{OortySpitzer_1955} give a detailed explanation of the
interaction between an O-type star and a cloud of neutral interstellar
gas, as observed in regions of massive star formation. They consider
three key elements: the ionizing star, the neutral interstellar cloud,
and the region between the star and the cloud. The interstellar cloud
must be much denser and colder than the intervening region, as shown
in Figure \ref{kahn_zones}.

\begin{figure}[htb]
    \centering
    \includegraphics[width= \textwidth]{artesanales/ImgFi01-5.pdf}
    \caption{Initial schematic used in \cite{OortySpitzer_1955}, in
      which the cloud is colder and denser than the region between the
      cloud and the star.}
    \label{kahn_zones}
\end{figure}

When UV radiation begins to heat the gas in the cloud, the ionized gas
expands toward the star, since in that direction the density is lower
than in the cloud and the gas can expand freely (see Figure
\ref{fig:evolucion de la nube}).

\begin{figure}[htb]
    \centering
    \includegraphics[width= \textwidth]{ultimos/implosion_inicial.pdf}
    \caption{Schematic of the initial implosion. When ionizing
      radiation (blue arrows) strikes the globule (green), it causes
      ionized gas (orange) to flow toward the star, as shown by the
      black arrows. In this phase an internal shock is produced that
      compresses the cloud. Both the shock and the ionization front
      (red) travel toward the center of the globule, as indicated by
      the yellow arrows.}
    \label{fig:evolucion de la nube}
\end{figure}

At first this radiation ionizes the neutral gas of the cloud at a very
rapid rate. As a result, a large number of ionized particles, coming
from the cloud, flow toward the star. As the process evolves, an
insulating layer forms around the cloud \citep{OortySpitzer_1955}.
This insulating layer is composed of ionized gas and can protect the
cloud from external flows or winds, as well as from radiation.

During this interaction there are both an ionization front and an
internal shock that travel through the cloud toward the rear (see
Figure \ref{fig:evolucion de la nube}). Initially, these two have a
similar velocity of $\sim\SI{10}{km.s^{-1}}$, but once recombinations
in the insulating layer become important, the ionization front begins
to decelerate. Meanwhile, the internal shock compresses the cloud
\citep{Bertoldi_1989}.

It is not always possible to observe a photoevaporative flow from
clouds in this type of interaction. For this,
\cite{Bertoldi_1989} note that if the ionization parameter, defined as
\begin{equation}
    \Gamma  \equiv \frac{F_\mathrm{i}}{n_0 c}
\end{equation}
where $F_\mathrm{i}$ is the incident flux of Lyman continuum photons,
$n_0$ is the density of neutral gas, and $c$ is the speed of light, is
less than \SI{e-7}{}, then the incident ionizing radiation will have
no dynamical effect on the cloud, and no photoevaporative flow will be
produced. On the other hand, if
\begin{equation}
    \delta'\equiv\frac{F_\mathrm{i}}{2\alpha_\mathrm{i} r_0 n_0^2}>1
\end{equation}
where $\alpha_\mathrm{i}$ is the recombination coefficient to all
states except the ground state, and $r_0$ is the radius of the cloud,
then the cloud will be completely ionized, since the ionizing flux is
greater than the recombinations.


\section{Wolf--Rayet stars and their winds}

Wolf--Rayet (WR) stars are the evolved descendants of massive stars,
such as O-type stars. These WR stars typically have masses of
10--\SI{25}{\msun} and are characterized by intense emission lines and
free--free emission at IR--mm--cm wavelengths \citep{crowther:2007}.
They also have high mass-loss rates, $\sim$2--\SI{10e-5}{\msun/yr},
driven by their strong stellar winds, which can reach velocities of
$\sim$\SI{1000}{km/s}, producing their broad emission lines
\citep{Hamman:2006}. They were named after Charles Wolf and Georges
Rayet, who first identified three stars in Cygnus with broad emission
lines of C, N, O, and He---the characteristic features of this class
\citep{WR:ref}.

These stars are classified according to the relative strengths of
their characteristic emission lines. \cite{VanDerHutch:2001} classify
them mainly as type WN when He and N are abundant, type WC when He and
C dominate, and type WO when He and O are abundant. Although many of
these stars show no hydrogen in their atmospheres, in some cases a
significant amount of H is detected, in which case the designation ``h''
is added \citep{SSM:1996}.

\section{The nebula M1-67}

M1-67 is the circumstellar nebula around the WR~124 star, which is of
type WN8h. Several studies have been carried out on the M1-67 nebula,
including three-dimensional models of its structure based on long-slit
spectroscopy \citep{Zavala:2022}. \cite{Marcel:2021} used the
$[\mathrm{S \scriptstyle{II}}]\lambda6717$ and $\lambda6731$ lines to
find that the electron density decreases with radius: close to the
star the electron density is $\sim$\SI{2000}{cm^{-3}}, while farther
away, at about \SI{40}{\arcsecond} (\SI{1.05}{pc}), it drops to
$\sim\SI{500}{cm^{-3}}$. \cite{Grosdidier:1998} found that the
H$\alpha$ surface brightness also decreases with radius as $r^{-0.8}$.
\cite{Mancherko:2010} measured an expansion velocity of
\SI{46}{km.s^{-1}} for the nebula using observations from 1997
\citep{Grosdidier:1998} and 2008, consistent with the measurement of
\cite{Zavala:2022}.

Figures \ref{fig:M1-67HST} and \ref{fig:M1-67JWST} show that this
nebula has a very complex structure. \cite{Grosdidier:1998} detected
some very bright and dense knots, but their nature was not clear. In
Chapter \ref{Chapter : 3} we will discuss in detail how these bright
and dense points are in fact globules located throughout much of the
nebula. These globules were revealed thanks to images from the James
Webb Space Telescope (JWST), which has higher resolution than the
Hubble Space Telescope (HST), and also offers a much greater variety
of filters.

Throughout this thesis we will use the data in Table
\ref{tab:parametros WR-124}. With the distance to the star, $D$, we
can derive physical distances as
\begin{equation}
    \left[\frac{R}{\mathrm{AU}}\right]=\left[\frac{D}{\mathrm{pc}}\right]\left[\frac{\theta}{\mathrm{arcsec}}\right]
\end{equation}
where $R$ is the desired physical distance and $\theta$ is the
separation measured directly from the observations in arcseconds. The
mass-loss rate, $\dot{M}$, and the terminal velocity of the stellar
wind, $v_\infty$, allow us to calculate the hydrodynamic (RAM)
pressure of the stellar wind, while the rate of ionizing photons from
the star is used to calculate the radiation pressure.

\begin{table}[htb]
    \centering
    \begin{tabular}{c c c}
        \toprule
        \multicolumn{3}{c}{Parameters of WR~124} \\ \midrule
         $D$ & 5.429$\pm$\SI{.54}{kpc} & J. Arthur, priv.~comm.\\
         $v_\infty$ & \SI{710}{km/s}  & \cite{Hamman:2006}\\
         $\dot{M}$ & $10^{-4.7}$\unit{M_\odot/yr}  & \cite{Crowther:1999}\\
         $S_*$ & \SI{1.25e49}{s^{-1}} & \cite{crowther:2007}  \\
         %$F_{H_\alpha}$ & \SI{3e-14}{erg.cm^{-2}.s^{-1}} & \cite{Grosdidier:1998}\\
         \bottomrule
    \end{tabular}
    \caption{Parameters of WR~124.}
    \label{tab:parametros WR-124}
\end{table}

\subsection{HST observations}

For the Hubble Space Telescope (HST) observations we used data from
the Hubble Legacy Archive. These are FITS images at calibration level
3, meaning they are mosaics created by combining multiple images to
cover a region of the sky\footnote{For more details see
  \url{https://hla.stsci.edu/hla_faq.html\#productlevels}}. We used
the F656N filter\footnote{Appendix \ref{App: Filtros} shows the range
  covered by this filter.} with the Wide Field and Planetary Camera 2
(WFPC2) to observe \unit{H\alpha} emission. We used the 1997 data from
proposal ID 6787, with a total exposure time of \SI{10 216}{s} from a
combination of 10 exposures. The 2008 data from proposal ID 11137 have
a total exposure time of \SI{4200}{s} from a combination of 8
exposures.

\begin{figure}[htb]
    \centering
    \includegraphics[width=0.9\textwidth]{ultimas correcciones/WR124_HST.pdf}
    \caption{Image of M1-67 in the \unit{H\alpha} filter
      \citep{Grosdidier:1998}. The \SI{30}{\arcsecond} scale
      corresponds to \SI{0.78}{pc}.
      https://esawebb.org/images/weic2307f/. The image has been
      rotated so that North is up and East is to the left.}
    \label{fig:M1-67HST}
\end{figure}

\subsection{JWST observations}

For the James Webb Space Telescope (JWST) observations we used data
obtained by Klaus M. Pontoppidan, proposal ID 2730. We used the NIRCam
filters F090W, F150W, F210M, F335M, F444W, and
F470N\footnote{Appendix \ref{App: Filtros} shows the wavelength range
  of each of these filters.}, with a total exposure time of
\SI{2662.72}{s}. These are level 3 images---mosaics combining multiple
exposures to cover a region of the sky---in FITS format.

The wide variety of JWST filters allows us to combine them to observe
different emission mechanisms, and its high resolution makes it
possible to resolve the detailed structures.

Unlike the H$\alpha$ image, the JWST observations use broad-band
filters. These bands include contributions from different emission
mechanisms: stellar continuum, some nebular emission lines, continuum
scattered by dust, thermal dust emission, and also emission from
polycyclic aromatic hydrocarbon (PAH) bands.

\begin{figure}[htb]
    \centering
    \includegraphics[width=0.9\textwidth]{ultimas correcciones/WR124_JWST.pdf}
    \caption{Image of M1-67 with JWST. Composite of filters f444w
      (gray), f335m (red), f210m (green), f150w (turquoise green), and
      f090w (blue). The \SI{30}{\arcsecond} scale corresponds to
      \SI{0.78}{pc}.
      https://www.flickr.com/photos/geckzilla/52757287572/. The image
      has been rotated so that North is up and East is to the left.}
    \label{fig:M1-67JWST}
\end{figure}

\section{Structure of the thesis}

In this thesis we propose a simple model to explain how the transonic
photoevaporative flow from a globule interacts with an external
pressure. We will apply this model to the knots in the M1-67 nebula,
and describe how the photoevaporative flow interacts with the
hydrodynamic (RAM) pressure of the stellar wind of WR~124.

In Chapter 2 we will see how the interaction between two supersonic
flows creates a thin shocked shell. Based on this, we describe a
steady-state hydrodynamic model in which the photoevaporative flow of
a globule interacts with an external pressure. In this interaction a
shocked shell can also be seen.

In Chapter 3 we will describe how we identified the knots in the M1-67
nebula, as well as observational evidence of the interaction between
the photoevaporative flow of the knots and the stellar wind of WR~124.

In Chapter 4 we will apply this model to the knots found in the
nebula. We will fit the radial brightness profiles in order to measure
the knots and their shocked shells. In addition, we can derive the
density of the ionized gas from the emission measure.

In Chapter 5 we will compare these results with the theoretical values
predicted by the proposed model. This comparison provides a clearer
understanding of the distribution of the knots in the nebula.

\chapter{Modelos analíticos de flujos fotoevaporativos interactuando
  con una presión externa}
\chaptermark{Modelo analítico}\label{Chapter : Modelo}

En este capítulo vamos a describir el modelo que se propone para
explicar la interacción que hay entre el flujo fotoevaporativo de un
glóbulo y una presión externa. Esta presión externa puede ser ejercida
por la misma estrella que esta foto evaporando al glóbulo. Este modelo
en principio se puede aplicar a cualquier tipo de glóbulo como los que
se mencionaron en el Capítulo \ref{Capitulo 1:introduccion}.

En este trabajo en especial vamos a tratar la interacción del flujo
fotoevaporativo de los glóbulos\footnote{Llamaremos glóbulos a los
nudos que hay en la nebulosa por simplicidad.} en la nebulosa M1-67 y
la presión RAM por parte del viento estelar de la estrella WR 124. En
el Capítulo \ref{Chapter : 3} hablaremos más acerca de cómo
encontramos estos glóbulos en la nebulosa M1-67, por ahora nos
enfocaremos solo en el modelo.

Para esto hemos considerado que ya han pasado todas las fases
mencionadas en la Sección \ref{Sec:fluijos fotoevaporativos} y ahora
estamos en un equilibrio de ionización. La forma del glóbulo en este
modelo será esférico por simplicidad.

En esta interacción entre el flujo fotoevaporativo y el viento
estelar, los cuales son supersónicos, se producen dos zonas chocadas y
entre estas dos zonas una discontinuidad de contacto como se describe
en la Figura \ref{fig:zones}. De estas zonas esperamos ver solo el
flujo fotoevaporativo chocado y no el viento estelar chocado ya que
este último es menos denso además de que es no radiativo y su longitud
de enfriamiento (zona 3) es más grande que la región de interacción
(zona 2).

\begin{figure}[htb]
    \centering    \includegraphics[width=\textwidth]{imagenes_corregidas/Arreglo 01.pdf}
    \caption{La interacción entre el flujo fotoevaporativo y el viento
      estelar de una estrella forma 4 zonas. El círculo naranja es el
      glóbulo. La línea punteada azul que une el centro del glóbulo
      con el centro de la estrella es el eje de simetría que vamos a
      considerar en el modelo, la estrella se localiza en el otro
      extremo de la línea punteada. Vemos como en este eje de simetría
      tanto el viento estelar (líneas naranjas) como la radiación
      inciden de forma perpendicular a la base del glóbulo y en
      dirección contraria viaja el flujo fotoevaporativo (líneas
      rojas). La zona 1 es donde el flujo fotoevaporativo sale de la
      superficie del glóbulo con un número de Mach igual a 1 y va
      aumentando. La zona 2 es el flujo fotoevaporativo chocado con el
      viento estelar, la cual esperamos ver en las observaciones como
      una cáscara y nos vamos a referir a ella como la cáscara
      chocada. La zona 3 es el viento estelar chocado con el flujo
      fotoevaporativo y la zona 4 es donde viaja el viento estelar
      supersónico, el cual es menos denso que el flujo
      fotoevaporativo. La discontinuidad de contacto se da entre las
      zonas 2 y 3, la línea gris.}
    \label{fig:zones}
\end{figure}

\begin{figure}[htb]
    \centering    \includegraphics[width=0.8\textwidth]{images Chapter 2/C2_Canto.jpg}
    \caption{Interacción de dos flujos supersónicos las cuales son
      producidas por dos fuentes (puntos negro en el eje z) a una
      distancia $D$. En esta interacción se produce una cáscara
      delgada $R(\theta)$ cuando los flujos llegan a un equilibrio. Para
      este problema se considera simetría cilíndrica
      \citep{Canto:1996}.}
    \label{fig:Canto1}
\end{figure}

\begin{figure}[htb]
    \centering    \includegraphics[width=\textwidth]{artesanales/ImgFi01-2.pdf}
    \caption{La radiación y el viento estelar viajan hacia el glóbulo
      como lo indican las flechas azules y vemos que inciden de manera
      perpendicular a la superficie del glóbulo en el eje de simetría.
      Mientras que el flujo fotoevaporativo por parte del glóbulo
      viaja como lo indican las flechas rojas, y por lo tanto en la
      interacción entre este flujo fotoevaporativo y el viento estelar
      debemos considerar cierto ángulo si no estamos en el eje de
      simetría.}
    \label{fig:cilindross}
\end{figure}

\cite{Canto:1996} trata de una manera más detallada la interacción
entre dos flujos hipersónicos en la cual considera dos fuentes
separadas a una distancia $D$. En este análisis suponen que se forma
una cáscara delgada cuando estos dos flujos llegan a un equilibrio de
presiones como se puede ver en la Figura \ref{fig:Canto1}. En nuestro
caso también esperamos que la zona 2 de la Figura \ref{fig:zones} sea
delgada \citep{Wil:2019}.

\section{Modelo hidrodinámico estacionario}

Para nuestro modelo es importante mencionar que no estamos
considerando ninguna fuerza de gravedad por parte de la estrella o del
mismo glóbulo, así como tampoco ninguna otra fuerza externa (ver
Apéndice \ref{App:fuerzas}). Vamos a considerar que solo el glóbulo
está dominado por un campo magnético, mientras que para el gas
ionizado no vamos a considerar ningún campo magnético.

Para este trabajo en particular solo vamos a considerar como presión
externa la presión RAM del viento estelar por parte de la estrella WR
124. Tomando en cuenta que el tiempo en el que ocurren las fases
mencionadas en la Sección \ref{Sec:fluijos fotoevaporativos} es muy
corto comparado con el tiempo de interacción que hay entre el flujo
fotoevaporativo y el viento estelar, vamos a suponer que la capa
aislante producida por la radiación UV ya se ha formado y ahora
estamos en equilibrio de ionización. Por lo que vamos a considerar
este modelo como estacionario, es decir, que los tamaños del glóbulo y
de la cáscara chocada los vamos a tomar como constantes ya que no
cambiaran sus tamaños de manera significativa.

En la Figura \ref{fig:cilindross} vemos que podemos simplificar este
problema si ponemos un cilindro de radio pequeño alrededor del eje de
simetría. En este cilindro podemos ignorar los movimientos
transversales ya que los gradientes en estas direcciones son muy
pequeños si los comparamos con los gradientes en la dirección axial.
Alrededor del eje de simetría podemos ver como tanto la radiación UV y
el viento estelar inciden de manera perpendicular a la base del
glóbulo y viajan en dirección contraria el flujo fotoevaporativo por
parte del glóbulo. Por lo que primero vamos a resolver este problema
solo en el eje de simetría, ya que aquí se vuelve unidimensional. Más
adelante vamos a considerar qué pasaría si resolvemos este problema
considerando un ángulo $\theta$ con respecto al eje de simetría.

\section{Ecuación de estado y equilibrio de ionización}

Para este caso vamos a considerar que el gas que está interaccionando
con el viento estelar de la estrella es un gas ideal, por lo que
nuestra ecuación de estado será
\begin{equation}
    PV = Nk_\mathrm{B}T,
\end{equation} 
donde $P$ es la presión del gas, $V$ su volumen, $N$ es el número de
partículas, $k_\mathrm{B}$ la constante de Boltzman y $T$ es la
temperatura. De aquí tenemos que \begin{equation} P = nk_\mathrm{B}T =
  \frac{\rho k_\mathrm{B} T}{\bar{m}}
\end{equation}
donde $n$ es la densidad numérica, $\rho$ la densidad de masa y
$\bar{m}$ es la masa promedio de las partículas en el gas. En este
caso estamos considerando que el gas contiene principalmente hidrógeno
un $90\%$ de todo el gas y una pequeña fracción de helio, por lo que
vamos a considerar una masa promedio de
$\bar{m}=\frac{0.9 m_\mathrm{p}+0.1\times4m_\mathrm{p}}{2}\approx0.6
m_{\mathrm{p}}$ en el gas ionizado.

En este gas ideal vamos a tomar la velocidad del sonido isotérmica, la
cual está dada por
\begin{equation}
    c_\mathrm{s}  = \sqrt{\frac{k_\mathrm{B} T}{\bar{m}}}
\end{equation} 
de esta manera vemos que la velocidad del sonido en el gas ionizado
solo depende de su temperatura. En este tipo de gases la velocidad del
sonido es típicamente del orden de \SI{e6}{cm.s^{-1}}.

Considerando que todas las fases mencionadas en la Sección
\ref{Sec:fluijos fotoevaporativos} ya han sucedido, por lo que vamos a
considerar que el flujo incidente de fotones ionizantes, $F_0$ está en
equilibrio con las recombinaciones por unidad de área
\begin{equation}
    F_0 = n_\mathrm{e}n_\mathrm{i}h_1\alpha_\mathrm{B}
\end{equation} 
donde $h_1$ es la anchura efectiva y $\alpha_\mathrm{B}$ es el coeficiente
de recombinación para el caso B (ver Apéndice \ref{App : tasa de
  fotoionizacion}). Este coeficiente de recombinación para el caso B
toma en cuenta las recombinaciones a todos los niveles, excepto al
nivel base, esto ya que el fotón liberado en esta recombinación puede
volver a ionizar algún otro átomo que se encuentre muy cerca
\citep{Dyson:book:1980}.

\section{Estructura del flujo fotoevaporativo}\label{Estructura}

En este modelo estamos considerando un frente de ionización D-crítico
\citep{Shuu:1992}, esto es, dentro del glóbulo vamos a tener una
expansión subsónica del gas desde el punto de vista del frente de
ionización y una expansión supersónica en el gas ionizado desde el
punto de vista del frente de ionización, por lo que tendremos un punto
sónico el cual está justo detrás del frente de ionización. En este
caso por simplicidad vamos a considerar el frente de ionización como
una discontinuidad en el que pasamos de tener gas no ionizado a tener
un gas totalmente ionizado, por lo que tomaremos que el punto sónico
se da justo en $r_0$, que es el radio del glóbulo (ver Figura
\ref{fig:parameters}). Así que vamos a considerar que el gas tiene un
número de Mach igual a 1 en $r_0$ y este va a ir aumentando conforme
atraviesa toda la zona 1 de la Figura \ref{fig:zones} ya que se va
expandiendo libremente. En principio podríamos considerar que tanto el
radio del glóbulo como la densidad en su superficie son parámetros
libres, pero con las observaciones podemos medir el radio y la
densidad la podemos calcular ya que esta debe ser consistente por
haber considerado equilibrio de ionización.

Con este modelo se pretende ver hasta donde llegamos a un equilibrio
de presión entre la presión por parte del flujo fotoevaporativo y la
presión RAM del viento estelar. Para el caso de la presión del flujo
fotoevaporativo vamos a considerar tanto la presión térmica como la
hidrodinámica, por lo que la presión total en la base del glóbulo está
dada por
\begin{equation}\label{eq: Presion total}
    P_\mathrm{tot}=P_\mathrm{ter}+P_\mathrm{hid}=n\bar{m}c_\mathrm{s}^2+n\bar{m}u^2=\rho c_\mathrm{s}^2(1+M^2),
\end{equation}
donde $u$ es la velocidad del gas en la parte ionizada y $M$ es el
número de Mach.

\begin{figure}[htb]
    \centering \includegraphics[width=\textwidth]{imagenes_corregidas/Arreglo 02_n.pdf}
    \caption{En este diagrama esquemático $r_0$ es el radio del
      glóbulo neutro, mientras que $\rho_0,M_0$ y $P_0$ son los valores
      que tenemos en la superficie del glóbulo y varían con dirección
      a la estrella hasta tener diferentes valores en
      $r_\mathrm{shell}$, que es el radio del centro del glóbulo hasta
      la base del flujo fotoevaporativo chocado, $\rho,M$ y $P$ son los
      valores que tendrán en la base del flujo fotoevaporativo
      chocado.}
    \label{fig:parameters}
\end{figure}

Este equilibrio de presión se logrará a un radio $r_\mathrm{shell}$
que es donde la presión del flujo fotoevaporativo ha disminuido una
fracción $f$ de lo que era la presión inicial. Por lo que la presión
cambia como
\begin{equation}\label{eq : 1}
f=\frac{P}{P_0}=\frac{\rho c_\mathrm{s}^2(1+M^2)}{\rho_0 c_\mathrm{s}^2(1+1)}=\frac{\rho}{\rho_0}\frac{1+M^2}{2},
\end{equation}
donde $P_0$ es la presión en la base del glóbulo, recordemos que aquí
estamos considerando el punto sónico, por lo que $M_0=1$. $P$ es la
presión del flujo fotoevaporativo justo antes de $r_\mathrm{shell}$.
Considerando la ecuación para la conservación de masa tenemos que
\begin{equation}\label{eq : 2}
\rho r^2M	=\rho_0 r_0^2.
\end{equation}
Finalmente, si consideramos la ecuación de Bernoulli isotérmica 
\begin{equation}
\frac{v^2}{2}+c_\mathrm{s}^2\ln\rho=\text{constante}
\end{equation}
en la Ecuación (\ref{eq: Presion total}) tenemos que \citep{Dyson:1968}
\begin{equation}\label{eq ; 3} \frac{r}{r_0}=M^{-1/2}e^{\frac{M^2-1}{4}}.
\end{equation}
Combinando las ecuaciones (\ref{eq : 1}), (\ref{eq : 2}) y (\ref{eq ;
  3}) tenemos la ecuación
\begin{equation}
    f=\frac{1+M^2}{2}exp\left(\frac{1-M^2}{2}\right),
\end{equation}
la cual solo depende de $M$ y podemos resolver de manera
numérica\footnote{En nuestro caso usamos la función fsolve de la
  paquetería scipy.optimize, la documentación se encuentra en
  \url{https://docs.scipy.org/doc/scipy-1.12.0/reference/generated/scipy.optimize.fsolve.html}}
dándole diferentes valores a $f$. Una vez que resolvemos esta ecuación
podemos saber los valores de las incógnitas $\rho/\rho_0$ y $r/r_0$ usando
las ecuaciones (\ref{eq : 2}) y (\ref{eq ; 3}). Obteniendo así, que
tanto la presión como la densidad decaen con el radio, mientras que el
número de Mach aumenta como vemos en la Figura \ref{fig:grafica_C2}.
Cabe mencionar que la Figura \ref{fig:grafica_C2} se obtiene
resolviendo las ecuaciones (\ref{eq : 1}), (\ref{eq : 2}) y (\ref{eq ;
  3}) en el eje de simetría. Pero esto cambiaría si consideramos
cierto ángulo $\theta\neq 0 $ ya que tanto la presión como la densidad escalan
con el ángulo como $\cos^{1/2}\theta$, pero $M$ no \citep{Tarango:2018}.

\begin{figure}[htb]
    \centering    \includegraphics[width=\textwidth]{Nuevas imagenes finales/C2_estructura.pdf}
    \caption{Gráfica del número de Mach $M$, densidad $\rho$ y la presión
      total $P$ normalizados como función de $r/r_0$. También podemos
      ver una caída que va como $\sim r^{-2}$ (línea negra). Podemos
      notar que la densidad cae más rápido que $r^{-2}$, mientras que
      la presión total cae más lento, en ambos casos se debe a la
      aceleración del flujo. }
    \label{fig:grafica_C2}
\end{figure}

% Assumption of isothermal equation of state
\section{Condiciones para la cáscara chocada}

\cite{Canto:1996} en su descripción para interacción de dos flujos
hipersónicos da la solución a distintos parámetros
$\beta=(\dot{M}^0_\mathrm{w}
v_\mathrm{w})/(\dot{M}^0_\mathrm{w1}v_\mathrm{w1})$, el cual es la
razón de los momentos de los flujos $\mathrm{w}$ y $\mathrm{w1}$. Esto
se puede ver en la Figura \ref{fig:Canto2} en la cual notamos que
cuando el momento de un flujo es muy grande comparado con el otro, la
cáscara chocada se vuelve muy curva y está más cerca de la fuente cuyo
flujo tiene menor momento.

\begin{figure}[htb]
    \centering    \includegraphics[width=0.8\textwidth]{images Chapter 2/C2_Canto2.jpg}
    \caption{Formas de las distintas cáscaras chocadas a diferentes
      parámetros $\beta$. La línea vertical en z/D=0.5 corresponde a un
      parámetro $\beta=1$ y las demás curvas corresponden a valores de
      0.5, 0.25, 0.125, 0.0625 y 0.03125, entre más pequeño es $\beta$ más
      curva se vuelve la cáscara chocada. La otra fuente se encuentra
      en z/D=1 \citep{Canto:1996}.}
    \label{fig:Canto2}
\end{figure}

En la Figura \ref{fig:zones_presiones} vemos los diferentes tipos de
presiones que hay en cada zona en esta interacción entre flujos
supersónicos. En la zona de discontinuidad, la línea gris en la Figura
\ref{fig:zones_presiones}, será la suma de la presión RAM del viento
estelar, $P_\mathrm{RAM}$, y de la presión térmica del viento estelar.
Dado que los vientos estelares en estrellas Wolf-Rayet llegan a ser
del orden de \SI{1e3}{km.s^{-1}} podríamos llegar a tener un número de
Mach del orden de 100 para el viento estelar, por lo que vamos a
considerar que $P_\mathrm{DC}=P_\mathrm{RAM}$. Por otro lado, para la
cáscara chocada, la zona 2 de la Figura \ref{fig:zones}, tenemos que
esta debe estar en doble equilibrio de presión. Por un lado, la
presión de la cáscara chocada, $P_\mathrm{shell}$ debe ser igual a la
presión del flujo fotoevaporativo justo antes del choque, es decir,
$P_\mathrm{shell}=P$. Por el otro lado, la presión de la cáscara
chocada debe ser igual a la presión RAM del viento estelar chocado.

\begin{figure}[htb]
    \centering    \includegraphics[width=\textwidth]{imagenes_corregidas/Arreglo 03.pdf}
    \caption{La base del glóbulo está dominada por la presión total,
      ecuación (\ref{eq: Presion total}). En la línea roja, que es la
      parte más interna de la cáscara chocada, tenemos que domina la
      presión térmica del flujo fotoevaporativo $P$. En la línea
      naranja domina la presión RAM del viento estelar
      $P_\mathrm{RAM}$, la cual es la única presión externa que
      estamos considerando. En la zona de discontinuidad tenemos una
      presión $P_\mathrm{DC}$, la cual es la suma de la presión RAM
      del viento estelar y la presión térmica del viento estelar
      chocado. }
    \label{fig:zones_presiones}
\end{figure}
% General solution for the internal structure of model




\chapter{Glóbulos en la nebulosa M1-67} \label{Chapter : 3}

Ahora vamos a describir cómo encontramos estos glóbulos en la nebulosa
M1-67. En la Figura \ref{fig:nudos WR124} vemos del lado derecho cómo
estos glóbulos los podemos identificar de manera visual por tres
componentes importantes. El glóbulo lo vemos como un círculo de color
blanco, su estela de color rosa que se encuentra justo detrás del
glóbulo en dirección opuesta a la estrella, y su cáscara chocada que
se ve de color gris. De esta manera, se pudieron localizar visualmente
168 glóbulos en total.

De las observaciones vemos que estos glóbulos tienen tamaños típicos
de 200--300 mili segundos de arco (5--\SI{7e-3}{pc}) de diámetro que son
relativamente pequeños si los comparamos con la nebulosa circunestelar
que es de unos $\sim\SI{60}{\arcsecond}$ (\SI{1.57}{pc}).

\begin{figure}[htb]
    \centering
    \includegraphics[width=\textwidth]{ultimas correcciones/WR124_glo_ej.pdf}
    \caption{En esta imagen capturada con el JWST vemos la nebulosa
      M1-67 y en ella la presencia de glóbulos en casi toda la
      nebulosa. Haciendo zoom en estos glóbulos (la emisión en color
      blanco y rosa), vemos como puede ser su morfología en la parte
      neutra debido a la emisión de PAHs. Mientras que en color gris
      vemos lo que parece ser su interacción con la estrella y su
      viento estelar. Las figuras A1, A2, B1 y B2 tienen medidas de
      $\SI{11.86}{\arcsecond}\times\SI{15.65}{\arcsecond}$
      ($\SI{0.31}{pc}\times\SI{.41}{pc}$),
      $\SI{2.94}{\arcsecond}\times\SI{5.76}{\arcsecond}$
      ($\SI{0.07}{pc}\times\SI{0.15}{pc}$),
      $\SI{15.82}{\arcsecond}\times\SI{19.21}{\arcsecond}$
      ($\SI{0.41}{pc}\times\SI{0.5}{pc}$) y
      $\SI{2.42}{\arcsecond}\times\SI{4.53}{\arcsecond}$
      ($\SI{0.06}{pc}\times\SI{0.11}{pc}$), respectivamente.}
    \label{fig:nudos WR124}
\end{figure}

A estos glóbulos los podemos localizar por su posición angular y
separación con respecto a la estrella. La posición angular es el
ángulo $\phi$ que tiene con respecto a la línea roja de la Figura
\ref{fig:ejemplo_PA_Sep}, esta línea roja representa la posición
angular a $0^\circ$, y el ángulo $\phi$ es tomado en sentido contrario a las
manecillas del reloj. La separación entre el glóbulo y la estrella es
medido directamente de las observaciones. En la Figura
\ref{fig:ejemplo_PA_Sep} vemos un ejemplo de como podemos conocer la
posición angular y separación para cada glóbulo.

\begin{figure}[htb]
    \centering
    \includegraphics[width=\textwidth]{ultimas correcciones/M167_PA.pdf}
    \caption{En esta imagen vemos un ejemplo de como obtuvimos la
      posición angular y la separación con respecto a la estrella. El
      círculo azul es el glóbulo que tomaremos como ejemplo, el ángulo
      $\phi$ con respecto a la línea roja es la posición angular y es
      tomada en sentido anti horario, la separación es la longitud de
      la línea amarilla, la cual une al glóbulo (círculo azul) con la
      estrella, la cual se encuentra en el centro de la imagen. Los
      puntos verdes marcan donde se localizan los otros glóbulos.}
    \label{fig:ejemplo_PA_Sep}
\end{figure}

En la Figura \ref{fig:dis_nudos} vemos del lado izquierdo la
localización de los glóbulos en la nebulosa, donde los puntos verdes
son los glóbulos que encontramos y la línea roja es nuestra referencia
para encontrar su posición angular. En el lado derecho vemos como
estos se distribuyen considerando su separación y posición angular. En
estas distribuciones podemos encontrar ciertas simetrías, por ejemplo,
en el histograma superior de la imagen izquierda vemos que muchos de
estos glóbulos se concentran en dos posiciones angulares en
particular, a $\sim90^\circ$ y $\sim200^\circ$, mientras que a
$120^\circ$ y $300^\circ$ no tenemos casi glóbulos. Por otro lado, en el
histograma del lado derecho vemos cómo en separación pareciera haber
una distribución bimodal con picos a \SI{11}{\arcsecond}
(\SI{.28}{pc)} y \SI{17}{\arcsecond} (\SI{.44}{pc}). Esta distribución
nos dice que la cantidad de glóbulos decae con la distancia de manera
no constante, mientras que los grupos de glóbulos encontrados parecen
ser los causantes de los picos de la distribución.

\begin{figure}[htb]
    \centering  
    \begin{subfigure}[b]{0.45\linewidth}
        \includegraphics[width=\textwidth]{ultimas correcciones/M167_glo.pdf}
    \end{subfigure}
    \begin{subfigure}[b]{0.45\linewidth}
        \includegraphics[width=\textwidth]{images Chapter 2/C2_nudos_distribucion.png}
    \end{subfigure}
    \caption{En la imagen izquierda vemos en puntos verde la
      localización de los glóbulos en la nebulosa M1-67 y la línea
      roja es la que usamos para conocer la posición angular de cada
      glóbulo. En la imagen derecha vemos como es la distribución de
      estos glóbulos (puntos azules) en separación y posición angular.
      En la parte superior de la imagen derecha vemos el histograma de
      la posición angular y del lado derecho el histograma de la
      separación de los glóbulos. Las líneas rojas son los
      isocontornos de la densidad suavizada de la distribución de los
      puntos.}
    \label{fig:dis_nudos}
\end{figure}

En la Figura \ref{fig:filters WR124} vemos una gran variedad de
glóbulos en una zona pequeña y sus diferentes morfologías que
presentan en los diferentes filtros.

Como ya hemos mencionado anteriormente, la interacción entre el flujo
fotoevaporativo de los glóbulos y el viento estelar nos produce una
cáscara chocada, la cual podemos ver debido a las recombinaciones, por
lo que la podemos ver como emisión en algunos filtros. Por ejemplo, en
el filtro f656n del HST y en los filtros f090w y f444w del JWST
podemos ver esta cáscara chocada que rodea los glóbulos. También se
realizó una combinación de filtros para poder ver solo la emisión de
gas ionizado (ver Apéndice \ref{AP: combos}). Por lo que en la imagen
de gas ionizado tenemos una imagen más clara del gas ionizado que se
encuentra en la cáscara chocada.

En el filtro f1130w del JWST se puede observar lo que parecen ser
parte de las estelas de los glóbulos. Algunos parecen ser más pequeños
que otros. También se pude notar que en algunos casos cuando los
glóbulos están muy cerca los unos de los otros sus estelas parecen
juntarse, incluso en algunos casos pareciera que estas estelas están
interaccionando con otros glóbulos. En este filtro también se puede
ver de manera muy tenue lo que pareciera ser la interacción del flujo
fotoevaporativo por parte del gas y el viento estelar.

También se hizo una combinación de filtros para poder ver solo la
emisión de gas neutro (ver Apéndice \ref{AP: combos}. En este caso se
puede observar mejor como es la morfología del glóbulo, por lo que
aquí podemos conocer mejor sus propiedades.

Con la gran variedad de imágenes que tenemos podemos hacer muchas
comparaciones y ver los glóbulos en diferentes filtros y resoluciones,
lo cual nos da muchas ventajas. Por ejemplo, las observaciones del
JWST tienen mayor resolución, mientras que en el caso del filtro f656n
del HST, debido a que es un filtro muy angosto, no se ve muy afectado
por la emisión de las estrellas, a diferencia de las observaciones con
el JWST.

\begin{figure}[htb]
    \centering
    \includegraphics[width=\textwidth]{imagenes_corregidas/Arreglo_04.pdf}
    \caption{La imagen del filtro f656n (esquina superior izquierda)
      es la emisión de H$\alpha$ por parte del HST. Las demás imágenes son
      utilizando los filtros del JWST. Para el caso del gas ionizado
      se usó la siguiente combinación de filtros
      f44w-0.43\,f335m-0.16(f150w-0.6\,f210m), mientras que para la
      emisión de gas neutro se utilizó la combinación
      1.4(f150w-0.6\;f210m)+f335m-0.95\;f210m. Los círculos verdes son
      de los glóbulos.}
    \label{fig:filters WR124}
\end{figure}

\section{Estimando la presión RAM del viento estelar}

Para este trabajo solo vamos a considerar la presión RAM del viento
estelar por parte de la estrella WR 124 como la única presión externa
de confinamiento para las cáscaras chocadas de interacción de los
glóbulos, la cual la vamos a tomar como
\begin{equation}
 P_\mathrm{RAM}= \frac{\dot{M}v_\infty}{4\pi R^2}    
\end{equation}
donde $\dot{M}$ es la tasa de pérdida de masa de la estrella,
$v_\infty$ la velocidad terminal del viento y $R$ la distancia del glóbulo
a la estrella. Los dos primeros datos están dados en la tabla
\ref{tab:parametros WR-124} y para $R$ vamos a considerar el rango de
separación de 0.1--\SI{0.9}{pc}, esto ya que las separaciones entre los
glóbulos y la estrella caen en este rango considerando que
\begin{equation}
    \Big[\frac{R}{\mathrm{UA}}\Big] = \Big[\frac{D}{\mathrm{pc}}\Big]\,\Big[\frac{\theta}{\mathrm{arcsec}}\Big],
\end{equation}
donde $\theta$ es la separación que medimos directamente de las
observaciones en arcsec y $D$ es la distancia de nosotros a la
estrella. En la Figura \ref{P_RAM} vemos como es esta presión RAM del
viento como función de la distancia.

\begin{figure}[htb]
    \centering    \includegraphics[width=\textwidth]{Nuevas imagenes finales/PRAMcgs_n.pdf}
    \caption{Presión RAM del viento estelar de la estrella WR 124 como función de la distancia en parsec}
    \label{P_RAM}
\end{figure}

\chapter{Ajuste del modelo a los perfiles de
  brillo}\chaptermark{Ajuste del modelo}\label{Chapter : Ajuste}

En este capítulo vamos a utilizar las distintas observaciones para
obtener el tamaño de los glóbulos, sus cáscaras chocadas y los anchos
de sus cáscaras chocadas. También podemos obtener el brillos en el
centro del glóbulo y en la cáscara chocada, con lo que podremos
conocer más adelante la densidad del gas ionizado en la cáscara
chocada.

La medición de los diferentes parámetros a los 168 glóbulos
encontrados lo haremos de la siguiente manera. Una vez que sabemos
donde se encuentran los glóbulos vamos a buscar la posición de su
máximo de emisión en H$\alpha$, el cual definiremos como el centro de
nuestros glóbulos\footnote{En las observaciones de gas ionizado
  tenemos solo una estimación para el tamaño de los glóbulos, en la
  Sección \ref{Sec : radio neutro} se realiza una medición mas real
  para el tamaño de los glóbulos usando gas neutro.}, después
tomaremos como el eje de simetría considerado en el modelo (Figura
\ref{fig:zones}) a la línea que une el centro del glóbulo con la
estrella. Como en nuestro caso nos interesa más describir el glóbulo y
su cáscara chocada solo vamos a trabajar con una máscara para poder
identificar bien estas dos partes. Al utilizar esta máscara nuestras
mediciones no se verán afectadas por la compleja estructura de la
nebulosa u otros glóbulos que se encuentren cerca. Esta máscara la
vamos a considerar de la siguiente manera : Vamos a tomar solo los
píxeles que estén a una distancia menor o igual a \SI{0.2}{\arcsecond}
y también los que estén a una distancia máxima de \SI{1.5}{\arcsecond}
y a un ángulo máximo de $60^\circ$ con respecto al eje de simetría. Con
estos valores los glóbulos y sus cáscaras chocadas están dentro de la
máscara en todos los casos, además de que se obtuvieron buenos
resultados para las distintas mediciones con estos valores.

Después graficamos el brillo de cada píxel dentro de esta máscara como
función de la distancia al centro del glóbulo, como vemos en la Figura
\ref{ejemplo mascara}. A estos puntos le vamos a ajustar dos
gaussianas y una constante.


\begin{figure}[htb]
  \begin{subfigure}[b]{0.5\textwidth}
    \includegraphics[width=\textwidth, height=0.9\textwidth]{Nuevas imagenes finales/F_4_1_A.pdf}
    \caption{Ejemplo de máscara usada para los glóbulos. La máscara
      son los píxeles en escala de grises.}
    \label{fig:f1}
  \end{subfigure}
  \hfill
  \begin{subfigure}[b]{0.5\textwidth}
    \includegraphics[width=\textwidth, height=0.9\textwidth]{Nuevas imagenes finales/F_4_1_B.pdf}
    \caption{Brillo de la máscara como función de la distancia al
      centro del glóbulo, considerando el ángulo con respecto al eje
      de simetría.}
    \label{fig:f2}
  \end{subfigure}
  \caption{En la imagen de la izquierda vemos un ejemplo de la máscara
    que vamos a utilizar para hacer el ajuste a los perfiles de brillo
    (los pixeles en escala de grises), en este caso utilizando las
    observaciones del HST. El centro del glóbulo se localiza en el
    centro de la imagen, la línea roja conecta el centro del glóbulo
    con la estrella (localizada a la izquierda de esta imagen) y es el
    eje de simetría que vamos a considerar. Vamos a considerar solo
    los píxeles que están en un círculo de \SI{0.2}{\arcsecond}
    alrededor del centro del glóbulo y los que se encuentren a una
    distancia menor a \SI{1.5}{\arcsecond} y que tengan un ángulo
    menor de $60^\circ$ con respecto al simetría que estamos considerando.
    En la imagen de la derecha vemos como es el brillo de estos
    píxeles considerados en la máscara como función de la distancia al
    centro del glóbulo y considerando su ángulo con respecto al eje de
    simetría. En este caso, los puntos verdes son los píxeles que
    están en la parte inferior de la imagen izquierda, por otro lado
    los puntos azules se encuentran en la parte superior de la imagen
    izquierda dentro de la máscara considerada.}
  \label{ejemplo mascara}
\end{figure}

Este ajuste es por la siguiente razón: El ancho de la primer gaussiana
centrada en cero nos dirá el tamaño del glóbulo, esto suponiendo que
el pico de emisión que encontramos se da justo en el centro del
glóbulo. La segunda gaussiana ajustada nos indica la ubicación de la
cáscara chocada y su ancho. Esperamos ver un pico de emisión en esta
cáscara chocada debido a las recombinaciones que hay en esta parte, es
por eso que también le ajustamos una gaussiana a la cáscara chocada.
La distancia entre los picos de estas dos gaussianas no dirá el radio
de la cáscara chocada. Finalmente, la constante es más un promedio del
brillo del fondo en esta región. En este ajuste le dimos menos peso a
los píxeles más alejados del centro del glóbulo y también a los que
tenían un gran ángulo con respecto al eje de simetría de la siguiente
manera
\begin{equation}\label{eq: peso}
    w = \frac{\cos^2(\theta)}{(0.3+r)^2},
\end{equation}
donde $\theta$ es el ángulo con respecto al eje de simetría y $r$ la
distancia al centro del glóbulo. Para hacer este ajuste utilizamos la
paquetería \verb|fitting.LevMarLSQFitter|\footnote{La paquetería se
  encuentra en
  \url{https://docs.astropy.org/en/latest/api/astropy.modeling.fitting.LevMarLSQFitter.html}}
de \verb|astropy.modeling|. Un ejemplo más claro de estos ajustes lo
podemos ver en la Figura \ref{ejemplo ajuste}
\begin{figure}[htbp]
    \centering
    \includegraphics[width=\textwidth]{imagenes_corregidas/Ejemplo_ajuste_final.pdf}
    \caption{Ejemplo del ajuste de las dos guassianas y una constante
      a los perfiles de brillo. Este ejemplo es el mismo de la Figura
      \ref{ejemplo mascara}, pero para la visualización del ajuste, la
      imagen del glóbulo fue rotada $180^\circ$. En este ejemplo vemos
      como obtuvimos los diferentes parámetros a través del ajuste.
      Los brillos de la parte interna como de la cáscara son el máximo
      de la primer y segunda gaussiana, respectivamente. La $\sigma$ de la
      primer gaussiana centrada en cero nos dice el tamaño de la parte
      interna. Para el ancho de la cáscara, lo consideramos a partir
      de la $\sigma$ de la segunda guassiana. Para el radio de la cáscara,
      lo tomamos como la distancia entre los picos de ambas
      gaussianas. Este es un ajuste a las observaciones del HST y
      podemos ver como se ve el ajuste directamente en la imagen.}
    \label{ejemplo ajuste}
\end{figure}

Este ajuste se hizo a las observaciones con el HST y para el caso del
JWST se hizo para el filtro f090w y a la combinación de filtros en la
que solo observamos el gas ionizado. De esta manera obtuvimos varias
mediciones independientes a los ajustes con distintas resoluciones.
Así tenemos una mejor certeza de que realmente estamos detectando una
cáscara y del tamaño de los diferentes parámetros ajustados.

\section{Medición del radio en la parte neutra} \label{Sec : radio neutro}

El ajuste anterior se realizó utilizando las imágenes en las que vemos
la emisión de gas ionizado por lo que la medición de los radios para
la parte neutra no estaría del todo bien. Así que para la medición de
la parte neutra se utilizó la combinación de los filtros f150w, f210m
y f335m del JWST para ver la emisión neutra (ver Apéndice \ref{AP:
  combos}). Con esta combinación evitamos que nuestras mediciones se
vean afectadas por la emisión de estrellas de campo o de las cáscaras
chocadas.

En este caso ajustamos solo una gaussiana y una constante al perfil de
brillo ya que aquí no podríamos ver la cáscara chocada. El ajuste se
realizó de la siguiente manera: Dado que el valor medio de los radios
de los glóbulos en el anterior ajuste es de \SI{0.14}{\arcsecond} con
una variación de $\pm\SI{0.04}{\arcsecond}$ decidimos poner una máscara
de \SI{0.2}{\arcsecond} alrededor del pico de emisión y unos conos con
un pequeño ángulo de apertura, estos conos son perpendiculares al eje
de simetría considerado en el modelo y tienen una longitud de
\SI{1.5}{\arcsecond}. Con esta máscara considerada esperamos tener
toda la emisión neutra en el círculo pequeño que consideramos, los
conos nos servirán para calcular mejor la constante ajustada ya que no
se verá afectada por las colas delos glóbulos. En la Figura
\ref{Medicion de r_0} vemos como el glóbulo cae dentro de la máscara
sin verse afectado por emisión de gas ionizado o estrellas de campo.

Con estos ajustes ya conocemos el radio de los glóbulos, $r_0$, el
radio de la cáscara chocada, $r_\mathrm{shell}$, y el ancho de la
cáscara chocada, $H_\mathrm{s}$. Debido a que tanto $r_0$ como
$H_\mathrm{s}$ se midieron usando la sigma de la gaussiana ajustada,
la cual es el ancho RMS del perfil, y para comparar con los anchos
instrumentales tenemos que nuestro ancho observacional está dado por
\begin{equation}
    W_\mathrm{obs}= 2\sqrt{2\ln{2}} \, \sigma_\mathrm{obs}
\end{equation}
donde $\sigma_\mathrm{obs}$ es nuestra medición de las observaciones, y
considerando la Función de dispersión de punto (PSF por sus siglas en
inglés) de cada telescopio, entonces tenemos que nuestro ancho real
estará dado como
\begin{equation}
    W_\mathrm{real} = \frac{\sqrt{W_\mathrm{obs}^2-\Delta^2}}{2}
\end{equation}
donde $\Delta$ es el PSF del telescopio, para el caso del HST
$\Delta=\SI{0.067}{\arcsecond}$ y para el caso del JWST
$\Delta=\SI{0.145}{\arcsecond}$, esto para el combo de filtros para
ver solo el gas ionizado. De esta manera tenemos una medición más
realista de estas dos cantidades.

\begin{figure}[htb]
    \centering
    \includegraphics[width=0.8\textwidth]{Nuevas imagenes finales/r_0_.pdf}
    \caption{Ejemplo de la máscara usada para la medición del radio en
      la parte neutra del glóbulo. La máscara son los píxeles en
      escala de grises. Para ver solo la emisión del gas neutro (PAHs)
      se usó la combinación de los filtros f150w, f210m y f335m. Estas
      son observaciones del JWST.}
    \label{Medicion de r_0}
\end{figure}

\section[Errores observacionales]{Estimación de incertidumbres en los parámetros observacionales}

Para el caso de los errores vamos a considerar que estos están dados
por in-homogeneidades en el brillo de la nebulosa, que no están
relacionadas con el glóbulo ni con su cáscara chocada, o por efectos
sistemáticos debido a la inadecuación del modelo y no por el ruido de
fotones. Así que vamos a estimar estos errores comparando las
distintas mediciones hechas, suponiendo que son independientes.

\subsection{\boldmath Incertidumbres en $r_\mathrm{shell}$ y $H_\mathrm{s}$}
\label{sec:error-shell}
En el caso del radio y ancho de la cáscara, $r_\mathrm{shell}$ y
$H_\mathrm{s}$, respectivamente, podemos notar que estos siguen una
buena tendencia (ver Figura \ref{fgi: Radios de la cascara}) si
comparamos las mediciones realizadas con el HST y el JWST. Para estas
mediciones vamos a considerar la incertidumbre RMS
\begin{equation}
    \sigma=\frac{\sqrt{\mathrm{Var}(x_\mathrm{J}-x_\mathrm{H})}}{\sqrt{2}}
\end{equation}
donde $x_\mathrm{J}$ son las mediciones hechas con el JWST, utilizando
el combo de gas ionizado, y $x_\mathrm{H}$ son las mediciones hechas
con el HST, en H$\alpha$ (ver Apéndice \ref{AP: errores r_s H_s}).

\begin{figure}[htb]
    \centering
    \includegraphics[width=\textwidth]{imagenes_corregidas/rshell.pdf}
    \caption{Comparación de los radios de las cáscaras (puntos de
      colores) obtenidos a partir de los ajustes a ambos telescopios.
      Cada punto de color representa un glóbulo al cual se le detectó
      una cáscara en ambos telescopios. Las mediciones con el HST es
      en el filtro f656n y en el JWST son mediciones hechas en el
      combo para observar solo el gas ionizado. La línea roja
      representa la relación de identidad 1:1, por lo que se puede ver
      que ambas mediciones son muy similares}
    \label{fgi: Radios de la cascara}
\end{figure}

\subsection{\boldmath Incertidumbre en $r_0$}

Para el caso de la parte interna tenemos una medición para el radio en
la parte neutra con un combinación de filtros del JWST y una
aproximación con el HST, ya que en este último vemos solo el gas
ionizado y el glóbulo contiene principalmente gas neutro. Aunque no
hay una buena correlación entre estas dos mediciones, como se puede
ver en la Figura \ref{fig:Rcore dis}, su valor promedio y desviación
estándar se parecen mucho. Por lo que vamos a considerar un radio
constante que será igual para todos los glóbulos. Este valor será
\SI{0.135}{\arcsecond} el cual es el promedio de los valores medios de
las dos mediciones y como error vamos a considerar la desviación
estándar de ambas mediciones ya que son iguales,
$\pm\SI{.03}{\arcsecond}$ (ver Figura \ref{fig:Rcore dis}).

\begin{figure}[htb]
    \centering
    \includegraphics[width=0.8\textwidth]{imagenes_corregidas/r_0.pdf}
    \caption{En esta imagen vemos la comparación de los radios medidos
      para la parte neutra con ambos instrumentos. El histograma
      superior es para las mediciones del radio del glóbulo utilizando
      el JWST, estas mediciones tienen un valor promedio de
      \SI{0.15}{\arcsecond} con una desviación estándar de
      \SI{0.03}{\arcsecond}. El histograma del lado derecho es para
      las mediciones del radio del glóbulo utilizando el HST, teniendo
      un valor promedio de \SI{0.12}{\arcsecond} con una desviación
      estándar de \SI{0.03}{\arcsecond}. }
    \label{fig:Rcore dis}
\end{figure}

\subsection{\boldmath Incertidumbres en $B_\mathrm{s}$ y $B_0$}

Para los errores de los brillos superficiales en la cáscara chocada y
en el glóbulo, $B_\mathrm{s}$ y $B_0$, respectivamente, vamos a
considerar como error la siguiente desviación estándar
\begin{equation}
  \epsilon_{B}=\sqrt{\mathrm{Var}\Big(\overline{(y-\overline{y})^2},w*w_2\Big)} 
\end{equation} 
donde $y$ es el perfil del brillo del filtro f656n del HST menos el
ajuste realizado a los perfiles de brillo, $w$ es el peso considerado
en el modelo (Ecuación (\ref{eq: peso})) y $w_2$ es la gaussiana
ajustada a la parte correspondiente, es decir, para la parte interna
$w_2$ es la primer gaussiana ajustada, la cual tiene su pico en
\SI{0}{\arcsecond}, y para la cáscara $w_2$ es la segunda gaussiana
ajustada (ver Figura \ref{ejemplo ajuste}).

Con la estimación de todas estas incertidumbres, usamos la paquetería
\verb|uncertainties|\footnote{La documentación de la paquetería se
  encuentra en \url{https://pythonhosted.org/uncertainties/}} para
tener nuestras barras de error. Esta paquetería calcula las
incertidumbres usando la teoría de propagación de errores lineales
tomando en cuenta si los datos son totalmente independientes o están
correlacionados.

\section{\boldmath Estimación de la densidad del gas ionizado a partir
  del brillo superficial de H$\alpha$}\label{Sec : estimacion de densidad}

Para estimar la densidad del gas ionizado, usamos primero la
definición de Medida de Emisión (EM por sus siglas en inglés)
\begin{equation}
\mathrm{EM}=\int_z n_\mathrm{i} n_\mathrm{e}dz    
\end{equation}
donde $n_\mathrm{i}$ es la densidad de iones, $n_\mathrm{e}$ la
densidad electrónica y en este caso estaremos integrando sobre nuestra
línea de visión $(dz)$. Si consideramos un gas totalmente ionizado,
donde los electrones provengan solo del H, entonces tenemos que
$n_\mathrm{e}=n_\mathrm{i}=n$\footnote{En principio también tenemos
  contribución de He, pero lo vamos a considerar como neutro debido
  que la estrella tiene pocos fotones energéticos como para ionizar el
  He, si lo comparamos con la tasa de fotones que ionizan el H
  \citep{Palmira:2020}.}. Entonces, tenemos que $\mathrm{EM}=n^2l$,
donde $n$ es la densidad promedio RMS y $l$ es la profundidad del gas
denso en nuestra línea de visión.
 
\begin{figure}[htb]
    \centering    \includegraphics[width=\textwidth]{artesanales/ImgFi01-4.pdf}
    \caption{Diagrama esquemático que muestra una línea de visión, con
      un máximo en la línea punteada $l$, y si consideramos una
      simetría esférica vamos a tener esta configuración donde $h$
      será el ancho de la cáscara chocada, $r$ el radio del centro del
      glóbulo hasta donde inicia el flujo fotoevaporativo chocado y
      $h<r$.}
    \label{fig:EM}
\end{figure}

Si suponemos una cáscara esférica de radio $r$, una anchura $h$ con
$h\ll r$ y una densidad constante, entonces tenemos que
\begin{equation}
  \mathrm{EM}=2\sqrt{2rh}n^2.
\end{equation}

Esto ya que de la Figura \ref{fig:EM} tenemos que por geometría
$r^2+\Big(\frac{l}{2}\Big)^2=(r+h)^2=r^2+2rh+h^2\approx r^2+2rh$, y por
tanto, $ l=2\sqrt{2rh}$. De esta manera, usando la EM tenemos
que
\begin{equation}
  n=\sqrt{\frac{\mathrm{EM}}{l}}=\sqrt{\frac{\mathrm{EM}}{2\sqrt{2rh}}}.
\end{equation} 

En este trabajo vamos a aplicarlo a las cáscaras detectadas donde
tomaremos a $r=r_\mathrm{s}$ y $h=H_\mathrm{s}$.

\subsection{Uso de la EM a partir de las observaciones} \label{Subsec : EM}

En nuestras observaciones del HST el brillo superficial está dado en
unidades de \unit{cuenta.s^{-1}}, por lo que vamos a usar el factor de
conversión 0.0137 para tenerlo en unidades de
\unit{erg.s^{-1}.cm^{-2}.sr^{-1}} (ver Apéndice \ref{AP : conversion
  EM}). Si dividimos entre la energía de H$\alpha$ tenemos que

\begin{equation}
    B\frac{0.0137}{(h\nu)_{\mathrm{H}\alpha}}=\int \frac{f_{\mathrm{H}\alpha}\alpha_\mathrm{B} n_\mathrm{e} n_\mathrm{p}}{4\pi} dz
\end{equation}
donde $B$ es el brillo superficial que tenemos directamente de
nuestras observaciones, $(h\nu)_{\mathrm{H}\alpha}$ la energía de
H$\alpha$, $f_{\mathrm{H}\alpha}$ es la fracción de todas las recombinaciones a
los niveles $n\geq 2$ que resultan en la emisión de H$\alpha$,
$\alpha_\mathrm{B}$ es el coeficiente de recombinación para el caso B,
$n_\mathrm{e}$ la densidad electrónica y $n_\mathrm{p}$ la densidad de
iones. La integral de la derecha es a lo largo de nuestra línea de
visión.

Si consideramos que tanto $f_\mathrm{H\alpha}$ como el coeficiente de
recombinación son constantes, entonces
\begin{equation}
    B\frac{0.0137}{(h\nu)_{\mathrm{H}\alpha}}=\frac{f_{\mathrm{H}\alpha}\alpha_\mathrm{B}}{4\pi}\int n_\mathrm{e} n_\mathrm{p} dz,
\end{equation}
donde la integral es la EM. Por lo que, si suponemos que
$f_\mathrm{H\alpha}$ es 0.5 y que
$\alpha_\mathrm{B}=\SI{2.3e-13}{cm^{3}.s^{-1}}$ entonces
$f_{\mathrm{H}\alpha}\alpha_\mathrm{B}=\SI{1.17e-13}{cm^3.s^{-1}}$, por lo que
podemos conocer la EM directamente de las observaciones como
\begin{equation}
    \mathrm{EM} = B\frac{0.0137}{(h\nu)_{\mathrm{H}\alpha}}\frac{4\pi}{\SI{1.17e-13}{cm^3.s^{-1}}},
\end{equation}
y como ahora todo está en unidades de cgs, la EM tendrá unidades de \unit{cm^{-5}}.


Es importante mencionar que para conocer la densidad usando la EM y
$l$, estamos considerando que $l$ es perpendicular a el eje de
simetría considerado en el modelo.

En la Figura \ref{fig:dens_angl} estamos considerando que la densidad
máxima $n_1$ está en el eje de simetría y decae con el ángulo como
$\cos^{1/2} i$ \citep{Tarango:2018}.

\begin{figure}[htb]
    \centering    \includegraphics[width=\textwidth]{Nuevas imagenes finales/densi_angle.pdf}
    \caption{En el eje de simetría tenemos una densidad máxima en la
      cáscara chocada $n_1$ (color más oscuro) y esta densidad va
      cayendo con el ángulo que tiene con respecto al eje de simetría
      (color más claro).}
    \label{fig:dens_angl}
\end{figure}


\section{Buenos ajustes}\label{Good results}

En 16 de los glóbulos se pudo tener un buen ajuste ya que las
mediciones obtenidas coincidían tanto con los glóbulos como con las
cáscaras chocadas que se podían identificar visualmente en las
imágenes. Por lo que con estos ajustes ahora conocemos $r_0$,
$r_\mathrm{shell}$ y $H_\mathrm{s}$. Además conocemos el brillo tanto
en la parte interna como en la cáscara chocada por lo que podemos
obtener más información. En la Sección \ref{Sec : estimacion de
  densidad} vimos como obtener la densidad en la cáscara chocada a
partir de su brillo, y con esto ahora podemos conocer la presión de la
cáscara, la cual está dada como
\begin{equation}
    P_\mathrm{shell}=\rho c_\mathrm{s}^2.
\end{equation}
Esta presión la podremos comparar con la presión RAM del viento
estelar. De igual manera podremos tener un estimado de la presión
interna del glóbulo y comparar directamente con el modelo.

Estos glóbulos a los cuales les encontramos un buen ajuste están en un
rango de separación a la estrella muy amplio, por lo que podemos
conocer mejor la nebulosa y los glóbulos en general también. En la
Figura \ref{Goog G} podemos apreciar algunos ejemplos de los buenos
ajustes donde se puede identificar el glóbulo y su cáscara chocada.

\begin{figure}[htb]
    \centering
    \includegraphics[width=\textwidth]{imagenes_corregidas/buenos_aj.pdf}
    \caption{Ejemplos de buenos ajustes. En los paneles izquierdos
      vemos los ajustes a los perfiles de brillo en el lado izquierdo
      y como se ven con los datos del HST en la derecha, mientras que
      los paneles del lado derecho son utilizando los datos del JWST,
      del lado derecho están los ajuste a los perfiles de brillo para
      el gas ionizado (Figura \ref{fig:filters WR124}), pero la
      visualización de estos ajustes los vemos en el filtro f090w ya
      que morfológicamente se parece mucho a las imágenes del HST.}
    \label{Goog G}
\end{figure}

\section{Ajustes recuperados}

Gracias a que se hizo el ajuste a los perfiles de brillo en dos
telescopios distintos y con distintas resoluciones se pudieron obtener
mejores resultados. Esto debido a que en algunos glóbulos a pesar de
poder observar sus cáscaras chocadas visualmente nuestro algoritmo no
las lograba detectar en alguno de las dos imágenes.

En la Figura \ref{Recuperados Globulos} vemos como en ambos casos se
puede ver una posible cáscara pero al momento de hacer el ajuste no
detecta debido a la dispersión que tenemos en ciertas direcciones o
por la falta de puntos, como es el caso de los datos del HST. Otro
problema del porqué no se detectaba bien la cáscara, era debido a que
esta solía tener un brillo muy bajo.

De esta manera se pudo tener una mayor muestra de la que ya se había
obtenido inicialmente. Por lo que tenemos una muestra final de 30
glóbulos en los que conocemos el tamaño de los glóbulos, sus cáscaras
chocadas, así como sus anchos y el brillo superficial en el centro del
glóbulo y en la cáscara chocada\footnote{Todos estos ajustes se pueden
  ver en el Apéndice \ref{App : ajustes}}. Con estas mediciones
podemos comparar la presión de las cáscaras chocadas con la presión
RAM del viento estelar, por lo que en el Capítulo \ref{Ch : balance de
  presiones} podremos comparar nuestro modelo con las observaciones.

\begin{figure}[htb]
    \centering
    \includegraphics[width=\textwidth]{imagenes_corregidas/recuperados_aj.pdf}
    \caption{Ejemplos de ajustes recuperados. En las imágenes,
      visualmente podemos observar una cáscara, pero al hacer el
      ajuste a los perfiles de brillo vemos que este no detecta la
      cáscara chocada en uno de los telescopios (panel superior
      izquierdo y panel inferior derecho) debido a la dispersión de
      los puntos.}
    \label{Recuperados Globulos}
\end{figure}

\section{Glóbulos descartados}\label{Bad globules}

A pesar de tener una buena cantidad de glóbulos para aplicar a este
modelo no se usaron todos por diferentes razones. Algunos de ellos
tenían un mal ajuste debido a la gran estructura de la nebulosa, había
estrellas de fondo o se veían afectados por las espigas de difracción
de estrellas brillantes. Debido a esto en algunos casos no se
alcanzaba a detectar bien la parte neutra o la cáscara chocada y en
algunos casos la detección de estas regiones estaban mal en cuanto a
sus tamaños. En la Figura \ref{Bad Globules} vemos algunos ejemplos de
los glóbulos a los que no se les pudo detectar bien su cáscara
chocada.

\begin{figure}[htb]
    \centering
    \includegraphics[width=\textwidth]{Nuevas imagenes finales/Malos_ajustes_final.pdf}
    \caption{Ejemplos de glóbulos descartados. Del lado izquierdo es
      la imagen con el HST y del lado derecho con el JWST en el filtro
      f090w. El primer glóbulo es descartado por que se ve afectado
      por la difracción del telescopio y por la gran estructura de la
      nebulosa, esto se puede deber a lo cerca que está de la estrella
      central. En los otros ejemplos vemos como la presencia de
      fuentes cercanas afecta nuestras estimaciones de posibles
      cáscaras chocadas en las observaciones.}
    \label{Bad Globules}
\end{figure}

En la Tabla \ref{tab:mean} tenemos los valores medios de las
diferentes mediciones realizadas, tomando como incertidumbre la
dispersión de los datos entre los diferentes glóbulos. Vemos que la
separación entre los glóbulos y la estrella tiene un valor medio de
$14.96\pm\SI{7.1}{\arcsecond}$. Por otro lado, podemos ver que tanto el
radio del glóbulo como el ancho de la cáscara es menor que el radio de
la cáscara. La densidad es típicamente del orden de \SI{e3}{cm^{-3}},
mientras que las presiones de las cáscaras son del orden de
\SI{e-9}{dyn.cm^{-2}}. Finalmente $n_\mathrm{i,0}$ es la densidad del
gas ionizado en el frente de ionización de los glóbulos, esto
considerando un equilibrio de ionización.

\begin{table}[htb]
    \centering
    \begin{tabular}{c c}
        \toprule
        \multicolumn{2}{c}{Resultados de los buenos ajustes} \\ \midrule
         Separación & 14.96$\pm$\SI{7.1}{\arcsecond}\\
         Radio del glóbulo $r_0$ & 0.135$\pm$\SI{0.03}{\arcsecond} \\
         Radio de la cáscara $r_\mathrm{shell}$& 0.62$\pm$\SI{0.13}{\arcsecond}\\
         Ancho de la cáscara $H_\mathrm{s}$ & 0.24$\pm$\SI{0.1}{\arcsecond}\\
         $n_\mathrm{shell}$ & 1.37$\pm$\SI{.05e3}{cm^{-3}}\\
         $n_\mathrm{i,0}$ & 5.01$\pm$\SI{.55e3}{cm^{-3}}\\
         $P_\mathrm{shell}$ & 1.89$\pm$\SI{.06e-9}{dyn.cm^{-2}}  \\
         $r_\mathrm{shell}/r_0$ & 4.4$\pm$1 \\\bottomrule
    \end{tabular}
    \caption{Valores medios de los resultados obtenidos a partir de
      los ajustes. En este caso tomamos las incertidumbres como la
      dispersión entre los diferentes glóbulos.}
    \label{tab:mean}
\end{table}

\chapter{Balance de presiones en las cáscaras}\label{Ch : balance de presiones}

En este capítulo vamos a utilizar las mediciones obtenidas a partir de
las diferentes observaciones a los glóbulos encontrados en la nebulosa
M1-67. Así, podemos investigar acerca del equilibrio de presiones que
existe entre el flujo fotoevaporativo por parte de los glóbulos y la
presión del viento estelar para finalmente compararlo directamente con
el modelo hidrodinámico estacionario que proponemos. Si las
observaciones concuerdan con nuestro modelo propuesto, entonces
podremos conocer mejor la distribución espacial de los glóbulos en la
nebulosa M1-67.

Se encontraron 168 nudos glóbulos en la nebulosa, los cuales están
distribuidos a una distancia de la estrella central de entre
3--\SI{35}{\arcsecond} (7--\SI{92e-2}{pc}). Estos glóbulos se pueden
observar ya sea en grupos o solos, como se puede ver en la Figura
\ref{fig:dis_nudos}.

A pesar de no tener un buen ajuste a todos los glóbulos, tenemos una
muestra muy buena ya que estos están distribuidos en un amplio rango
de distancia a la estrella y también una gran variedad en la medición
de algunos parámetros.

\section{Balance interno de presión}\label{Sec : comparacion-modelo}

Ahora vamos a comparar directamente con el modelo (ver Figura
\ref{fig:grafica_C2}). Para esto vamos a comparar como se ve la razón
de las presiones que hay en la cáscara chocada del glóbulo y en la
superficie del glóbulo contra la razón del radio de la cáscara y el
radio del glóbulo. Recordemos que en la Figura \ref{fig:grafica_C2}
vemos a $P/P_0$ como función de $r/r_0$.

Como estamos considerando un modelo estacionario, tenemos que la
presión de la cáscara chocada, en la cual estamos considerando solo
presión térmica, debe ser igual a la presión total que hay justo antes
de $r_\mathrm{shell}$. En el Capítulo \ref{Chapter : Modelo} hablamos
de como el modelo propuesto predice $f$ como función de $r/r_0$, que
nos dice como debe ser la presión del flujo fotoevaporativo justo
antes del choque interior normalizado con la presión que hay en la
base del glóbulo, esto en función del radio.

En el caso de la razón entre los radios es muy sencillo, pues hemos
obtenido estos parámetros de una manera muy directa para cada glóbulo.
Dado que también hemos podido medir los brillos y tamaños, tanto del
glóbulo como de la cáscara chocada, podemos calcular sus respectivas
densidades.

Por lo que para la razón entre las presiones que hay en la base del
glóbulo y en la cáscara chocada del glóbulo, primero, recordando de la
sección \ref{Subsec : EM} tenemos que
$\mathrm{B}\propto \mathrm{EM}=n^2l$ por lo que
\begin{equation}
\frac{\mathrm{B_\mathrm{s}}}{\mathrm{B_0}}=\frac{n_\mathrm{s}^2l_\mathrm{s}}{n_0^2l_0},
\end{equation}
donde $\mathrm{B_\mathrm{s}}, \; n_\mathrm{s}$ y $l_\mathrm{s}$ son el
brillo, la densidad y la línea, perpendicular al eje de simetría,
donde estamos integrando para calcular la EM en la cáscara (ver Figura
\ref{fig:EM}) y $\mathrm{B_0}, \; n_0$ y $l_0$ son el brillo, la
densidad y la línea donde estamos integrando para conocer la EM en la
parte interna. Por otra parte, para las observaciones tenemos que
$f_\mathrm{obs} =
\frac{P_\mathrm{shell}}{P_0}=\frac{\rho_\mathrm{shell}}{2\rho_0} \Rightarrow
\frac{\rho_\mathrm{shell}}{\rho_0}=2f_\mathrm{obs}$, por lo que
\begin{equation}
\frac{\mathrm{B_\mathrm{s}}}{\mathrm{B_0}}=4f_\mathrm{obs}^2\frac{l_\mathrm{s}}{l_0}\Rightarrow f_\mathrm{obs}= \frac{1}{2}\Big(\frac{\mathrm{B_\mathrm{s}}/\mathrm{B_0}}{l_\mathrm{s}/l_0}\Big)^{1/2}.
\end{equation}
De esta manera podemos comparar con el modelo directamente. En el
apéndice \ref{App:brillos} damos a detalle las correcciones para los
brillos.

Debido a que el tamaño del glóbulo es muy pequeño, vamos a considerar
que $r_0\approx l_0$\footnote{De la sección \ref{Subsec : EM} tenemos que
  $l_0=2\sqrt{2hr}=2\sqrt{2h/r_0}r_0=0.98 r_0\approx r_0$, esto considerando
  que $h/r_0\approx0.12$ (ver Apéndice \ref{App : tasa de
    fotoionizacion}).}. Así la comparación con el modelo saldrá
directamente de las observaciones.

\begin{figure}[htb]
    \centering
    \includegraphics[width=\textwidth]{imagenes_corregidas/Model.pdf}
    \caption{$f_\mathrm{model}$ es la línea teórica que obtuvimos a partir del modelo propuesto, mientras que $f_\mathrm{obs}$ es la razón entre la presión de la cáscara chocada  y la presión en la superficie del glóbulo a partir de las observaciones.}
    \label{Resultados_modelo}
\end{figure}

En la Figura \ref{Resultados_modelo} podemos observar como los
resultados a partir de las observaciones se parecen bastante a las
predicciones que se obtienen del modelo propuesto. En el caso de las
observaciones vemos que la mayoría de los datos caen en un rango de
$r_\mathrm{shell}/r_0=$3--6, el cual es la razón del radio de la
cáscara entre el radio del glóbulo, mientras que para la razón entre
las presiones caen en un rango de $f_\mathrm{obs}=$0.05--0.15. También
podemos notar, para los datos observados, una anti-correlación con
bastante dispersión.

Dado que el modelo no tiene parámetros libres y que los resultados
obtenidos a partir de las observaciones tienen una gran concordancia
con las predicciones del modelo, podemos decir que este modelo
funciona para intentar explicar la interacción que hay entre los
flujos supersónicos. En la Figura \ref{Resultados_modelo} vemos como
las mediciones de nuestras observaciones, $f_\mathrm{obs}$ se ajustan
a las que predice nuestro modelo $f_\mathrm{model}=\frac{P}{P_0}$
(Ecuación \ref{eq : 1}) si consideramos las barras de error.

\section{Balance externo de presión}\label{Sec:proyeccion}

En la Figura \ref{graf_presion}vemos como la presión de las cáscaras
chocada de los glóbulos como función de la separación que hay entre la
estrella y cada glóbulo. Podemos apreciar como la presión de las
cáscaras de los glóbulos, puntos morados, cae por debajo de la presión
RAM del viento estelar, la línea roja. En nuestro modelo consideramos
que la presión de la cáscara debe estar en equilibrio con la presión
RAM del viento estelar, por lo que podemos pensar que la separación
que estamos viendo, en realidad, no es más que una separación
proyectada.

\begin{figure}[htb]
    \centering
    \includegraphics[width=\textwidth]{imagenes_corregidas/S_R.pdf}
    \caption{En esta gráfica vemos como las presiones de la cáscara de
      los glóbulos (círculos morados) estarían por debajo de la
      presión RAM (línea roja) si la separación proyectada fuera igual
      a la separación real.}
    \label{graf_presion}
\end{figure}

Si suponemos que el glóbulo está proyectado por un ángulo $i$ como
vemos en la Figura \ref{Ang proyeccion}, entonces tenemos que
\begin{equation}
R\cos i=R_\mathrm{p}
\end{equation}\label{eq:sep_real} 

\noindent donde $R$ es la separación real del glóbulo a la estrella,
$i$ el ángulo de inclinación y $R_\mathrm{p}$ la distancia proyecta,
que es la que observamos. Debemos tomar en cuenta que la presión de la
cáscara es igual a la presión RAM del viento estelar solamente en el
eje de simetría, y como vimos en la Sección \ref{Subsec : EM}, fuera
del eje de simetría, la presión de la cáscara disminuye por un factor
de $\cos^{1/2}i$ \citep{Tarango:2018}. Por lo que considerando este
ángulo de inclinación para la presión de la cáscara tenemos que
\begin{equation}
    P_\mathrm{shell}(i)=\rho\cos^{1/2}(i) c_\mathrm{s}^2 \Rightarrow P_\mathrm{shell}(i)\cos^{-1/2}(i)=\rho c_\mathrm{s}^2 
\end{equation}
por lo que  
\begin{equation}
P_\mathrm{shell}(i)\cos^{-1/2}(i)=\frac{\dot{M}v_\infty}{4\pi R^2} \Rightarrow
P_\mathrm{shell}(i)=\frac{\dot{M}v_\infty}{4\pi R_\mathrm{p}^2}\cos^{5/2}i.
\end{equation}\label{eq:cos 5_2}

\begin{figure}[htb]
    \centering
    \includegraphics[width=\textwidth]{artesanales/ImgFi01-6.pdf}
    \caption{Representación de como algunos glóbulos se ven afectados
      por un ángulo de inclinación con respecto a nuestra línea de
      visión. Para el glóbulo de abajo tenemos que la distancia real a
      la estrella es $R$, mientras que nosotros vemos $R_\mathrm{p}$
      que es la distancia proyectada a un ángulo $i$. Por otro lado el
      glóbulo de arriba no se ve afectado por alguna proyección.}
    \label{Ang proyeccion}
\end{figure}

Con esta ecuación, podemos conocer el coseno del ángulo, ya que
conocemos tanto la presión de la cáscara como su separación
proyectada. Usando este coseno del ángulo en la Ecuación
\ref{eq:sep_real} podemos conocer cuál es la separación real entre la
estrella y el glóbulo, por lo que podemos conocer cuál es la
distribución real de los glóbulos en cuanto a separación.

En la Figura \ref{graf_presion_ang} vemos como la presión de los
glóbulos se puede ajustar a la presión RAM si consideramos cierto
ángulo de inclinación $i$. De aquí podemos obtener el ángulo de
inclinación de cada
glóbulo.

\begin{figure}[htb]
    \centering
    \includegraphics[width=\textwidth]{imagenes_corregidas/S_52.pdf}
    \caption{Ejemplos de como podemos conocer el ángulo de inclinación
      de cada glóbulo si ajustamos una línea continua a la presión de
      la cáscara chocada de cada glóbulo (círculos morados) de tal
      manera que esta presión sea similar a la presión RAM del viento
      estelar (líneas continuas).}
    \label{graf_presion_ang}
\end{figure}

En la Figura \ref{fig:ncos_2} vemos en líneas continuas como sería el
ángulo de inclinación si consideramos que la densidad de la cáscara no
decae con el ángulo, es decir,
$P_\mathrm{shell}(i)=\frac{\dot{M}v_\infty}{4\pi R_\mathrm{p}^2}\cos^2i$. Y
vemos que a ángulos pequeños las líneas continuas se pegan mucho a las
líneas punteadas (cuando la densidad decae con el ángulo), por lo que
la diferencia de ángulos no es muy grande, sin embargo, a grandes
ángulos esto comienza a tomar relevancia. De igual manera, podemos
apreciar que a grandes distancias las líneas continuas y discontinuas
se pegan mucho, esto se puede deber a que en los glóbulos mas lejanos
a la estrella su separación real y proyectada sean muy similares. En
el siguiente capítulo vamos a discutirlo más a detalle.

\begin{figure}[htb]
    \centering
    \includegraphics[width=\textwidth]{imagenes_corregidas/S_2.pdf}
    \caption{La línea continua roja es la presión RAM del viento
      estelar como función de la distancia real, las demás líneas
      continuas son las presiones de las cáscaras considerando cierto
      ángulo de inclinación $i$ y que la densidad no decae con el
      ángulo. Las líneas discontinuas son las presiones de las
      cáscaras considerando cierto ángulo $i$ y que la densidad
      también decae con el ángulo como $\cos^{-1/2}(i)$.}
    \label{fig:ncos_2}
\end{figure}

En la Tabla \ref{tab:mean_i} vemos los valores medios considerando
este ángulo de inclinación que hemos encontrado, considerando que la
densidad decae con el ángulo. Como el valor medio del $\cos(i)$ es
0.65, podemos notar que en realidad sí estamos viendo la separación
proyectada de los glóbulos, de hecho de casi todos. Por lo que en
realidad están más lejanos de lo que apreciamos en las imágenes. En la
Figura \ref{fig:hist_sep_ryp} las barras naranjas son el histograma de
la separación proyectada entre los glóbulos y la estrella, la cual
medimos directamente de las observaciones, y podemos ver que muchos de
estos glóbulos parecían estar cerca de la estrella. Las barras de colo
azul es la separación real, encontrada a partir del equilibrio de
presiones, y vemos como todos están a más de \SI{10}{\arcsecond}
(\SI{0.26}{pc}) de distancia, así como también tenemos una
concentración de glóbulos a $\sim$\SI{17}{\arcsecond} (\SI{0.44}{pc}).

\begin{table}[htb]
    \centering
    \begin{tabular}{c c}
        \toprule
        \multicolumn{2}{c}{Resultados con ángulo de inclinación} \\ \midrule
         cos(i) & 0.65$\pm$0.01 \\
         $R$ & 22.08$\pm$\SI{0.34}{\arcsecond}\\
         $n_\mathrm{shell}(i)$ & 1.81$\pm$\SI{0.05}{cm^{-3}}\\
         $P_\mathrm{shell}(i)$ & 1.89$\pm$\SI{0.06e-9}{dyn.cm^{-2}} \\
         \bottomrule
    \end{tabular}
    \caption{Valores típicos de los resultados obtenidos considerando
      el ángulo de inclinación $i$.}
    \label{tab:mean_i}
\end{table}

\begin{figure}[htb]
    \centering
    \includegraphics[width=\textwidth]{ultimos/Hist_seprarcion(1).pdf}
    \caption{Histogramas de la separación real (color azul) y la
      separación proyectada (color naranja) de los glóbulos medida en
      arcsec.}
    \label{fig:hist_sep_ryp}
\end{figure}


\chapter{Discusión}\label{Chp:conclusiones}

Ahora vamos a discutir algunos problemas que que se presentaron al
momento de hacer los ajustes a los perfiles de brillos, así como
también algunas propiedades de los glóbulos, entre otras cosas. Debido
a que son varias cosas a discutir, vamos a dividir la discusión en
secciones.

Considerando que \cite{Zavala:2022} estima que la nebulosa
circunestelar se formó hace unos \SI{11.8e3}{a}, podríamos suponer que
las fases mencionadas en la Sección \ref{Sec:fluijos fotoevaporativos}
ya han pasado y esto es importante porque si estuviéramos en otra fase
como en la de implosión, los radios cambiarían además que las
presiones que consideramos podrían aún no estar en equilibrio. En el
caso de no estar aún en el equilibrio de presiones indicaría que esta
interacción entre el flujo fotoevaporativo y el viento estelar se
formó recientemente.

\section{Identificación de los glóbulos y sus cáscaras}\label{dis:casaras}

Si bien la presencia de estos glóbulos y su interacción con el viento
estelar se puede apreciar en las imágenes del JWST, no en todos los
casos es claro ya que hay algunos glóbulos que se encuentran en grupo
tal como se ve en la Figura \ref{globule_group}. En estos casos la
detección de la cáscara chocada es un poco difícil por las siguientes
razones. En la imagen de arriba de la Figura \ref{globule_group} vemos
un claro ejemplo de que cuando dos glóbulos estén muy cercanos se
pueden confundir con que sea un solo glóbulo, como es el caso de los
que están marcados con círculos azules. De esta manera nos podríamos
confundir en el tamaño de la parte neutra del glóbulo y sobrestimarla.
Detrás de estos glóbulos se encuentran otros dos glóbulos cercanos
(marcados con círculos negros), debido a la proyección en el cielo,
estos parecen estar en la estela de los glóbulos marcados con círculos
azules, por lo que en principio no podríamos detectar bien una cáscara
chocada. Por otro lado tenemos el glóbulo marcado con un círculo rojo.
Este glóbulo es más pequeño que los otros que están en el grupo y
aparentemente su cáscara chocada está muy cerca del glóbulo pero en
realidad esta cáscara parece ser del par de glóbulos marcados con
círculos azules. Algo similar pasa con el glóbulo marcado con el
círculo verde, el cual aparentemente no tiene una cáscara pero
pareciera estar en la cáscara chocada de algún otro glóbulo.

En la imagen inferior de la Figura \ref{globule_group} de igual manera
vemos un grupo de glóbulos, pero en esta ocasión están lo
suficientemente lejos como para no confundir sus respectivas cáscaras.
El problema aquí es que ahora las cáscaras están cerca la una de la
otra, por lo que la emisión de una afecta a otras. En este ejemplo en
particular, vemos que las cáscaras más grandes contaminan a las más
pequeñas en cuanto a su emisión.

\begin{figure}[htb]
    \centering
    \includegraphics[width=\textwidth]{imagenes_corregidas/n_gruops_globules.pdf}
    \caption{Estos son algunos ejemplos de los grupos de glóbulos que
      se encontraron en la nebulosa. En la imagen de arriba tenemos
      varios problemas, entre ellos identificar si es uno o más
      glóbulos para el caso de los que están muy cerca y saber de qué
      glóbulo viene cada cáscara chocada, en caso de detectarla. En la
      imagen de abajo vemos como la cercanía entre glóbulos hace que
      la emisión de una cáscara chocada se vea afectada por otra
      cercana.}
    \label{globule_group}
\end{figure}

\section{Masa de los glóbulos}\label{app:masa_glo}

Para la densidad del gas ionizado tenemos que \cite{Grosdidier:1998}
en su análisis a la nebulosa M1-67, encuentra unos puntos brillantes
de 0.2--\SI{0.3}{\arcsecond}, a los cuales les estima una densidad de
gas ionizado de 4800--\SI{12 000}{cm^{-3}}. Estos puntos brillantes
parecen ser en su mayoría nuestros glóbulos encontrados. Su estimación
de la densidad del gas ionizado de los glóbulos es congruente con
nuestras estimaciones, donde nosotros encontramos una densidad típica
de \SI{5e3}{cm^{-3}} (Tabla \ref{tab:mean}).

Para estimar la densidad en la parte neutra vamos a suponer un
equilibrio de presiones entre la parte neutra y la ionizada, por lo
que tendríamos que
\begin{equation}
    2 n_\mathrm{i,0}c_\mathrm{s,i}^2 = n_\mathrm{n}(c_\mathrm{s,n}^2+\frac{1}{2}v_\mathrm{A}^2),
\end{equation}
donde $n_\mathrm{i,0}, c_\mathrm{s,i}$ es la densidad y la velocidad,
respectivamente, en la parte ionizada, $n_\mathrm{n}, c_\mathrm{s,n}$
la densidad y la velocidad del sonido, respectivamente, en la parte
neutra y $v_\mathrm{A}$ la velocidad de Alfvén, la cual esta definida
como $v_\mathrm{A}=\frac{B}{\sqrt{4\pi \rho}}$, donde $B$ es el campo
magnético y $\rho$ la densidad. Como suponemos que la parte neutra está
dominada por el campo magnético, tenemos valores en el rango de
1--\SI{3}{km.s^{-1}} para la $v_\mathrm{A}$, esto considerando campos
magnéticos del orden de micro Gauss \citep{Bertoldi_1989}. Mientras
que la velocidad del sonido en la parte neutra es de
\SI{0.5}{km.s^{-1}} si consideramos una masa promedio de
1.3$m_\mathrm{p}$ y una temperatura de \SI{300}{K}. Así tenemos que la
razón entre la densidad neutra e ionizada
\begin{equation}
    \frac{n_\mathrm{n}}{n_\mathrm{i,0}}=\frac{2c_\mathrm{s,i}^2}{c_\mathrm{s,n}^2+\frac{1}{2}v_\mathrm{A}^2},
\end{equation}
está en el rango de 42--266. Por lo que para estimar la masa de los
glóbulos vamos a considerar que $n_\mathrm{n}=100n_\mathrm{i,0}$ ya
que se encuentra dentro de este rango, aunque claro, tendríamos una
gran incertidumbre por un factor de 2 aproximadamente. Tomando esto en
cuenta y que en el modelo estamos considerando que los glóbulos son
esféricos, entonces, la masa de cada glóbulo estaría dada por
\begin{equation}
    M_\mathrm{g} = \frac{4}{3}\pi r_0^3 100n_\mathrm{i,0} m_\mathrm{H}.
\end{equation}

Con estas densidades obtenidas para la parte neutra, tenemos un valor
promedio en la masa de \SI{4.5e-3}{\msun}. Si consideramos que todos
los glóbulos tienen esta masa promedio, entonces tendríamos que la
masa total de todos los glóbulos encontrados es
$168\times\SI{4.5e-3}{\msun}=$\SI{0.75}{\msun}\footnote{Como solo estamos
  estimando la masa de los glóbulos y no de sus cáscaras, vamos a
  tomar en cuenta los 168 glóbulos encontrados. }, la cual es menor
que la masa de gas ionizado que calcula \cite{Grosdidier:1998}, que es
de \SI{1.33}{\msun}.

%Masas y tasa de pérdida de masas 

\section{Distribución tridimensional de los glóbulos} \label{sec: distrtibucion}

\cite{Zavala:2022} utilizando espectroscopia de rendijas en ciertas
partes de la nebulosa midió las velocidades en estas regiones (ver
Figura \ref{fig:zavala_rendijas_nebula}). Una de las observaciones se
realizó en H$\alpha$ y además se realizó por donde están algunos glóbulos
que se encontraron en este trabajo.

En la Figura \ref{fig:zavala_nudos} vemos como los grupos de glóbulos
están en una elipse, lo que indica que están en una cáscara
esférica\footnote{En el Apéndice \ref{AP : PV} se da un poco de más
  detalle.}. En la rendija H es donde podemos ver el radio de estas
cáscaras en el eje vertical, que es la posición. En la misma rendija,
usando el eje horizontal que mide la velocidad, podemos ver que todas
estas cáscaras tienen una velocidad de \SI{40}{km.s^{-1}}. De esta
manera podemos obtener la separación real que hay entre los glóbulos y
la estrella. Por lo que conociendo su separación proyectada, que es la
que medimos directamente de las imágenes, y su separación real a
partir de estos espectro de rendija, podemos saber cuál debería ser su
ángulo de inclinación.

En la Figura \ref{fig:ang_Will} vemos como nuestras mediciones (puntos
rojos), obtenidos a partir de comparar la presión de la cáscara de los
glóbulos con la presión exterior del viento (Sección
\ref{Sec:proyeccion}) concuerdan con la mediciones realizada a partir
de espectros bidimensionales (puntos azules) (Henney, W., in prep.).
Estas dos mediciones son totalmente independientes y podemos ver una
clara anti-correlación entre el ángulo de inclinación y la separación
que observamos. En esta imagen no esperamos ver glóbulos con un ángulo
de inclinación de $\pm 90^\circ$ ya que estos estarían totalmente de
espaldas o de frente, en cualquier caso estos no los podríamos ver
porque se verían afectados por la emisión de la estrella WR ya que en
proyección estarían muy cerca de la estrella.

El grupo de glóbulos más cercanos a la estrellas parece estar a unos
\SI{15}{\arcsecond} (\SI{0.39}{pc}) según la Figura
\ref{fig:zavala_nudos}, los cuales coinciden con nuestras mediciones
que vemos en la Figura \ref{fig:hist_sep_ryp}. También en la Figura
\ref{fig:hist_sep_ryp} podemos ver que la separación real de muchos
glóbulos está cerca de \SI{18}{\arcsecond} (\SI{0.47}{pc}).

\begin{figure}[htb]
    \centering
    \includegraphics[width=\textwidth]{Nuevas imagenes finales/rendijas_zavala.pdf}
    \caption{Las lineas punteadas muestran donde \cite{Zavala:2022}
      realizaron espectroscopia de rendija larga en la nebulosa M1-67.
      Estas observaciones se realizaron en San Pedro Mártir. La línea
      continua en la esquina superior derecha representa
      \SI{20}{\arcsecond}.}
    \label{fig:zavala_rendijas_nebula}
\end{figure}

En la Figura \ref{fig:zavala_nudos} podemos apreciar que la mayoría de
los grupos de glóbulos están corridos al rojo y muy pocos están
corridos al azul, es decir, que la mayoría de los grupos de glóbulos
localizados en estas rendijas se están alejando de nosotros y solo
unos pocos se están acercando.

\begin{figure}[htb]
    \centering
    \includegraphics[width=\textwidth]{Nuevas imagenes finales/zavala-slit-spectra-select-annotated.pdf}
    \caption{Aquí vemos la espectroscopia de rendijas de
      \cite{Zavala:2022}, en la parte superior vemos la emisión de
      H$\alpha$ y en la parte inferior la emisión de NII. Los círculos
      amarillos representan donde se encuentran los grupos de glóbulos
      en cada rendija. Las elipses representan modelos cáscaras
      esféricas con radios de \SI{15}{\arcsecond} (elipses moradas),
      \SI{20}{\arcsecond} (elipses rosas), \SI{25}{\arcsecond}
      (elipses amarillas), \SI{30}{\arcsecond} (elipses verde claro) y
      \SI{35}{\arcsecond} (elipses verdes), todos con una velocidad de
      \SI{40}{km.s^{-1}}.}
    \label{fig:zavala_nudos}
\end{figure}

\begin{figure}[htb]
    \centering
    \includegraphics[width=\textwidth]{imagenes_corregidas/W.pdf}
    \caption{Los datos en color rojo son nuestras estimaciones de los
      ángulos de inclinación realizados en la sección
      \ref{Sec:proyeccion}, en la que suponemos que la presión de la
      cáscara está en equilibrio de presión con la presión externa del
      viento. Los datos en color azul son las estimaciones de los
      ángulos de inclinación usando los espectros bidimensionales de
      \cite{Zavala:2022}}
    \label{fig:ang_Will}
\end{figure}

\chapter{Conclusiones}

En esta tesis se ha propuesto un modelo hidrodinámico estacionario
para explicar la interacción que hay entre el flujo fotoevaporativo de
los glóbulos encontrados en la nebulosa M1-67 y el viento estelar por
parte de la estrella WR-124.

Estos glóbulos fueron posibles de detectar, así como su cáscara
chocada, gracias a las diferentes observaciones. De estas
observaciones fue posible encontrar diferentes parámetros físicos
haciendo un ajuste de dos gaussianas y una constante como vimos en el
capítulo \ref{Chapter : Ajuste}. Estos parámetros físicos fueron:

\begin{itemize}
    \item Separación proyectada entre el glóbulo y la estrella
    \item Radio de los glóbulos
    \item Radio de la cáscara
    \item Ancho de la cáscara chocada
    \item Brillo del glóbulo
    \item Brillo de la cáscara
\end{itemize}

Usando la EM calculamos la densidad en la cáscara chocada (Sección
\ref{Sec : estimacion de densidad}), y suponiendo que la cáscara
chocada está dominada por presión térmica, pudimos conocer la presión
de las cáscaras de los glóbulos.

En la Figura \ref{graf_presion} vemos que la presión de los glóbulos
es menor que la presión RAM del viento estelar, sin embargo, podemos
encontrar un ángulo de inclinación respecto al plano del cielo con el
cuál podemos tener un equilibrio de presiones entre la presión de la
cáscara y la presión RAM del viento estelar (Figura
\ref{graf_presion_ang}). Estas estimaciones de los ángulos de
inclinación en los que suponemos un equilibrio de presiones (Sección
\ref{Sec:proyeccion}), son consistentes con la mediciones hechas a
partir de espectros bidimensionales (Sección \ref{sec:
  distrtibucion}).

Además, usando las observaciones del HST, comparamos la razón de la
presión térmica de la cáscara entre la presión del flujo
fotoevaporativo en la base del glóbulo (Sección \ref{Sec :
  comparacion-modelo}). Afortunadamente, estas mediciones concuerdan
con nuestro modelo teórico como podemos ver en la Figura
\ref{Resultados_modelo}.

Por lo que podemos decir que este modelo sencillo es un buen modelo
para explicar la interacción del flujo fotoevaporativo de los glóbulos
y el viento estelar por parte de la estrella WR 124. Además, no
tenemos parámetros libres, por lo que la consistencia con otros
trabajos apoya nuestro modelo propuesto. Por lo que nuestras
suposiciones de que estos glóbulos ya han pasado por la fases
mencionadas en el Capítulo \ref{Capitulo 1:introduccion} y que estamos
en un equilibrio de ionización son ciertas.

Este modelo propuesto en un principio está puesto para un escenario
sencillo, un glóbulo que es radiado por una fuente, pero esto podría
ampliarse un poco más si tenemos varias fuentes que radian al glóbulo
y una de ellas es la que domina en cuanto al flujo de fotones
ionizantes.

\appendix

\chapter{Filtros de las observaciones}\label{App: Filtros}

En este apéndice vamos a dar más detalles acerca de los diferentes
filtros que se utilizaron en las observaciones.
\section{Filtro del HST}
Para el caso del HST solo se utilizaron las observaciones del filtro
f656n, en el cual podemos ver la emisión de H$\alpha$. Este filtro observa
desde \SI{6548.77}{A} hasta \SI{6674.27}{A} y está centrada en
\SI{6563.8}{A} como se puede ver en la Figura \ref{fig:filtro f656n}.

\begin{figure}
    \centering
    \includegraphics[width=\textwidth]{Appendices/f656n_filter.png}
    \caption{Transmisión del filtro f656n del HST}
    \label{fig:filtro f656n}
\end{figure}

\section{Filtros del JWST}\label{AP S:JWST}

Para el caso de las observaciones del JWST se utilizaron diferentes
filtros, tanto del NIRcam como del MIRI. El filtro f1130w es el único
filtro del MIRI y observa desde 10.953--\SI{11.667}{\mum} y está
centrado en \SI{11.3}{\mum}. En la Figura \ref{fig:filtos MIRI} vemos
la transmisión de este filtro.

\begin{figure}
    \centering
    \includegraphics[width=\textwidth]{Appendices/miri_img_pces_ETC4.0.png}
    \caption{Transmisión de los filtros de MIRI. El filtro f1130w está en azul.}
    \label{fig:filtos MIRI}
\end{figure}

Para los filtros del NIRcam tenemos a los filtros f090w, f150w, f444w,
f210m y f335m, en la tabla \ref{tab:filtros} podemos ver en que
longitudes de onda observa cada filtro. En la Figura \ref{fig:filtros
  JWST} vemos la transmisión de cada filtro.

En el filtro f090w podemos ver algunas líneas de emisión como
$[\mathrm{S \scriptstyle{III}}]$ en las longitudes de onda de .906 y
.935 $\mu$m, así como una línea de la serie de Paschen de
$\mathrm{H\scriptstyle{I}}$ en la longitud de onda de .954 $\mu$m. En el
filtro f150w se puede ver líneas de emisión de
$\mathrm{He\scriptstyle{I}}$ en las longitudes de onda de 1.34 y 1.5
$\mu$m, así como una línea de la serie de Brackett de
$\mathrm{H\scriptstyle{I}}$ en la longitud de onda de 1.64 $\mu$m. En el
filtro f210m podemos encontrar líneas de emisión como las de
$\mathrm{He\scriptstyle{I}}$ en la longitud de onda de 2.05 $\mu$m, una
línea de $\mathrm{H_2}$ en la longitud de onda de 2.12 $\mu$m y una
línea de la serie de Brackett de $\mathrm{H\scriptstyle{I}}$ en la
longitud de onda de 2.16 $\mu$m. En el filtro f333m podemos ver líneas
de emisión de PAHs en la longitud de onda de 3.3 $\mu$m, una línea de la
seria Pfund de $\mathrm{H\scriptstyle{I}}$ en la longitud de onda de
3.29 $\mu$m y una línea de $\mathrm{He\scriptstyle{I}}]$ en la longitud
de onda de 3.36 $\mu$m. En el filtro f444w podemos encontrar líneas de
emisión de $\mathrm{H\scriptstyle{I}}$ en las longitudes de onda de
4.05 y 4.65 $\mu$m, así como una línea de emisión de
$[\mathrm{Mg\scriptstyle{IV}}]$ en la longitud de onda de 4.48 $\mu$m.
Estas son algunas líneas de emisión que se pueden encontrar en los
diferentes filtros de acuerdo a la literatura.

\begin{table}[htb]
    \centering
    \begin{tabular}{c c c c}
        \toprule
        filtro & $\lambda_0$($\mu$m) & $\lambda_{min}$($\mu$m) & $\lambda_{max}$($\mu$m) \\ 
        \midrule
         f090w & .903 & .795 & 1.005\\
         f150w &1.501 &1.331 & 1.668\\
         f210m &2.096 &1.992 & 2.201\\
         f335m &3.362 &3.177 & 3.537\\
         f444w &4.401 &3.880 & 4.981\\
         \bottomrule
    \end{tabular}
    \caption{Rango en el que observa cada filtro utilizado para las
      observaciones utilizadas. $\lambda_0$ es la longitud de onda a la que
      está centrado cada filtro, $\lambda_\mathrm{min}$ es la longitud de
      onda mínima a la que observa y $\lambda _\mathrm{max}$ es la longitud
      de onda más grande a la que observa cada filtro.}
    \label{tab:filtros}
\end{table}

\begin{figure}
    \centering
    \includegraphics[width=\textwidth]{Appendices/multi_filter_plot_per_detector_May2024.png}
    \caption{Transmisión de los filtros de NIRcam. Los filtros f090w,
      f150w y f444w se encuentra en la segunda imagen, mientras que
      los filtros f210m y f335m se encuentran en la tercera imagen.}
    \label{fig:filtros JWST}
\end{figure}

\chapter{Estimación de fuerzas en el flujo fotoevaporativo ionizado}\chaptermark{Fuerzas en el flujo fotoevaporativo} \label{App:fuerzas}

En estas estimaciones de las diferentes fuerzas solo haremos
aproximaciones, por lo que para los cálculos vamos a usar los valores
típicos de los ajustes (tabla \ref{tab:mean} y \ref{tab:mean_i}).

Para comparar las distintas fuerzas es más conveniente comparar las
presiones o aceleraciones ya que estas son fuerza por unidad de área o
fuerza por unidad de masa, respectivamente.


Primero vamos a considerar la aceleración provocada por el gradiente
de presión, para esto tomamos la fuerza por unidad de masa la cual
está dada por
\begin{equation}
\rho a = \frac{dP}{dr}\Rightarrow a= \frac{1}{\rho}\frac{dP}{dr}=\frac{1}{\rho}\frac{\partial \rho}{\partial r}\frac{\partial P}{\partial\rho}=\frac{c_\mathrm{s}^2}{h}
\end{equation}
como estamos considerando un gas isotérmico, vemos que del lado
derecho tenemos los factores de la velocidad del sonido cuadrada y la
escala de altura $h$, la cual está definida por
\begin{equation}
h^{-1}=\Big|\frac{1}{\rho_0}\frac{\partial\rho}{\partial r}\Big|=\Big|\frac{d \ln \rho}{dr}\Big|
\end{equation}
usando las ecuaciones (\ref{eq : 2}) y (\ref{eq ; 3}) tenemos que 
\begin{equation}
\frac{\rho}{\rho_0}=e^{\frac{1-M^2}{2}}\Rightarrow\ln\frac{\rho}{\rho_0}=\frac{1-M^2}{2}
\end{equation}
\begin{equation}
\Rightarrow d\ln\frac{\rho}{\rho_0}=-M dM
\end{equation}
por otro lado usando que 
\begin{equation}
\frac{r}{r_0}=M^{-1/2}e^{\frac{M^2-1}{2}}
\end{equation}
tenemos que 
\begin{equation}
d\frac{r}{r_0}=e^{\frac{M^2-1}{4}}\Big(-\frac{M^{-3/2}}{2}+\frac{M}{2M^{1/2}}\Big)dM = \frac{e^{\frac{M^2-1}{4}}}{2}\Big(M^{1/2}(1-1/M^2) \Big)dM
\end{equation}
por lo que 
\begin{equation}
h^{-1}=\Big|\frac{2M}{e^{\frac{M^2-1}{4}}\Big(M^{1/2}-M^{-3/2}\Big)}\Big|
\end{equation}
que como podemos ver en la Figura \ref{fig:h} cuando $r/r_0\sim 1$
tenemos un valor del orden de 0.2\footnote{En la Figura \ref{fig:h}
  vemos que cuando $r=r_0$ $h\to 0$, por lo que tendríamos una
  aceleración infinita.}. Por lo que el gradiente de presión nos da
una aceleración de
\begin{equation}
a_\mathrm{p} \approx \frac{c_\mathrm{s}^2}{0.2 r_0} = \SI{3.8e-4}{cm.s^{-2}}
\end{equation}
a un radio típico $r_0\sim \SI{0.135}{\arcsecond}$. Como el modelo se
resolvió en el eje de simetría, tenemos que aquí los gradientes
transversales son cero, mientras que si consideramos el modelo a
cierto ángulo debemos considerar que el gradiente de densidad
transversal es más pequeño, por un factor de 10 aproximadamente.

\begin{figure}
    \centering
    \includegraphics[width=\textwidth]{imagenes_corregidas/h.pdf}
    \caption{Gráfica de $h$ con respecto al radio normalizado. Vemos que cerca de donde tenemos la emisión en $r/r_0$ el valor de $h$ es del orden de 0.1}
    \label{fig:h}
\end{figure}

\section{Fuerzas de gravedad} \label{F gravedad}

\cite{Hamann:2019} estima una masa de 20-$\SI{22}{\msun}$ para la
estrella WR 124, por lo que la fuerza de gravedad por parte de la
estrella nos da un aceleración de
\begin{equation}
a_*=\frac{GM_*}{R^2}\approx \SI{1.97e-9}{cm/s^2}
\end{equation}
con $R$ una distancia típica entre la estrella y el glóbulo de
\SI{14.96}{\arcsecond}. Si tomamos la distancia típica considerando el
ángulo de inclinación $i$ (Sección \ref{Sec:proyeccion}), vemos que
esta aceleración es todavía más pequeña.

Ahora vamos a considerar la aceleración por parte de la gravedad del
mismo glóbulo. Para esto, vamos a considerar la masa neutra del
glóbulo y la masa ionizada. En la sección \ref{app:masa_glo} hablamos
de como obtener la masa neutra de los glóbulos, para la masa ionizada
vamos a considerar la masa que se encuentra en la mitad de la cáscara
que hay entre $r_0$ y $r_\mathrm{shell}$, por lo que a masa ionizada
es
\begin{equation}
    M_\mathrm{i} = \rho_1 \frac{2\pi}{3}(r_\mathrm{shell}^3-r_0^3)
\end{equation}
donde $\rho_1$ es la densidad por unidad de masa en la parte ionizada. De
esta manera, tenemos que la masa del glóbulo a es
$M_\mathrm{g} = M_\mathrm{i} + M_\mathrm{n} \approx \SI{7.87e-4}{\msun}$
(considerando una $v_\mathrm{A}=\SI{1}{km.s^{-1}}$ para la parte
neutra) y la aceleración por parte de la fuerza de gravedad del mismo
glóbulo es de
\begin{equation}
a_\mathrm{g}=\frac{G M_\mathrm{g}}{r_\mathrm{shell}^2}\approx \SI{4.55e-11}{cm/s^2}.
\end{equation}
Si en esta estimación tomamos una $v_\mathrm{A}$ mayor, tendríamos una
aceleración un poco menor. De igual, si consideramos la densidad
ionizada de la tabla \ref{tab:mean_i}, esta aceleración no cambia
mucho. Para el caso de lo glóbulos que están en grupo, de igual manera
podemos despreciar la aceleración por parte de los demás glóbulos, ya
que por muy cercanos que estén, podemos considerar una distancia
mínima de $r_\mathrm{shell}$.

De esta manera tenemos que
\begin{equation}
a_\mathrm{g}<a_*\ll a_\mathrm{p}
\end{equation}
donde $a_\mathrm{g}$ es la aceleración provocada por la gravedad del
glóbulo, $a_*$ la aceleración provocada por la estrella WR 124 y
$a_\mathrm{p}$ la aceleración provocada por la diferencia de presiones
en la superficie del glóbulo.

\section{Presión de radiación}

Vamos a considerar la presión de radiación ya que podemos suponer que
todo el momento de los fotones ionizantes se va al flujo
fotoevaporativo, por lo que si consideramos que todo la radiación
ionizante es absorbida en el flujo fotoevaporativo entonces tenemos
que para la radiación ionizante $Q = \SI{1.25e49}{s^{-1}}$, según la
tabla \ref{tab:parametros WR-124}, tendríamos una intensidad de
\begin{equation}
\frac{Q h\nu}{4\pi R^2} = \frac{\SI{2.74e38}{erg.s^{-1}}}{4\pi R^2} \approx \SI{6.8}{erg.s^{-1}.cm^{-2}}
\end{equation}
para la frecuencia que corresponde a $\SI{1}{Ry}$, el cual es un
límite inferior para los fotones que son capaces de ionizar el gas
neutro, esto a una distancia típica de los glóbulos. Por lo que
tendríamos una presión de radiación de
$P_\mathrm{r}\approx \SI{2.26e-10}{dyn.cm^{-2}} < P_\mathrm{shell}$.

A pesar de que consideramos que el glóbulo esta soportado
principalmente por un campo magnético, no vamos a considerar presión
magnética en el gas ionizado ya que este sería despreciable con la
presión térmica \citep{Will:2009}.

\chapter{Predicciones del modelo fotoevaporativo para la densidad ionizada en el frente de ionización }\chaptermark{Densidad ionizada en el frente de ionización}\label{App : tasa de fotoionizacion}

En el modelo suponemos un estado estacionario y además, también
suponemos que no hay absorción por polvo, por lo que el flujo
incidente de fotones, $F_0$, debe ser igual a la suma de dos términos,
las recombinaciones por unidad de área y las nuevas ionizaciones
\begin{equation}\label{eq:flujo_io}
F_0 = n_\mathrm{i,0} u_\mathrm{i,0} +\int n^2\alpha_\mathrm{B}dr = n_\mathrm{i,0}u_\mathrm{i,0}+n_\mathrm{i,0}^2h_1\alpha_\mathrm{B}
\end{equation}
donde $F_0$ es la tasa de fotones ionizantes por unida de área,
$n_\mathrm{i,0}$ la densidad del gas ionizado, $u_0$ la velocidad del
gas ionizado y $h_1$ es la anchura efectiva de la capa ionizada que se
define como
\begin{equation}
n_0^2h_1=\int n^2dr,
\end{equation}
el cual se puede estimar usando las ecuaciones (\ref{eq : 1}),
(\ref{eq : 2}) y (\ref{eq ; 3}). Por lo que tendríamos que
\begin{equation}
h_1=\int_0^\infty \Big(\frac{n(r)}{n_0}\Big)^2dr=r_0\int_1^\infty\frac{exp(\frac{3}{4}(1-M^2))}{2}(M^{1/2}-M^{3/})dM\approx0.12r_0.\end{equation}

Tomando los valores de $F_0$ y $r_0$ de nuestros glóbulos, tenemos
que, el primer término del lado derecho de la Ecuación
(\ref{eq:flujo_io}) es muy pequeño en comparación con el segundo
término. Por lo que podemos despreciar este término.

\chapter{Combos de filtros}\label{AP: combos}

En este apéndice vamos a explicar como se obtuvieron los combos de gas
ionizado y gas neutro utilizando las diferentes observaciones del
JWST. Usamos estas combinaciones para hacer ajustes a los perfiles de
brillo y así obtener mediciones del radio del glóbulo, de la cáscara
chocada y su ancho.

Como usamos diferentes filtros, los cuales tienen diferente
resolución, hicimos una convolución para que todos tuvieran la misma
resolución y así poder combinar imágenes. En este caso convolucionamos
las imágenes para tener la misma resolución que el filtro f444w, el
cual tiene la menor resolución de los filtros usados. Una vez que
todas las imágenes tienen la misma resolución, normalizamos la emisión
de las estrellas, el gas ionizado y PAHs a la emisión en el filtro
f210m de sus respectivos mecanismos de emisión como se puede ver en la
Tabla \ref{tab:emision en filtros}. Como estos cocientes de imágenes
se realizaron visualmente, tomamos el error como el rango completo en
donde se encontraban estos cocientes.
\begin{table}[htb]
    \centering
    \begin{tabular}{l L L L}
        \toprule
        Filtro & \text{Estrellas} & \text{Gas ionizado}   & \text{PAHs} \\
        \midrule
         f090w & 0.4 \pm 0.15 & 0.57\pm 0.05 & 0.4\pm 0.3 \\
         f150w & 1.1\pm0.1 & 0.6\pm0.05 & 0.6\pm0.3 \\
         f210m & 1.0 & 1.0 & 1.0 \\
         f335m & 0.25\pm0.05 & 0.95\pm0.15 & 7.0\pm0.4\\
         f444w & 0.19\pm0.07 & 1.9\pm0.5 & 3.0\pm1.0 \\
         %$F_{H_\alpha}$ & \SI{3e-14}{erg.cm^{-2}.s^{-1}} & \cite{Grosdidier:1998}\\
         \bottomrule
    \end{tabular}
    \caption{Emisión de los diferentes componentes normalizados al filtro f210m.}
    \label{tab:emision en filtros}
\end{table}

De la Tabla \ref{tab:emision en filtros} podemos observar que la
emisión de gas ionizado y PAHs en el filtro f150w es igual, por lo que
con una simple combinación podemos tener solo la emisión de las
estrellas como se puede ver en la Tabla \ref{tab:emision en
  filtros_primera combinacion}. Combinando los filtro f335m y f210m se
pudo quitar la emisión de gas ionizado, de manera similar, combinando
los filtros f444w y f335m quitamos la emisión de los PAHs (ver Tabla
\ref{tab:emision en filtros_primera combinacion}).
\begin{table}[htb]
    \centering
    \begin{tabular}{l L L L L}
        \toprule
            &\text{Combinación} & \text{Estrellas} & \text{Gas ionizado} & \text{PAHs} \\
        \midrule
         A & \text{f150w} - 0.6 \, \text{f210m} & 0.5 \pm 0.1 & 0.00 \pm 0.05 & 0.00 \pm 0.30\\
         B &  \text{f335m} - 0.95 \, \text{f210m} & -0.70 \pm 0.05 & 0.00 \pm 0.15 & 6.05 \pm 0.40\\
         C &  \text{f444w} - 0.43 \, \text{f335m} & 0.08 \pm 0.07 & 1.49 \pm 0.50 & -0.01 \pm 1.01 \\
         %$F_{H_\alpha}$ & \SI{3e-14}{erg.cm^{-2}.s^{-1}} & \cite{Grosdidier:1998}\\
         \bottomrule
    \end{tabular}
    \caption{Primera combinación de filtros.}
    \label{tab:emision en filtros_primera combinacion}
\end{table}

Con la primer combinación de la Tabla \ref{tab:emision en
  filtros_primera combinacion} podemos quitar la emisión de las
estrellas en las otras combinaciones y así poder ver solo la emisión
de gas ionizado o de PAHs. En la primer combinación de la Tabla
\ref{tab:emision en filtros_segunda combinacion} vemos solo la emisión
de gas neutro y en la segunda combinación vemos solo la emisión de gas
ionizado.
\begin{table}[htb]
    \centering
    \begin{tabular}{L L L L}
        \toprule
        \text{Combinación} & \text{Estrellas} & \text{Gas ionizado} & \text{PAHs} \\
        \midrule
         1.4 \text{A}+\text{B} & 0.00 \pm 0.15 & 0.00 \pm 0.17 & 6.05 \pm 0.58\\
         \text{C}-0.16 \text{A} & 0.00 \pm 0.07 & 1.49 \pm 0.50 & -0.01 \pm 1.01 \\
         \bottomrule
    \end{tabular}
    \caption{Combinación de filtros para ver solo la emisión de gas ionizado y PAHs.}
    \label{tab:emision en filtros_segunda combinacion}
\end{table}

\chapter{Errores en $r_\mathrm{shell}$ y $H_\mathrm{s}$}\label{AP: errores r_s H_s}

Si consideramos que cada medición realizada en el gas ionizado en el
JWST , $x_\mathrm{J}^i$, y que cada medición en en H$\alpha$ en el HST,
$x_\mathrm{H}^i$, lo podemos ver como la medición real más un error,
es decir,
\begin{equation}
    x_\mathrm{J}^i=x+\epsilon_\mathrm{J}^i,  \qquad
    \qquad
    x_\mathrm{H}^i=x+\epsilon_\mathrm{H}^i
\end{equation}
donde $x$ es la medición real y que $\epsilon_\mathrm{J}^i$,
$\epsilon_\mathrm{H}^i$ son respectivamente los errores en las mediciones de
los dos telescopios. Entonces tenemos que
\begin{equation}
    x_\mathrm{J}^i-x_\mathrm{H}^i=\epsilon_\mathrm{J}^i-\epsilon_\mathrm{H}^i,
\end{equation}
donde estamos considerando que los errores en los dos telescopios son
independientes. Considerando que la varianza sobre toda la muestra
está dada como
\begin{equation}
    \mathrm{Var}(x_\mathrm{J}-x_\mathrm{H})=\frac{\sum_{i=1}^N ((x_\mathrm{J}^i-x_\mathrm{H}^i)-\mu)^2}{N}
\end{equation}
donde $N$ es el tamaño de la muestra y
$\mu = N^{-1} \sum_{i=1}^N (x_\mathrm{J}^i-x_\mathrm{H}^i)$ es la media. Si
las distribuciones son simétricas, esperamos que $\mu \approx 0$ para
$N$ grande, esto fue verificado para nuestras muestras. Entonces,
suponiendo que no hay correlación entre los errores
$\epsilon_\mathrm{J}^i$ y $\epsilon_\mathrm{H}^i$, tenemos que
\begin{equation}
    \mathrm{Var}(x_\mathrm{J}-x_\mathrm{H})=\mathrm{Var}(\epsilon_\mathrm{J})+\mathrm{Var}(\epsilon_\mathrm{J})=\sigma_\mathrm{J}^2+\sigma_\mathrm{H}^2,
\end{equation}
donde $\sigma_\mathrm{J}$ y $\sigma_\mathrm{H}$ son los promedios RMS de los
errores en las mediciones. Si además suponemos que
$\sigma_\mathrm{J} = \sigma_\mathrm{H} \equiv \sigma$, tenemos que
$\mathrm{Var}(x_\mathrm{J}-x_\mathrm{H})=2\sigma^2\Rightarrow\sigma=\sqrt{\mathrm{Var}(x_\mathrm{J}-x_\mathrm{H})}/\sqrt{2}$,
el cual usamos como el estimado del incertidumbre en estas mediciones
en la Sección~\ref{sec:error-shell}.


\chapter{Constante de conversión en las observaciones del HST}\chaptermark{constante de conversión} \label{AP : conversion EM}

En este Apéndice vamos a justificar la constante de conversión para
pasar de unidades del telescopio, $\unit{cuenta.s^{-1}.pix^{-1}}$, a
unidades físicas de brillo superficial,
$\unit{erg.cm^{-2}.s^{-1}.sr^{-1}}$, en las observaciones de H$\alpha$ en
el HST. Para esto vamos a utilizar la calibración de
\cite{Grosdidier:1998} en la cual calculó un flujo total de la
nebulosa de \SI{2.08e-10}{erg.s^{-1}.cm^{-2}} en las observaciones de
H$\alpha$ en el HST. En esta calibración quita la contaminación de la línea
$[\mathrm{N\scriptstyle{II}}]\,\lambda6548$ y del continuo. Por otro lado,
calculamos un flujo en unidades del telescopio al sumar las cuentas de
todos los píxeles de la nebulosa, después de aplicar una máscara para
eliminar la contribución de las estrellas, teniendo un valor de
\SI{64822.82}{cuenta.s^{-1}}. Esta máscara toma en cuenta a los
píxeles que se encuentra a una distancia mínima de \SI{1}{\arcsecond}
y a una distancia máxima de \SI{60}{\arcsecond} de la estrella WR 124,
así no estamos considerando la emisión por parte de la estrella, de
igual manera, no sumamos los píxeles que tuvieran un valor mayor a 3
para no incluir la emisión de las estrellas de campo. Además, tomamos
en cuenta el tamaño del píxel de
$\SI{0.1}{\arcsecond} \times \SI{0.1}{\arcsecond} = \SI{2.34e-13}{sr}$. Por
lo que nuestro factor de conversión está dado como
\begin{align}
    \nonumber
    \SI{1}{cuenta.s^{-1}.pix^{-1}}
    & =
    \frac{\num{2.08e-10}}{64822.82 \times \num{2.34e-13}} \;
    \unit{erg.cm^{-2}.s^{-1}.sr^{-1}} \\[\medskipamount]
    & \approx \SI{0.0137}{erg.cm^{-2}.s^{-1}.sr^{-1}}.
\end{align}




\chapter{Corrección en la estimación de los brillos}\label{App:brillos}

En la Sección \ref{Sec : comparacion-modelo} usamos los resultados
obtenidos a partir de las observaciones para comparar con el modelo.
Por lo que ahora vamos a calcular las correcciones a los brillos
estimados, tanto de la parte interna como de la cáscara, por efectos
instrumentales.

En el caso de la cáscara, vamos a ignorar estas correcciones debido a
que está bien resuelta y el brillo no se ve reducido por el PSF del
telescopio.

Por otro lado, para la parte interna tenemos un radio muy pequeño. De
hecho es casi un píxel en las observaciones del HST. Así que para esta
corrección vamos a considerar dos efectos instrumentales, uno por el
efecto del PSF y el otro por el tamaño del píxel.

Si asumimos que el perfil de brillo real tiene un perfil gaussiano
como función de la distancia $r$
\begin{equation}
B(r)= B_0 e^{-r^2/2\sigma_0}
\end{equation} 
tenemos que el flujo total está dado como 
\begin{equation}
F_0=\iint_{-\infty}^\infty B(r)dxdy=B_0 \pi r_\mathrm{eff}^2=B_02\pi \sigma_0^2.
\end{equation}

Para considerar estas correcciones por los dos efectos instrumentales,
vamos a asumir que estos también están dados por perfiles gaussianos.
De este modo, al convolucionar dos perfiles gaussianos con parámetros
$\sigma_1$ y $\sigma_2$, tenemos que
\begin{equation}
\sigma^2=\sigma_1^2+\sigma_2^2
\end{equation}
donde $\sigma$ sería el parámetro de los dos perfiles convolucionados.

En nuestro caso, lo que observamos es el perfil del brillo real
convolucionado con el perfil del PSF y el perfil de las rendijas de
los píxeles. Entonces
\begin{equation}
\sigma_\mathrm{obs}^2=\sigma_0^2+\sigma_\mathrm{PSF}^2+\sigma_\mathrm{pix}^2
\end{equation}
donde $\sigma_\mathrm{PSF}=\frac{W_\mathrm{PSF}}{2\sqrt{2\ln{2}}}$, siendo
$W_\mathrm{PSF}$ el ancho del PSF a la altura media, y
$\sigma_\mathrm{pix}=\frac{\Delta X_\mathrm{pix}}{\sqrt{2\pi}}$, siendo
$\Delta X_\mathrm{pix}$ el tamaño del píxel. Entonces, para comparar el los
brillos reales y los observados tenemos que
\begin{equation}
\frac{F_0}{F_\mathrm{obs}}=\frac{B_0\sigma_0^2}{B_\mathrm{obs}\sigma_\mathrm{obs}^2}=\frac{B_0}{B_\mathrm{obs}}\frac{\sigma_0^2}{\sigma_0^2+\sigma_\mathrm{PSF}^2+\sigma_\mathrm{pix}^2}
\end{equation} 
\begin{equation}
\Rightarrow \frac{B_0}{B_\mathrm{obs}}=1+\frac{\sigma_\mathrm{PSF}^2+\sigma_\mathrm{pix}^2}{\sigma_0^2}.
\end{equation}

Debido a que usamos un solo radio para todos los glóbulos, esta
corrección es solo una constante, y gracias a este valor, los datos
obtenidos a partir de las observaciones se ajustan muy bien al modelo
propuesto.


\chapter{Escalas de tiempo}

\section{Tiempo dinámico}

Usando los valores de la tabla \ref{tab:mean}, para el flujo
fotoevaporativo tenemos un tiempo dinámico
\begin{equation}
t_\mathrm{DF} = \frac{r_\mathrm{shell}}{v} = \frac{\SI{0.01}{pc}}{\SI{10}{km.s^{-1}}}\approx \SI{5.27e10}{s}  = \SI{1.67e3}{a}.
\end{equation}

\cite{Mancherko:2010} estima una velocidad de expansión para la
nebulosa de 42--\SI{46}{km.s^{-1}} por lo que para la nebulosa tenemos
un tiempo dinámico de
\begin{equation}
t_\mathrm{DN}= \frac{R_\mathrm{nebula}}{v_\mathrm{exp}}\approx\frac{\SI{1.5}{pc}}{\SI{46}{km.s^{-1}}}= \SI{1e12}{s}=\SI{3.18e4}{a}
\end{equation}

\section{Tiempo de recombinación}

Usando la densidad promedio de la tabla \ref{tab:mean} tenemos un
tiempo de recombinación
\begin{equation}
t_\mathrm{r} = \frac{1}{\alpha_\mathrm{B} n} \approx \SI{3.64e9}{s}= \SI{115.64}{a}
\end{equation}

Para el caso del tiempo de calentamiento-enfriamiento, vamos a
considerar que es de 3--5 veces menor que el tiempo de recombinación.
Esto considerando que
$t_\mathrm{c}=\frac{3P}{2\mathcal{L}}=\frac{3k_\mathrm{B}T}{\Lambda n}$ donde
$P=2nk_\mathrm{B}T$ es la presión y $\mathcal{L}=\Lambda n^2$, así tenemos que
$\frac{t_\mathrm{c}}{t_\mathrm{r}}=\frac{3k_\mathrm{B}T\alpha_\mathrm{B}}{\Lambda}$
y considerando que
$3k_\mathrm{B}T\approx\SI{4e-12}{erg},\;
\alpha_\mathrm{B}=\SI{2.6e-13}{cm^3.s^{-1}}$ y
$\Lambda\approx\SI{3e-24}{erg.cm^3.s^{-1}}$ entonces
\begin{equation}
    \frac{t_\mathrm{c}}{t_\mathrm{r}}\approx\frac{\SI{e-24}{erg.cm^3.s^{-1}}}{\SI{3e-24}{erg.cm^3.s^{-1}}}=\frac{1}{3}
\end{equation}

\section{Tiempo de vida de los glóbulos}

Para calcular el tiempo de vida de los glóbulos, primero vamos a
considerar que su tasa de pérdida de masa está dada por
\begin{equation}
    \dot{M} =\pi r_0^2n_\mathrm{i,0}c_\mathrm{s,i}m_\mathrm{H}
\end{equation}
usando los valores típicos de la tabla \ref{tab:mean}, tenemos que
$\dot{M} = \SI{4.91e-8}{\msun.a^{-1}}$. Por lo que el tiempo de vida
de los glóbulos es
\begin{equation}
    t_\mathrm{glo}=\frac{M_\mathrm{g}}{\dot{M}}=\SI{9.16e4}{a}
\end{equation}

\subsection{Comparación de las diferentes escalas de tiempo}

De lo anterior, tenemos que
\begin{equation}
    t_ \mathrm{cool} < t_\mathrm{r} \ll t_\mathrm{DF}  \ll t_\mathrm{DN} \sim t_\mathrm{glo}.
\end{equation}
Con lo que podemos asumir un equilibrio de ionización/recombinación,,
así como un equilibrio de calentamiento/enfriamiento dado que estos
dos procesos micro físicos ocurren en un tiempo de escala mucho menor
que los demás. También podemos justificar que el modelo propuesto es
estacionario ya que este tienen una escala de tiempo mucho menor que
la expansión de la nebulosa.

\chapter{Proyección de posición y velocidad}\label{AP : PV}

En la Sección \ref{sec: distrtibucion} analizamos las observaciones de
rendija larga de los glóbulos para poder determinar su distribución
tridimensional. A pesar de que no podemos medir directamente la
posición a lo largo de la vista de visión, si suponemos que los
glóbulos se alejan de la estrella WR 124 estrictamente de manera
radial a una velocidad $v(r)$, podemos usar la velocidad observada a
lo largo de la línea de visión para obtener esta componente espacial
que nos falta.

Consideremos un glóbulo (o un grupo de glóbulos) con coordenadas
cartesianas $(x,y,x)$, donde la estrella WR 124 se encuentra en el
origen. En este análisis vamos a considerar que el eje $y$ es paralelo
a nuestra línea de visión, por lo que las posiciones en $x$ y$z$ son
las posiciones en el plano del cielo, y las rendijas serán paralelas
al eje $z$. Por simplicidad vamos a usar las coordenadas
adimensionales $X=x/r$, $Y=y/r$ y $Z=z/r$, donde
$r = \left( x^2+y^2+z^2\right)^{1/2}$ es el radio. De esta manera, si
suponemos que el vector de velocidad, $(v_x,v_y,v_z)$, es paralelo al
vector de posición, entonces tenemos que $X=v_x/v$, $Y=v_y/v$ y
$Z=v_z/v$, donde $v = \left( v_x^2+v_y^2+v_z^2\right)^{1/2}$ es la
magnitud de la velocidad. La cual estamos considerando unicamente como
función del radio.

Considerando estas coordenadas adimensionales, en la Figura \ref{fig:
  ap PV rendija} podemos ver la esfera unitaria
\begin{equation}
X^2+Y^2+Z^2=1
\end{equation}
en el plano $(X,Z)$, el cual representa el plano del cielo en el que
vemos, y nuestra línea de visión sería el eje $Y$, el cual apunta
hacia adentro de la imagen. La línea azul paralela al eje $Z$
representa una de las rendijas usadas por \cite{Zavala:2022} en la
nebulosa. Notemos que si está rendija se encuentra en $X_0$, podemos
dibujar el círculo
\begin{equation}\label{eq : ap circulo}
 Y^2+Z^2=1-X_0^2 \equiv \xi^2
\end{equation}
en el plano paralelo a $(Y,Z)$ que pasa por $X_0$, como se ve en la
Figura \ref{fig:ap PV esfera3d}. Este círculo representa la posición
de la elipse tanto en posición como en velocidad. Mientras que las
elipses de la Figura \ref{fig:zavala_nudos} se caracterizan por los
semi-ejes en coordenadas físicas $\xi r$ en la dirección espacial a lo
largo de la rendija y $\xi v(r)$ en la velocidad a lo largo de la línea
de visión. En la Sección \ref{sec: distrtibucion} consideramos dos
casos limitantes para la velocidad. Un caso con una velocidad
constante $v(r)=\SI{46}{km.s^{-1}}$ y el caso donde la velocidad es
proporcional al radio
$v(r)=\SI{46}{km-s^{-1}}(r/\SI{20}{\arcsecond})$. Por lo que, si
tomamos $Z=z/r$ y $Y=v_y/v$ entonces el círculo dado por la Ecuación
(\ref{eq : ap circulo}) representa las elipses de la Figura
\ref{fig:zavala_nudos} pero normalizados.

Podemos observar también que el círculo $Y^2+Z^2=1-X_0^2$ tiene un
radio menor que la esfera unitaria para $X_0\neq0$ (ver Figura \ref{fig:
  ap PV modelo}), por lo que entre más lejos este $X_0$ del origen, el
círculo se hace más pequeño. Es por eso que las elipses de la Figura
\ref{fig:zavala_nudos} se van haciendo más pequeñas conforme la
rendija usada en la que se encuentran está más alejada de la estrella
WR 124.

Notemos que además que si $Z=0$, entonces tenemos el máximo en $Y$,
que es el radio del círculo $\sqrt{1-X_0^2}$. De igual manera para el
caso $Y=0$. Estas proyecciones son usadas para la discusión en la
Sección \ref{sec: distrtibucion}

\begin{figure}
    \centering
    \includegraphics[width=\textwidth]{imagenes_corregidas/PV 01.pdf}
    \caption{Visualización de como se vería una rendija que pasa en $X_0$ sobre la esfera unitaria. El plano $(X,Z)$ representa el plano en el cielo y nuestra línea de visión va en dirección del eje $Y$, hacia adentro de la imagen.}
    \label{fig: ap PV rendija}
\end{figure}

\begin{figure}
    \centering
    \includegraphics[width=\textwidth]{imagenes_corregidas/PV 02.pdf}
    \caption{El círculo $Y^2+Z^2=1-X_0^2$ con centro en $(0,0,X_0)$ esta sobre un plano paralelo al plano $(Y,Z)$.La flecha negra indica el vector de posición,o de velocidad ya que los estamos tomando como paralelos, y las flechas de colores son sus proyecciones en sus respectivos ejes.}
    \label{fig:ap PV esfera3d}
\end{figure}


\begin{figure}
    \centering
    \includegraphics[width=\textwidth]{imagenes_corregidas/PV 03.pdf}
    \caption{En la imagen de la izquierda vemos como se vería el esquema de Posición-Velocidad con los ejes normalizados a un radio $r$ y una velocidad $v(r)$, se puede apreciar que conforme $X_0$ se aleja del origen, nuestro círculo se hace cada vez más pequeños. Es por eso que en la imagen de la derecha las elipses se hacen más pequeñas conforme las rendijas están más alejadas de la rendija H, que es la que pasa por la estrella. }
    \label{fig: ap PV modelo}
\end{figure}


\chapter{Imágenes de ajustes}\label{App : ajustes}

En este Apéndice mostramos los ajustes realizados a los glóbulos. En
la parte izquierda mostramos el ajuste a los perfiles de brillo para
las observaciones del HST y a su lado la visualización de estos
ajustes en el filtro f656n del HST. Del lado derecho vemos el ajuste a
los perfiles de brillo para el gas ionizado usando los datos del JWST,
y a su lado la representación de estos ajustes en el filtro f090w del
JWST, ya que morfológicamente se parece a las imágenes del HST.

\begin{figure}[htb]
    \centering
    % Crear un nodo en la ubicación deseada
    \begin{tikzpicture}[overlay, remember picture]
        \node[anchor=south west, xshift=.2cm, yshift=3cm] at (current page.south west) {
            \includegraphics[width=1.7\textwidth]{Nuevas imagenes finales/Aj_01.pdf}
        };
    \end{tikzpicture}
    \caption{Visualización de los ajustes realizados a los glóbulos.}
    \label{fig:ajuestes_apendice}
\end{figure}

\newpage

\begin{tikzpicture}[overlay, remember picture]
  % La imagen se coloca en una posición específica
  \node[anchor=south west, xshift=1.3cm, yshift=.2cm] at (current page.south west) {
    \includegraphics[width=7.5in]{Nuevas imagenes finales/Aj_02.pdf}
  };
\end{tikzpicture}

\newpage

\begin{tikzpicture}[overlay, remember picture]
  % La imagen se coloca en una posición específica
  \node[anchor=south west, xshift=1.3cm, yshift=.2cm] at (current page.south west) {
    \includegraphics[width=7.5in]{Nuevas imagenes finales/Aj_03.pdf}
  };
\end{tikzpicture}

\newpage

\begin{tikzpicture}[overlay, remember picture]
  % La imagen se coloca en una posición específica
  \node[anchor=south west, xshift=1.3cm, yshift=.2cm] at (current page.south west) {
    \includegraphics[width=7.5in]{Nuevas imagenes finales/Aj_04.pdf}
  };
\end{tikzpicture}

\newpage

\begin{tikzpicture}[overlay, remember picture]
  % La imagen se coloca en una posición específica
  \node[anchor=south west, xshift=1.3cm, yshift=.2cm] at (current page.south west) {
    \includegraphics[width=7.5in]{Nuevas imagenes finales/Aj_05.pdf}
  };
\end{tikzpicture}

\newpage

\begin{tikzpicture}[overlay, remember picture]
  % La imagen se coloca en una posición específica
  \node[anchor=south west, xshift=1.3cm, yshift=.2cm] at (current page.south west) {
    \includegraphics[width=7.5in]{Nuevas imagenes finales/Aj_06.pdf}
  };
\end{tikzpicture}

\newpage

\begin{tikzpicture}[overlay, remember picture]
  % La imagen se coloca en una posición específica
  \node[anchor=south west, xshift=.5cm, yshift=5cm] at (current page.south west) {
    \includegraphics{Nuevas imagenes finales/Aj_07.pdf}
  };
\end{tikzpicture}


\bibliography{references}

\end{document}

%%% Local Variables:
%%% mode: LaTeX
%%% TeX-master: t
%%% End:
