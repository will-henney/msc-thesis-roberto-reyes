\documentclass{article}
\usepackage{graphicx} % Required for inserting images

\title{ideas}
\author{r.reyes }
\date{December 2023}

\begin{document}

\maketitle

\section{Modelo}

Para este modelo consideramos glóbulos relativamente pequeños, además de estar lo suficientemente lejos de una estrella masiva como para que su radiación UV llegue desde un pequeño ángulo e interactúe con el glóbulo, de esta manera podemos decir que la radiación incide perpendicularmente a la superficie del glóbulo.

Si suponemos una simetría esférica para el glóbulo, entonces podemos ver que tiene una simetría que con respecto a la radiación incidente, la cual tomaremos como el eje de simetría.  Por lo que trataremos este problema de manera unidimensional. 

En general en el medio interestelar hay muchas fuerzas que afectan el cambio en la materia, pero en este caso vamos a despreciar la fuerza de gravedad ya que consideramos que la estrella está lo suficientemente lejos como para atraer el glóbulo, así como tampoco vamos a considerar ninguna fuerza pr parte de algún campo magnético ya que...
Por lo que solo vamos a considerar las fuerzas de presión por parte del flujo fotoevaporativo y del viento exterior, como el viento estelar.

Si consideramos el tiempo en el que un bulto sale del glóbulo para interactuar con el medio podemos decir que este tiempo es mayo que el tiempo en el que los flujos transonicos se mantienen en equilibrio de ionización, por lo que podemos considerar un modelo estacionario que no cambia con el tiempo, con esto y considerando solo el eje de simetría, podemos incluso tomar el problema de manera radial sin vernos afectados por el ángulo.

Para el flujo fotoevaporativo por parte del glóbulo vamos a considerar las ecuaciones de hidrodinámica, la ecuación de conservación de masa ya que como dijimos antes, el tiempo en el que un bulto sale de esta interacción le toma bastante tiempo, además durante el proceso de ionización si bien, considerando solo hidrógeno molecular, el número de partículas se duplica, estos están en una parte donde el volumen es mayor que si están en l a superficie o dentro del glóbulo. Usaremos también la ecuación de Bernuoulli para un gas isotermo, donde la velocidad del sonido en este gas dependerá de la temperatura.Para a tercer ecuación tomaremos cuando las dos presiones, tanto de flujo fotoevaporativo como del viento exterior sean iguales, para la presión exterior consideraremos tanto la presión térmica como la hidrodinámica, mientras que para la del flujo fotoevaporativo la consideraremos que inicialmente tiene un número de Mach de 1.


Que este modelo sea estacionario no quiere decir que no haya reacciones o cambios, es más bien para que podamos usar las ecuaciones antes mencionadas. Esto es porque estamos considerando que ya está en equilibrio de ionización y el tiempo en el que está ocurriendo esta interacción ...
\end{document}
