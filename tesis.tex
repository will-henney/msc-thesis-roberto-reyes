\documentclass{book}
\usepackage{graphicx} % Required for inserting images
\usepackage{subcaption}

\usepackage[utf8]{inputenc}
\usepackage[T1]{fontenc}
\usepackage{listings}
\usepackage{appendix}
\usepackage{siunitx}
\usepackage{multirow}
\usepackage[version=4]{mhchem}
\usepackage{natbib}
\usepackage{booktabs}
\usepackage{bachelorstitlepageUNAM}
\usepackage[spanish,es-nodecimaldot,es-tabla]{babel}
\usepackage{tikz} 
\usepackage{tocloft}
\graphicspath{{./figs/}}
\usepackage{setspace}
\usepackage{hyperref}
\usepackage{float}              % Para usar [H] en las figures

\usepackage{tikz}
\usepackage{lipsum}
\usepackage{caption}

\DeclareSIUnit\msun{\text{M\ensuremath{_\odot}}}
\DeclareSIUnit\mum{\text{\ensuremath{\mu}m}}
\usepackage{astrojournals}
\bibliographystyle{aasjournal}
\renewcommand{\lstlistingname}{Algoritmus}

\title{Avances de tesis}
\author{r.reyes }
\date{December 2023}

\begin{document}

    \begin{titlepage}
        \thispagestyle{empty}
        \begin{minipage}[c][0.17\textheight][c]{0.25\textwidth}
            \begin{center}
                \includegraphics[width=3.5cm, height=3.5cm]{Escudo-UNAM.pdf}
            \end{center}
        \end{minipage}
        \begin{minipage}[c][0.195\textheight][t]{0.75\textwidth}
            \begin{center}
                \vspace{0.3cm}
                \textsc{\large Universidad Nacional Aut\'onoma de M\'exico}\\[0.5cm]
                \vspace{0.3cm}
                \hrule height2.5pt
                \vspace{.2cm}
                \hrule height1pt
                \vspace{.8cm}
                \textsc{Instituto de Radioastronomía y astrofísica}\\[0.5cm] %
            \end{center}
        \end{minipage}

        \begin{minipage}[c][0.81\textheight][t]{0.25\textwidth}
            \vspace*{5mm}
            \begin{center}
                \hskip2.0mm
                \vrule width1pt height13cm 
                \vspace{5mm}
                \hskip2pt
                \vrule width2.5pt height13cm
                \hskip2mm
                \vrule width1pt height13cm \\
                \vspace{5mm}
                \includegraphics[height=4.0cm]{logoIRyAnegro_1.png}
            \end{center}
        \end{minipage}
        \begin{minipage}[c][0.81\textheight][t]{0.75\textwidth}
            \begin{center}
                \vspace{1cm}

                {\large\scshape El impacto del viento estelar y la radiación en glóbulos neutros alrededor de una estrella Wolf-Rayet}\\[.2in]

                \vspace{2cm}            

                \textsc{\LARGE T\hspace{1.5cm}E\hspace{1.5cm}S\hspace{1.5cm}I\hspace{1.5cm}S}\\[0.5cm]
                \textsc{\large que para obtener el grado de:}\\[0.5cm]
                \textsc{\large Maestro en ciencias (astrofísica)}\\[0.5cm]
                \textsc{\large presenta:}\\[0.5cm]
                \textsc{\large {Roberto Reyes Cabañas}}\\[2cm]          

                \vspace{0.5cm}

                {\large\scshape Tutor:\\[0.3cm] {Dr. William John Henney }}\\ [.2in]

                \vspace{0.5cm}

                \large{Morelia, Michoacán,}{ }{2024}
            \end{center}
        \end{minipage}
    \end{titlepage}

%\maketitle

\chapter*{Resumen}

La nebulosa circunestelar M1-67 alrededor de la estrella Wolf-Rayet WR 124 contiene cientos de pequeños glóbulos neutros, como revelan imágenes recientes del  James Webb Space Telescope (JWST). La emisión ionizada de la nebulosa muestra un patrón intrincado de cáscaras y filamentos, muchos de los cuales parecen asociados con los glóbulos pero desplazados hacia la estrella central. Proponemos un modelo simple para la nebulosa en el que los flujos de fotoevaporación desde la superficie irradiada de los glóbulos interactúan con el viento estelar de la estrella Wolf-Rayet para formar cáscaras de emisión hemisféricas. Probamos este modelo contra imágenes del JWST y H$\alpha$ del Hubble Space Telescope (HST) de la nebulosa, encontrando una buena concordancia para los glóbulos mejor observados y más aislados. El modelo proporciona una explicación física para la morfología observada de la nebulosa y los glóbulos, y sugiere que los glóbulos están protegidos hidrodinámicamente del viento estelar por los flujos de fotoevaporación. Derivamos una estimación independiente de la fuerza del viento estelar, que es consistente con valores previamente derivados de la modelación de la atmósfera estelar. También somos capaces de restringir la distribución tridimensional de los glóbulos.

\newpage

\tableofcontents

\newpage

\chapter{Introducción}\label{Capitulo 1:introduccion}

Los \textit{glóbulos} son concentraciones densas de gas y polvo en el medio interestelar que se cree que se forman por inestabilidades térmicas, colapso gravitacional o turbulencia \citep{Ballesteros:2011,Padoan:2002}. Estos glóbulos se pueden formar en regiones de formación estelar masiva o en nebulosas alrededor de estrellas evolucionadas, como en nebulosas planetarias \citep{O'Dell:2007}.

En general los glóbulos tienen una gran variedad de tamaños. Por ejemplo, cuando nos referimos a los glóbulos en regiones de formación estelar masiva comúnmente suelen ser de gran tamaño, $\sim$\SI{0.1}{pc} \citep{Schenider:2016}, mientras que en nebulosas alrededor de estrellas evolucionadas son de un tamaño más pequeño, $\sim$\SI{e-2}{pc} \citep{GFGahm:2013}.

Los primeros glóbulos fueron observados por Bart Bok en 1940. Como podemos ver en la Figura \ref{fig:Banard}, debido a que las estrellas de fondo sufren enrojecimiento por polvo, estos glóbulos se pueden apreciar como nubes oscuras ya que tienen gran cantidad de gas neutro y polvo. Estos glóbulos contienen principalmente hidrógeno molecular en su interior, así como también pueden tener otras moléculas, metales e incluso algunos silicatos. Si bien puede haber formación estelar en su interior, no se puede observar la radiación ionizante de estas estrellas ya que es absorbida por el hidrógeno neutro y el polvo que hay entre las estrellas y el observador, es por eso que se ven oscuras.

Cuando los glóbulos se encuentran en regiones de formación estelar masiva, estos pueden interactuar con la radiación ultravioleta (UV) de las estrellas jóvenes masivas que se encuentren cerca, o con la de la estrella central en el caso de que los glóbulos estén en una nebulosa circunestelar. En estos casos se puede observar el frente de ionización como un borde brillante de emisión (ver Figuras \ref{fig:Pillars} y \ref{fig:nudos}).

\begin{figure}[htb]
    \centering
    \includegraphics[width=0.85\textwidth]{images Chapter 1/C1_Bok_globule.jpg}
    \caption{Ejemplo de un glóbulo de Bok. Imagen de Banard 68 visto con Very Large Telescope FORS1 en 440 nm, 557 nm y 768 nm. Se puede apreciar una zona oscura el cual es el glóbulo y lo que pareciera ser enrojecimiento de las estrellas por polvo en la superficie del glóbulo. En esta imagen podemos ver que no hay evidencia de alguna fotoevaporación externa por parte de estrellas. \citep{Alves:2001}.}
    \label{fig:Banard}
\end{figure}

Esta interacción entre estrellas y glóbulos se puede dar a diferentes escalas, lo que nos da una gran variedad de estructuras. Entre las de mayor tamaño se encuentran lo que parecen ser columnas, pilares o trompas de elefantes, como se les conoce en la literatura, que llegan a tener un tamaño de $\sim$\SI{1}{pc} y una densidad del orden de \SI{e3}{cm^{-3}}. Esta interacciones también se puede dar dentro de regiones HII, como vemos en la Figura \ref{fig:Pillars}.

\begin{figure}[htb]
    \centering
    \includegraphics[width=1 \textwidth]{images Chapter 1/C1_Pillars.jpg}
    \caption{En \textbf{A} vemos dos ejemplos de pilares, donde la imagen derecha de cada ejemplo es vista a \SI{2.12}{\mu m} (\SI{}{H_2}) y la imagen izquierda es \SI{}{H_2-Br_{\gamma}} \citep{Hartigan:2015}. \textbf{B} es un ejemplo de una trompa de elefante, es una imagen de M16 tomada con WFPC2 con el filtro F656N, los \SI{30}{\arcsecond} corresponden \SI{9e17}{cm} (\SI{0.29}{pc}) \citep{JJHester:1996}. \textbf{C} es el outflow de Tadpole globule, el cual consta del sistema HH900 jet+outflow, la imagen de abajo es vista en \SI{}{H\alpha} con el continuo 
    \citep{MeganReiter:2019}. }
    \label{fig:Pillars}
\end{figure}

En escalas más pequeñas están lo que se conoce como EGGs (Evaporating Gaseous Globule) los cuales tienen tamaños de $\sim$\SI{e-2}{pc}, y los proplyds que tienen tamaños $\le\SI{e-2}{pc}$. 
Estos glóbulos no solo se encuentran en regiones de formación estelar, también hay dentro de nebulosas alrededor de estrellas evolucionadas donde se les conoce más comúnmente como \textit{nudos}, un ejemplo de esto es la imagen \textbf{D} de la Figura \ref{fig:nudos}. En este trabajo estudiaremos más a fondo los nudos que hay en una nebulosa alrededor de cierta estrella evolucionada.

\begin{figure}[htb]
    \centering
    \includegraphics[width=1 \textwidth]{images Chapter 1/C1_Globulettes.jpg}
    \caption{En \textbf{A} vemos proplyds con su bowshock en Orion tomado con HST planetary camera, la barra negra indica una medida de \SI{1}{\arcsecond} que corresponde a 430 AU (\SI{2e-3}{pc}) \citep{Garcia-Arredondo:2001}. \textbf{B} son ejemplos de EGGs en Carina, tomado con WFC3, ACS, WFPC2, las líneas blancas de \SI{1}{\arcsecond} corresponden a una distancia física de \SI{e-2}{pc} \citep{Mesa-Delgado:2016}. \textbf{C} es el globulette denso RN88 visto en \SI{}{H_\alpha} con un diámetro de \SI{6}{\arcsecond} (\SI{4e-2}{pc}) en la nebulosa de Rosette \citep{GFGahm:2013}. \textbf{D} son ejemplos de nudos en la nebulosa de la Hélice, los mosaicos tienen una medida de \SI{47.5}{\arcsecond}$\times$\SI{44.8}{\arcsecond} (\SI{4.76e-2}{pc}$\times$\SI{4.49e-2}{pc}) \citep{O'Dell:2007}. }
    \label{fig:nudos}
\end{figure}



\section{Flujos de fotoevaporación ionizada} \label{Sec:fluijos fotoevaporativos}

Todos los ejemplos de las Figuras \ref{fig:Banard}, \ref{fig:Pillars} y \ref{fig:nudos} se encuentran ya sea en regiones de formación estelar o en nebulosas alrededor de estrellas evolucionadas. Lo interesante en todos estos ejemplos es la forma que toman al interaccionar con las estrellas más masivas que se encuentran cerca, esto para los glóbulos que se encuentran en regiones de formación estelar.  Mientras que los que se encuentran en nebulosas planetarias interactúan con la estrella evolucionada. Durante estas interacciones, en algunos casos podemos ver lo que se conoce como \textit{flujos fotoevaporativos}, los cuales explicaremos mejor a continuación.

Cuando la radiación ionizante incide en la superficie del glóbulo, este comienza a ionizar el gas neutro. A este flujo de gas ionizado que sale de la base del glóbulo y que viaja en dirección a la fuente ionizante se le conoce como flujo fotoevaporativo. 

En el caso de las regiones de formación estelar podemos considerar una estrella masiva y una nube densa de gas neutro. Por lo que para poder ver el flujo fotoevaporativo es necesario que la estrella sea masiva, o que tenga un gran flujo ionizante como para poder ionizar el gas neutro, de lo contrario no podremos ver el flujo fotoevaporativo. Recordemos que en las regiones de formación estelar hay muchas estrellas nuevas de baja masa que emiten principalmente en radio o infrarrojo, por lo que no todas las estrellas nuevas pueden ionizar el gas neutro.

\cite{OortySpitzer_1955} explican de manera detallada como es la interacción entre una estrella tipo O y una nube interestelar de gas neutro. La cual se puede observar en regiones de formación estelar masiva. Ellos consideran tres elementos importantes para esto: La estrella ionizante, la nube interestelar de gas neutro y la región que hay entre la estrella y la nube interestelar. La nube interestelar debe ser mucho más densa y fría que la región que hay entre la estrella y la nube como vemos en la Figura \ref{kahn_zones}.

\begin{figure}[htb]
    \centering
    \includegraphics[width= \textwidth]{artesanales/ImgFi01-5.pdf}
    \caption{Esquema inicial utilizado en \cite{OortySpitzer_1955}, donde podemos apreciar que la nube es más fría y densa que la región que hay entre la nube y la estrella.}
    \label{kahn_zones}
\end{figure}

Cuando la radiación UV comienza a calentar el gas de la nube, el gas ionizado comienza a expandirse en dirección a la estrella, esto ya que en esta dirección la densidad es menor que la de la nube y puede expandirse libremente (Figura \ref{fig:evolucion de la nube}). 

\begin{figure}[htb]
    \centering
    \includegraphics[width= \textwidth]{ultimos/implosion_inicial.pdf}
    \caption{Esquema de la implosión inicial. Cuando la radiación ionizante (flechas azules) inciden en el glóbulo (color verde)  produce que el gas ionizado (color naranja) viaje en dirección a la estrella como lo muestran las flechas negras. En esta fase se produce un choque interno que hará que la nube se comprima. Tanto el choque como el frente de ionización  (color rojo) viajan hacia el centro del glóbulo como lo muestran las flechas amarillas.}
    \label{fig:evolucion de la nube}
\end{figure}

En un inicio esta radiación ioniza el gas neutro de la nube a una tasa muy rápida. Esto causa que una gran cantidad partículas ionizadas, provenientes de la nube, viajen en dirección a la estrella. Conforme esto va evolucionando se va formando una capa aislante alrededor de la nube \citep{OortySpitzer_1955}. Esta capa aislante esta conformada por el gas ionizado, y puede proteger a la nube de flujos o vientos externos, así como de la radiación.

Durante esta interacción tenemos tanto un frente de ionización como un choque interno que viajan a través de la nube a la parte trasera. Al inicio estos dos tienen una velocidad similar de $\sim\SI{10}{km.s^{-1}}$, pero una vez que las recombinaciones se vuelven importantes en la capa aislante, el frente de ionización comienza a desacelerar. Mientras que el choque interno hace que la nube se comprima \citep{Bertoldi_1989}.

No siempre podemos ver un flujo fotoevaporativo por parte de las nubes en este tipo de interacción, para esto \cite{Bertoldi_1989} nos dice que si el parámetro de ionización, definido como 
\begin{equation}
    \Gamma  = \frac{F_i}{n_0 c}
\end{equation}
donde $F_i$ es el flujo incidente del continuo de Lyman, $n_0$ la densidad del gas neutro,y $c$ la velocidad de la luz, es menor que \SI{e-7}{}, entonces la radiación ionizante incidente no tendrá un efecto dinámico sobre la nube por lo que no tendremos un flujo fotoevaporativo por parte de la nube. Por otro lado, si 
\begin{equation}
    \frac{F_i}{2\alpha_i r_0 n_0}>1
\end{equation}
donde $\alpha_i$ es el coeficiente de recombinación a todos los estados, excepto al nivel base y $r_0$ el radio de la nube, entonces la nube se ionizará por completo.

%En un inicio esta radiación ioniza el gas a una tasa muy rápida causando una ``tormenta de partículas ionizantes'' que viene por parte de la nube, conforme esto va evolucionando se crea una capa aislante alrededor de la nube. Esta radiación produce un choque ionizante lo cual provoca que la nube se comprima. También se produce un frente de ionización que avanza hacia la nube haciendo que esta se vaya alejando de la estrella. Mientras esto ocurre hay un choque interno que va viajando a través de la nube a la parte trasera \citep{Bertoldi_1989}. Para que esto ocurra \cite{Kahn:1954} da cierto criterios, en los cuales dice que la radiación no debe ser muy débil o demasiado fuerte.

\section{Estrellas Wolf-Rayet y sus vientos}

Las estrellas WR (Wolf-Rayet) son estrellas evolucionadas de estrellas masivas, como estrellas tipo O, que tienen una alta pérdida de masa,  $\sim$2--\SI{10e-5}{\msun/yr}, debido a sus fuertes vientos, que pueden ser del orden de $\sim$\SI{1000}{km/s}, los cuales provocan sus líneas de emisión anchas \citep{Hamman:2006}. Estas estrellas fueron nombradas así después de que Charles Wolf y Georges Rayet identificaran 3 estrellas en Cygnus con sus  anchas líneas de emisión de C, N, O y He que las caracterizan muy bien \citep{WR:ref}. 

Típicamente son estrellas de 10--\SI{25}{\msun} que son descendientes de estrellas tipo O. Se caracterizan por tener intensas líneas de emisión y emisión libre-libre en longitudes de onda IR-mm-cm \citep{crowther:2007}.

A estas estrellas se les clasifica por el cociente que hay entre sus líneas de emisión intensa que las caracterizan. \cite{VanDerHutch:2001} las clasifica principalmente como tipo WN a aquellas que son abundantes en He y N, tipo WC a las que son abundantes en He y C, y tipo WO a aquellas que son abundantes en He y O. Si bien muchas de estas estrellas no tienen hidrógeno en su atmósfera estelar, en algunas podemos detectar una cantidad considerable de H en su atmósfera, por lo que se les pone también una ``h'' \citep{SSM:1996}.

%\begin{table}[h]
 %   \begin{center}
  %      \begin{tabular}{c c}
   %     \toprule
    %    \multicolumn{2}{c}{Clasificación de estrellas WR} \\ \midrule
     %   Índice espectral    & Tipo de estrellas\\ \midrule
      %  WN2--5     & Tipo temprana, WNE (Early WN)\\
       % WN7--9 & Tipo tardía, WNL (Late WN) \\
        %WC4--6 & WCE\\
        %WC7--9 & WCL \\ \bottomrule
        %\end{tabular}
    %\caption{Principal clasificación de estrellas WR  según sus líneas de emisión. Dentro de las WC, las que se clasifican como WC6 pueden ser de tipo tardía o temprana.}
    %\label{tab:WR-clasificacion}
    %\end{center}
%\end{table}


\section{Nebulosa M1-67}

M1-67 es la nebulosa circunestelar alrededor de la estrella WR 124, que es del tipo WN8h. 
Se han realizado varios estudios en la nebulosa M1-67 en la que se han propuesto modelos de su estructura tridimensional usando espectroscopia de rendija larga \citep{Zavala:2022}. \cite{Marcel:2021} usa las líneas de $[\mathrm{S \scriptstyle{II}}]\lambda6717$ y $\lambda6731$ para encontrar que la densidad electrónica decae con el radio, encontrando que en las partes más cercanas a la estrella la densidad electrónica es de $\sim$\SI{2000}{cm^{-3}}, mientras que en las partes más lejanas, a unos \SI{40}{\arcsecond}, la densidad electrónica es de $\sim\SI{500}{cm^{-3}}$. \cite{Grosdidier:1998} encuentra que el brillo superficial de H$\alpha$ también decae con el radio como $r^{-0.8}$. \cite{Mancherko:2010} midió una velocidad de expansión de la nebulosa de \SI{46}{km.s^{-1}} usando usando las observaciones de 1997 \citep{Grosdidier:1998} y 2008, la cual coincide con la medición de \cite{Zavala:2022}. 

En las Figuras \ref{fig:M1-67HST} y \ref{fig:M1-67JWST} se puede apreciar que esta nebulosa tiene una estructura muy compleja. \cite{Grosdidier:1998} detectó unos puntos muy brillantes y densos, pero no quedaba muy claro cuál era su naturaleza. En el capítulo \ref{Chapter : 3} hablaremos más a detalle de cómo estos puntos brillantes y densos en realidad son glóbulos que se localizan en gran parte de la nebulosa, estos glóbulos se pudieron detectar gracias a las imágenes del  Telescopio Espacial James Webb (JWST por sus siglas en inglés), ya que tienen una mayor resolución que las imágenes del Telescopio Espacial Hubble (HST por sus siglas en inglés), además de que también se tiene una gran variedad de filtros con el JWST.

A lo largo de la tesis vamos a usar los datos de la tabla \ref{tab:parametros WR-124}. Con la distancia a la estrella, $D$, podremos dar distancias físicas de la siguiente manera
\begin{equation}
    \Big[\frac{R}{\mathrm{AU}}\Big]=\Big[\frac{D}{\mathrm{pc}}\Big]\Big[\frac{\theta}{\mathrm{arcsec}}\Big]
\end{equation}
donde $D$ es la distancia a la estrella y $\theta$ la separación que medimos directamente de las observaciones, la tasa de pérdida de masa, $\dot{M}$, y la velocidad terminal del viento estelar, $v_\infty$, para calcular la presión hidrodinámica (presión RAM en inglés) del viento estelar, así como la tasa de fotones ionizantes de la estrella para calcular la presión de radiación.

\begin{table}[htb]
    \centering
    \begin{tabular}{c c c}
        \toprule
        \multicolumn{3}{c}{Parámetros de la estrella WR 124} \\ \midrule
         $D$ & 5.429$\pm$\SI{.54}{kpc} & J. Arthur, priv.~comm.\\
         $v_\infty$ & \SI{710}{km/s}  & \cite{Hamman:2006}\\
         $\dot{M}$ & $10^{-4.7^{+.6}_{-.25}}$\unit{M_\odot/años}  & \cite{Crowther:1999}\\
         $S_*$ & \SI{1.25e49}{s^{-1}} & \cite{crowther:2007}  \\
         %$F_{H_\alpha}$ & \SI{3e-14}{erg.cm^{-2}.s^{-1}} & \cite{Grosdidier:1998}\\
         \bottomrule
    \end{tabular}
    \caption{Parámetros de WR 124. La incertidumbre en la tasa de pérdida de masa es considerando los diferentes resultados en la literatura.}
    \label{tab:parametros WR-124}
\end{table}

\subsection{Observaciones con HST}

Para las observaciones con el Telescopio Espacial Hubble (HST por sus siglas en inglés) utilizamos los datos del Hubble Legacy Archive, los cuales son tipo imagen en formato FITS con un nivel de calibración 3, este nivel de calibración es un mosaico de la combinación de varias imágenes para cubrir un área del cielo\footnote{Para más detalles ver \url{https://hla.stsci.edu/hla_faq.html\#productlevels}}.  Se observó en el filtro F656N\footnote{En el Apéndice \ref{App: Filtros} vemos el rango en que observa este filtro.} utilizando la cámara Wide Field and Planetary Camera 2 (WFPC2) para ver la emisión en \unit{H\alpha}. Utilizamos las observaciones de 1997 con ID de la propuesta 6787, las cuales tienen un tiempo de exposición de \SI{10 216}{s} y es una combinación de 10 exposiciones. Para las observaciones de 2008 con ID de la propuesta 11137, estas tienen un tiempo de exposición de \SI{4200}{s} y es una combinación de 8 exposiciones. Usando los otros filtros y espectroscopia, notamos que la contaminación por parte del NII y el continuo en el filtro H$\alpha$ es despreciable.

\begin{figure}[htb]
    \centering
    \includegraphics[width=0.9\textwidth]{m1-67-comp-full-hst.jpg}
    \caption{Imagen de M1-67 con el filtro \unit{H\alpha} \citep{Grosdidier:1998}. https://esawebb.org/images/weic2307f/}
    \label{fig:M1-67HST}
\end{figure}

\subsection{Observaciones con JWST}

Para las observaciones con el Telescopio Espacial James Webb (JWST por sus siglas en inglés) usamos las observaciones propuestas por Klaus M. Pontoppidan con número de propuesta 2730. Se usaron los filtros F090W, F150W, F210M, F335M, F444W, F470N de la NIRCAM\footnote{En el Apéndice \ref{App: Filtros} vemos el rango en que observa cada uno de estos filtros.} con un tiempo de exposición de \SI{2662.72}{s}. Estos archivos son tipo imagen de nivel 3, este nivel de imagen es un mosaico de la combinación de varias imágenes para cubrir un área del cielo, y están en formato FITS.

Con la gran variedad de filtros del JWST podemos usar combinaciones para ver diferentes tipos de emisión y gracias a su gran resolución ver mejor las diferentes estructuras.

A diferencia de la imagen en H$\alpha$, los filtros que se utilizaron para las observaciones con el JWST tienen bandas anchas. En estas bandas podemos encontrar la contribución de diferentes mecanismos de emisión como lo es el continuo de la estrella, algunas líneas de emisión de la nebulosa, el continuo dispersado por polvo, así como su emisión térmica y también podemos ver la emisión de algunas bandas de Hidrocarburos Aromáticos Policíclicos  (PAHs por sus siglas en inglés).

\begin{figure}[htb]
    \centering
    \includegraphics[width=0.9\textwidth]{M1-67-JWST.jpg}
    \caption{Imagen de M1-67 con JWST. Vemos la composición de los filtros f444w (gris), f335m (rojo), f210m (verde), f150w (verde turquesa) y f090w (azul). https://www.flickr.com/photos/geckzilla/52757287572/}
    \label{fig:M1-67JWST}
\end{figure}

\section{Estructura de la tesis}

En esta tesis proponemos un modelo simple para explicar como interactúa el flujo fotoevaporativo transónico de un glóbulo con una presión externa. Este modelo lo vamos a aplicar  a los nudos que se encuentran en la nebulosa M1-67, por lo que se explicará la interacción del flujo fotoevaporativo con la presión hidrodinámica (presión RAM) del viento estelar de la estrella WR-124.

En el Capítulo 2 veremos como es que la interacción entre dos flujos supersónicos crean una cáscara chocada delgada. Con base en esto, describiremos el modelo hidrodinámico estacionario, en el cual el flujo fotoevaporativo de un glóbulo interactúa con una presión externa. En esta interacción también podemos ver una cáscara chocada.

En el Capítulo 3 hablaremos de como encontramos los nudos en la nebulosa M1-67, así como también se puede ver evidencia observacional de la interacción que hay entre el flujo fotoevaporativo de los nudos y el viento estelar de la estrella WR 124.

En el Capítulo 4 vamos aplicar este modelo a los nudos encontrados. Para esto haremos un ajuste a los perfiles de brillo radial, del cual podemos encontrar las mediciones de los nudos y sus cáscaras chocadas. De igual manera podemos encontrar la densidad de gas ionizado usando la medida de emisión

En el Capítulo 5 vamos a comparar estos resultados obtenidos con los valores teóricos del modelo que proponemos. Con base en esta comparación, se pudo obtener una mejor idea de como es la distribución de los nudos en la nebulosa.  

\chapter{Modelos analíticos de flujos fotoevaporativos interactuando con una presión externa}
\chaptermark{Modelo analítico}\label{Chapter : Modelo}

En este capítulo vamos a describir el modelo que se propone para explicar la interacción que hay entre el flujo fotoevaporativo de un glóbulo y una presión externa. Este modelo en principio se puede aplicar a cualquier tipo de glóbulo como los que se mencionaron en el Capítulo \ref{Capitulo 1:introduccion}. 

En este trabajo en especial vamos a tratar la interacción del flujo fotoevaporativo de los glóbulos\footnote{Llamaremos glóbulos a los nudos que hay en la nebulosa por simplicidad.} en la nebulosa M1-67 y la presión RAM por parte del viento estelar de la estrella WR 124. En el Capítulo \ref{Chapter : 3} hablaremos más acerca de cómo encontramos estos glóbulos en la nebulosa M1-67.

Para esto hemos considerado que ya han pasado todas las fases mencionadas en la Sección \ref{Sec:fluijos fotoevaporativos} y ahora estamos en un equilibrio de ionización. La forma del glóbulo en este modelo será esférico por simplicidad.

En esta interacción entre el flujo fotoevaporativo y el viento estelar, los cuales son supersónicos, se producen dos zonas chocadas y entre estas dos zonas una discontinuidad de contacto como se describe en la Figura \ref{fig:zones}. De estas zonas esperamos ver solo el flujo fotoevaporativo chocado y no el viento estelar chocado ya que este último es menos denso además de que es no radiativo y su longitud de enfriamiento es más grande que la región de interacción.

\begin{figure}[htb]
    \centering    \includegraphics[width=\textwidth]{Nuevas imagenes finales/C2_zone_new_01.pdf}
    \caption{La interacción entre el flujo fotoevaporativo y el viento estelar de una estrella forma 4 zonas. El círculo naranja es el glóbulo. La línea punteada azul que une el centro del  glóbulo con el centro de la estrella es el eje de simetría que vamos a considerar en el modelo, la estrella se localiza en el otro extremo de la línea punteada. Vemos como en este eje de simetría tanto el viento estelar (líneas amarillas) como la radiación inciden de forma perpendicular a la base del glóbulo y en dirección contraria viaja el flujo fotoevaporativo (líneas rojas). La zona 1 es donde el flujo fotoevaporativo sale de la superficie del glóbulo con un número de Mach igual a 1 y va aumentando. La zona 2 es el flujo fotoevaporativo chocado con el viento estelar, la cual esperamos ver en las observaciones como una cáscara y nos vamos a referir a ella como la cáscara chocada. La zona 3 es el viento estelar chocado con el flujo fotoevaporativo y la zona 4 es donde viaja el viento estelar supersónico, el cual es menos denso que el flujo fotoevaporativo. La discontinuidad de contacto se da entre las zonas 2 y 3, la línea gris.}
    \label{fig:zones}
\end{figure}

\begin{figure}[htb]
    \centering    \includegraphics[width=0.8\textwidth]{images Chapter 2/C2_Canto.jpg}
    \caption{Interacción de dos flujos supersónicos las cuales son producidas por dos fuentes (puntos negro en el eje z) a una distancia D. En esta interacción se produce una cáscara delgada $R(\theta)$ cuando los flujos llegan a un equilibrio. Para este problema se considera simetría cilíndrica \citep{Canto:1996}.}
    \label{fig:Canto1}
\end{figure}

\begin{figure}[htb]
    \centering    \includegraphics[width=\textwidth]{artesanales/ImgFi01-2.pdf}
    \caption{La radiación y el viento estelar viajan hacia el glóbulo como lo indican las flechas azules y vemos que inciden de manera perpendicular a la superficie del glóbulo en el eje de simetría. Mientras que el flujo fotoevaporativo por parte del glóbulo viaja como lo indican las flechas rojas, y por lo tanto en la interacción entre este flujo fotoevaporativo y el viento estelar debemos considerar cierto ángulo si no estamos en el eje de simetría.}
    \label{fig:cilindross}
\end{figure}

\cite{Canto:1996} trata de una manera formal la interacción entre dos flujos supersónicos en la cual considera dos fuentes separadas a una distancia D. Cuando estos flujos que están interaccionando llegan a un equilibrio de presiones se forma una cáscara delgada como se puede ver en la Figura \ref{fig:Canto1}, por lo que esperamos que la zona 2 de la Figura \ref{fig:zones} sea delgada.

%\begin{figure}[h]
%    \centering    \includegraphics[width=0.75\textwidth]{images Chapter 2/Chp2_cilinders.jpg}
 %   \caption{Si consideramos una simetría cilíndrica con un radio muy pequeño, podemos tratar este problema de manera radial y de tal manera que la radiación y el viento estelar inciden de manera perpendicular al glóbulo.}
 %   \label{fig:cilinders}
%\end{figure}

\section{Modelo hidrodinámico estacionario}

Para nuestro modelo es importante mencionar que no estamos considerando ninguna fuerza de gravedad por parte de la estrella o del mismo glóbulo, así como tampoco ninguna otra fuerza externa (ver Apéndice \ref{App:fuerzas}). Vamos a considerar que solo el glóbulo está dominado por un campo magnético, mientras que para el gas ionizado no vamos a considerar ningún campo magnético.

Para este trabajo en particular solo vamos a considerar como presión externa la presión RAM del viento estelar por parte de la estrella WR 124.

Tomando en cuenta que el tiempo en el que ocurren las fases mencionadas en la Sección \ref{Sec:fluijos fotoevaporativos} es muy corto comparado con el tiempo de interacción que hay entre el flujo fotoevaporativo y el viento estelar, vamos a suponer que la capa aislante producida por la radiación UV ya se ha formado y ahora estamos en equilibrio de ionización. Por lo que vamos a considerar este modelo como estacionario, es decir, que los tamaños del glóbulo y de la cáscara chocada los vamos a tomar como constantes ya que no cambiaran sus tamaños de manera significativa.

En la Figura \ref{fig:cilindross} vemos que podemos simplificar este problema si ponemos un cilindro de radio pequeño alrededor del eje de simetría.  En este cilindro podemos ignorar los movimientos transversales ya que los gradientes en estas direcciones son muy pequeños si los comparamos con los gradientes en la dirección axial. Alrededor del eje de simetría podemos ver como tanto la radiación UV y el viento estelar inciden de manera perpendicular a la base del glóbulo y viajan en dirección contraria el flujo fotoevaporativo por parte del glóbulo. Por lo que primero vamos a resolver este problema solo en el eje de simetría, ya que aquí se vuelve unidimensional. Más adelante vamos a considerar qué pasaría si resolvemos este problema considerando un ángulo $\theta$ con respecto al eje de simetría.

% Balance between ionizations and recombinations

\section{Ecuación de estado y equilibrio de ionización}

Para este caso vamos a considerar que el gas que está interaccionando con el viento estelar de la estrella es un gas ideal, por lo que nuestra ecuación de estado será
\begin{equation}
    PV = Nk_\mathrm{B}T,
\end{equation} 
donde $P$ es la presión del gas, $V$ su volumen, $N$ es el número de partículas, $k_\mathrm{B}$ la constante de Boltzman y $T$ es la temperatura. De aquí tenemos que \begin{equation}
    P = nk_\mathrm{B}T = \frac{\rho k_\mathrm{B} T}{\bar{m}}
\end{equation}
donde $n$ es la densidad numérica, $\rho$ la densidad de masa y $\bar{m}$ es la masa promedio de las partículas en el gas. En este caso estamos considerando que el gas contiene principalmente hidrógeno un $90\%$ de todo el gas y una pequeña fracción de helio, por lo que vamos a considerar una masa promedio de $\bar{m}=\frac{0.9 m_\mathrm{p}+0.1\cdot4m_\mathrm{p}}{2}\approx0.6 m_{\mathrm{p}}$ en el gas ionizado.

En este gas ideal vamos a tomar la velocidad del sonido isotérmica, la cual está dada por
\begin{equation}
    c_\mathrm{s}  = \sqrt{\frac{k_\mathrm{B} T}{\bar{m}}}
\end{equation} 
de esta manera vemos que la velocidad del sonido en el gas ionizado solo depende de su temperatura. En este tipo de gases la velocidad del sonido es típicamente del orden de \SI{e6}{cm.s^{-1}}.

Considerando que todas las fases mencionadas en la Sección \ref{Sec:fluijos fotoevaporativos} ya han sucedido, por lo que vamos a considerar que el flujo incidente de fotones ionizantes, $F_0$ está en equilibrio con las recombinaciones por unidad de área 
\begin{equation}
    F_0 = n_\mathrm{e}n_\mathrm{i}h_1\alpha_\mathrm{B}
\end{equation} 
donde $h_1$ es la anchura efectiva y  $\alpha_\mathrm{B}$ es el coeficiente de recombinación para el caso B (ver Apéndice \ref{App : tasa de fotoionizacion}). Este coeficiente de recombinación para el caso B toma en cuenta las recombinaciones a todos los niveles, excepto al nivel base, esto ya que el fotón liberado en esta recombinación puede volver a ionizar algún otro átomo que se encuentre muy cerca. 

\section{Estructura del flujo fotoevaporativo}\label{Estructura}

En este modelo  estamos considerando un frente de ionización D-crítico \citep{Shuu:1992}, esto es, dentro del glóbulo vamos a tener una expansión subsónica del gas y una expansión supersónica en el gas ionizado, por lo que tendremos un punto sónico el cual está justo detrás del frente de ionización. En este caso por simplicidad vamos a considerar el frente de ionización como una discontinuidad en el que pasamos de tener gas no ionizado a tener un gas totalmente ionizado, por lo que tomaremos que el punto sónico se da justo en $r_0$, que es el radio del glóbulo (ver Figura \ref{fig:parameters}). Así que vamos a considerar que el gas tiene un número de Mach igual a 1 en $r_0$ y este va a ir aumentando conforme atraviesa toda la zona 1 de la Figura \ref{fig:zones} ya que se va expandiendo libremente. En principio podríamos considerar que tanto el radio del glóbulo como la densidad en su superficie son parámetros libres, pero con las observaciones podemos medir el radio y la densidad la podemos calcular ya que esta debe ser consistente por haber considerado equilibrio de ionización.

Con este modelo se pretende ver hasta donde llegamos a un equilibrio de presión entre la presión por parte del flujo fotoevaporativo y la presión RAM del viento estelar. Para el caso de la presión del flujo fotoevaporativo vamos a considerar tanto la presión térmica como la hidrodinámica, por lo que la presión total en la base del glóbulo está dada por
\begin{equation}\label{eq: Presion total}
    P_\mathrm{tot}=P_\mathrm{ter}+P_\mathrm{hid}=n\bar{m}c_\mathrm{s}^2+n\bar{m}u^2=\rho c_\mathrm{s}^2(1+M^2),
\end{equation}
donde $u$ es la velocidad del gas en la parte ionizada y $M$ es el número de Mach. En este caso, como ya habíamos mencionado antes, no estamos considerando presión magnética ni de radiación.

\begin{figure}[htb]
    \centering \includegraphics[width=\textwidth]{artesanales/ImgFi01-3.pdf}
    \caption{En este diagrama esquemático $r_0$ es el radio del glóbulo neutro, mientras que $\rho_0,M_0$ y $P_0$ son los valores que tenemos en la superficie del glóbulo y varían con dirección a la estrella sus valores van cambiando hasta tener diferentes valores en $r_1$, que es el radio del centro del glóbulo hasta la base del flujo fotoevaporativo chocado, $\rho_{shell},M$ y $P_{shell}$ son los valores que tendrán en la base del flujo fotoevaporativo chocado.}
    \label{fig:parameters}
\end{figure}

Este equilibrio de presión se logrará a un radio $r_{shell}$ que es donde la presión del flujo fotoevaporativo ha disminuido una fracción $f$ de lo que era la presión inicial. Por lo que la presión cambia como 
\begin{equation}\label{eq : 1}
f=\frac{P}{P_0}=\frac{\rho c_s^2(1+M^2)}{\rho_0 c_s^2(1+1)}=\frac{\rho}{\rho_0}\frac{1+M^2}{2},
\end{equation}
donde $P_0$ es la presión en la base del glóbulo, recordemos que aquí estamos considerando el punto sónico, por lo que  $M=1$. $P$ es la presión del flujo fotoevaporativo justo antes de $r_{shell}$.
Considerando la ecuación para la conservación de masa tenemos que
\begin{equation}\label{eq : 2}
\rho r^2M	=\rho_0 r_0^2
\end{equation}
y finalmente, si consideramos la ecuación de Bernoulli isotérmica 
\begin{equation}
\frac{v^2}{2}+c_s^2\ln\rho=\text{constante}
\end{equation}
en la ecuación (\ref{eq: Presion total}) tenemos que \citep{Dyson:1968}
\begin{equation}\label{eq ; 3} \frac{r}{r_0}=M^{-1/2}e^{\frac{M^2-1}{4}}.
\end{equation}
Tomando diferentes valores para $f$, podemos resolver las ecuaciones (\ref{eq : 1}), (\ref{eq : 2}) y (\ref{eq ; 3}) de manera numérica\footnote{En nuestro caso usamos la función fsolve de la paquetería scipy.optimize, la documentación se encuentra en \url{https://docs.scipy.org/doc/scipy-1.12.0/reference/generated/scipy.optimize.fsolve.html}} para las incógnitas $\rho/\rho_0$, $M$ y $r/r_0$. Obteniendo así, que tanto la presión como la densidad decaen con el radio, mientras que el número de Mach aumenta como vemos en la Figura \ref{fig:grafica_C2}. Cabe mencionar que la Figura \ref{fig:grafica_C2} se obtiene resolviendo las ecuaciones (\ref{eq : 1}), (\ref{eq : 2}) y (\ref{eq ; 3}) en el eje de simetría. Pero esto cambiaría si consideramos cierto ángulo $\theta\neq 0 $ ya que tanto la presión como la densidad escalan con el ángulo como $\cos^{1/2}\theta$, pero $M$ no.

\begin{figure}[htb]
    \centering    \includegraphics[width=\textwidth]{Nuevas imagenes finales/C2_estructura.pdf}
    \caption{Gráfica del número de Mach $M$, densidad $\rho$ y la presión total $P$ normalizados como función de $r/r_0$. También podemos ver una caída que va como $\sim r^{-2}$ (línea negra). Podemos notar que la densidad cae más rápido que $r^{-2}$, mientras que la presión total cae más lento, en ambos casos se debe a la aceleración del flujo.
    }
    \label{fig:grafica_C2}
\end{figure}

% Assumption of isothermal equation of state
\section{Condiciones para la cáscara chocada}

\cite{Canto:1996} en su descripción para interacción de dos flujos supersónicos da la solución a distintos parámetros $\beta$, el cual es la razón de los momentos de los flujos. Esto se puede ver en la Figura \ref{fig:Canto2} en la cual notamos que cuando el momento de un flujo es muy  grande comparado con el otro la cáscara chocada se vuelve muy curva y está más cerca de la fuente cuyo flujo tiene menor momento.

En nuestro caso se puede ver que esta cáscara chocada está más cerca del glóbulo que de la estrella.

\begin{figure}[htb]
    \centering    \includegraphics[width=0.8\textwidth]{images Chapter 2/C2_Canto2.jpg}
    \caption{Formas de las distintas cáscaras chocadas a diferentes parámetros $\beta$. La línea vertical en z/D=0.5 corresponde a un parámetro $\beta=1$ y las demás curvas corresponden a valores de 0.5, 0.25, 0.125, 0.0625 y 0.03125, entre más pequeño es $\beta$ más curva se vuelve la cáscara chocada. La otra fuente se encuentra en z/D=1 \citep{Canto:1996}.}
    \label{fig:Canto2}
\end{figure}

%De la figura \ref{fig:zones} vemos que en la línea naranja tendremos la presión RAM del viento estelar y este viento tiene una desaceleración por la diferencia de presiones hasta llegar a la discontinuidad de contacto donde tendremos una presión $P_{DC}$ que si la comparamos con la presión RAM del viento tendríamos que
%\[P_{RAM}=P_{DC}\Big(1+\frac{\alpha}{M^2}\Big)\] donde $\alpha$ es del orden de 1 y como recordamos el viento estelar es supersónico, en estrellas WR puede ser de hasta de \SI{1e8}{cm.s^{-1}} por lo que podríamos tener un número de Mach del orden de 100. Por lo que podríamos considerar que la presión en la discontinuidad de contacto está dada por la presión RAM del viento estelar.

En la Figura \ref{fig:zones_presiones} vemos los diferentes tipos de presiones que hay en cada zona en esta interacción entre flujos supersónicos. En la zona de discontinuidad, la línea gris en la Figura \ref{fig:zones_presiones}, será la suma de la presión RAM del viento estelar, $P_\mathrm{RAM}$, y de la presión térmica del viento estelar. Dado que los vientos estelares en estrellas Wolft-Rayet llegan a ser del orden de \SI{1e8}{cm.s^{-1}} podríamos llegar a tener un número de Mach del orden de 100 para el viento estelar, por lo que vamos a considerar que $P_{DC}=P_\mathrm{RAM}$. Por otro lado, para la cáscara chocada, la zona 2 de la Figura \ref{fig:zones}, tenemos que esta debe estar en doble equilibrio de presión. Por un lado, la presión de la cáscara chocada, $P_{shell}$ debe ser igual a la presión del flujo fotoevaporativo justo antes del choque, es decir, $P_{shell}=P$. Por el otro lado, la presión de la cáscara chocada debe ser igual a la presión RAM del viento estelar chocado.

\begin{figure}[htb]
    \centering    \includegraphics[width=\textwidth]{Nuevas imagenes finales/zones_presiones_01.pdf}
    \caption{La base del glóbulo está dominada por la presión total, ecuación (\ref{eq: Presion total}). En la línea roja, que es la parte más interna de la cáscara chocada, tenemos que domina la presión térmica del flujo fotoevaporativo $P$. En la línea naranja domina la presión RAM del viento estelar $P_{RAM}$, la cual es la única presión externa que estamos considerando. En la zona de discontinuidad tenemos una presión $P_{DC}$, la cual es la suma de la presión RAM del viento estelar y la presión térmica del viento estelar chocado. }
    \label{fig:zones_presiones}
\end{figure}
% General solution for the internal structure of model




\chapter{Glóbulos en la nebulosa M1-67} \label{Chapter : 3}

Ahora vamos a describir cómo encontramos estos glóbulos en la nebulosa M1-67.
En la Figura \ref{fig:nudos WR124} vemos del lado derecho cómo estos glóbulos los podemos identificar de manera visual por tres componentes importantes. El glóbulo lo vemos como un círculo de color blanco,  su estela de color rosa que se encuentra justo detrás del glóbulo en dirección opuesta a la estrella, y su cáscara chocada que se ve de color gris. De esta manera, se pudieron localizar visualmente 168 glóbulos en total.

%Como ya mencionamos, estos glóbulos fueron posibles de encontrar debido a la gran variedad de filtros y la gran resolución del JWST como se puede ver la Figura \ref{fig:nudos WR124}. Se encontraron más de 100 glóbulos en casi toda la nebulosa de los cuales algunos estaban en grupos y otros solos. En las observaciones del JWST se puede llegar a observar lo que pareciera ser la cáscara chocada (Capítulo \ref{Chapter : Modelo}).

De las observaciones vemos que estos glóbulos tienen tamaños típicos de 200--300 mili segundos de arco de diámetro que son relativamente pequeños si los comparamos con la nebulosa circunestelar que es de unos $\sim\SI{60}{\arcsecond}$.

\begin{figure}[htb]
    \centering
    \includegraphics[width=\textwidth]{Nuevas imagenes finales/Globulos_JWS_1.pdf}
    \caption{En esta imagen capturada con el JWST vemos la nebulosa M1-67 y en ella la presencia de glóbulos en casi toda la nebulosa. Haciendo zoom en estos glóbulos (la emisión en color blanco y rosa), vemos como puede ser su morfología en la parte neutra debido a la emisión de PAHs. Mientras que en color gris vemos lo que parece ser su interacción con la estrella y su viento estelar.}
    \label{fig:nudos WR124}
\end{figure}

A estos glóbulos los podemos localizar por su posición angular y separación con respecto a la estrella. La posición angular es el ángulo $\phi$ que tiene con respecto a la línea roja de la Figura \ref{fig:ejemplo_PA_Sep}, y el ángulo $\phi$ es tomado en sentido contrario a las manecillas del reloj. La separación entre el glóbulo y la estrella es medido directamente de las observaciones. En la Figura \ref{fig:ejemplo_PA_Sep} vemos un ejemplo de como podemos conocer la posición angular y separación para cada glóbulo.

\begin{figure}[htb]
    \centering
    \includegraphics[width=\textwidth]{ultimos/ejemplo_phi.pdf}
    \caption{En esta imagen vemos un ejemplo de como obtuvimos la posición angular y la separación con respecto a la estrella. El círculo azul es el glóbulo que tomaremos como ejemplo, el ángulo $\phi$ con respecto a la línea roja es la posición angular y es tomada en sentido anti horario, la separación es la longitud de la línea amarilla, la cual une al glóbulo (círculo azul) con la estrella, la cual se encuentra en el centro de la imagen. Los puntos verdes marcan donde se localizan los otros glóbulos.}
    \label{fig:ejemplo_PA_Sep}
\end{figure}

En la Figura \ref{fig:dis_nudos} vemos del lado derecho la localización de los glóbulos en la nebulosa, donde los puntos verdes son los glóbulos que encontramos y la línea roja es nuestra referencia para encontrar su posición angular. En el lado izquierdo vemos como estos se distribuyen considerando su separación y posición angular. En estas distribuciones podemos encontrar ciertas simetrías, por ejemplo, en el histograma superior de la imagen izquierda vemos que muchos de estos glóbulos se concentran en dos posiciones angulares en particular, a $\sim90^\circ$ y $\sim200^\circ$, mientras que a $120^\circ$ y $300^\circ$ no tenemos casi glóbulos. En la imagen de la derecha los podemos ver como la mayoría se localiza en dos conos. Por otro lado, en el histograma del lado derecho vemos cómo en separación pareciera haber una distribución bimodal con picos a \SI{11}{\arcsecond} y \SI{17}{\arcsecond}.

\begin{figure}[htb]
    \centering  
    \begin{subfigure}[b]{0.45\linewidth}
        \includegraphics[width=\textwidth]{Nuevas imagenes finales/m1_67_glo_.png}
    \end{subfigure}
    \begin{subfigure}[b]{0.45\linewidth}
        \includegraphics[width=\textwidth]{images Chapter 2/C2_nudos_distribucion.png}
    \end{subfigure}
    \caption{En la imagen derecha vemos en puntos verde la localización de los glóbulos en la nebulosa M1-67 y la línea roja es la que usamos para conocer la posición angular de cada glóbulo. En la imagen derecha vemos como es la distribución de estos glóbulos (puntos azules) en separación y posición angular. En la parte superior de la imagen derecha vemos el histograma de la posición angular y del lado derecho el histograma de la separación de los glóbulos. Las  líneas rojas son los isocontornos de la densidad suavizada de la distribución de los puntos.} 
    \label{fig:dis_nudos}
\end{figure}

En la Figura \ref{fig:filters WR124} vemos una gran variedad de glóbulos en una zona pequeña y sus diferentes morfologías que presentan en los diferentes filtros. 

Como ya hemos mencionado anteriormente, la interacción entre el flujo fotoevaporativo de los glóbulos y el viento estelar nos produce una cáscara chocada, la cual podemos ver debido a las recombinaciones, por lo que la podemos ver como emisión en algunos filtros. Por ejemplo, en el filtro f656n del HST y en los filtros f090w y f444w del JWST podemos ver esta cáscara chocada que rodea los glóbulos. También se realizó una combinación de filtros para poder ver solo la emisión de gas ionizado (ver Apéndice \ref{AP: combos}), en esta combinación se quitó la emisión de otros mecanismos como la emisión de las estrellas o de la nebulosa. Por lo que en la imagen de gas ionizado tenemos una imagen más clara del gas ionizado que se encuentra en la cáscara chocada.

En el filtro f1130w del JWST se puede observar lo que parecen ser parte de las estelas de los glóbulos. Algunos parecen ser más pequeños que otros. Un caso muy llamativo es cuando los glóbulos están muy cerca los unos de los otros y pareciera que sus estelas se juntan, incluso en algunos casos pareciera que estas estelas están interaccionando con otros glóbulos. En este filtro también se puede ver de manera muy tenue lo que pareciera ser la interacción del flujo fotoevaporativo por parte del gas y el viento estelar.

También se hizo una combinación de filtros para poder ver solo la emisión de gas neutro (ver Apéndice \ref{AP: combos}. En este caso se puede observar mejor como es la morfología del glóbulo, por lo que aquí podemos conocer mejor sus propiedades.

Con la gran variedad de imágenes que tenemos podemos hacer muchas comparaciones y ver los glóbulos en diferentes filtros y resoluciones, lo cual nos da muchas ventajas. Por ejemplo, las observaciones del JWST tienen mayor resolución, mientras que en el caso del filtro f656n, debido a que es un filtro muy angosto, no se ve muy afectado por la emisión de las estrellas, a diferencia de las observaciones con el JWST.

\begin{figure}[htb]
    \centering
    \includegraphics[width=\textwidth]{Nuevas imagenes finales/filtros.pdf}
    \caption{La imagen del filtro f656n (esquina superior izquierda) es la emisión de H$\alpha$ por parte del HST. Las demás imágenes son utilizando los filtros del JWST. Para el caso del gas ionizado se usó la siguiente combinación de filtros f44w-0.43\,f335m-0.16(f150w-0.6\,f210m), mientras que para la emisión de gas neutro se utilizó la combinación 1.4(f150w-0.6\;f210m)+f335m-0.95\;f210m. Los círculos verdes son de los glóbulos.}
    \label{fig:filters WR124}
\end{figure}

\section{Estimando la presión RAM del viento estelar}

Para este trabajo solo vamos a considerar la presión RAM del viento estelar por parte de la estrella WR 124 como la única presión externa de confinamiento para las cáscaras chocadas de interacción de los glóbulos, la cual la vamos a tomar como \begin{equation}
 P_\mathrm{RAM}= \frac{\dot{M}v_\infty}{4\pi R^2}    
\end{equation}
donde $\dot{M}$ es la tasa de pérdida de masa de la estrella, $v_\infty$ la velocidad terminal del viento y $R$ la distancia del glóbulo a la estrella. Los dos primeros datos están dados en la tabla \ref{tab:parametros WR-124} y para $R$ vamos a considerar el rango de separación de 0.1--\SI{0.9}{pc}, esto ya que las separaciones entre los glóbulos y la estrella caen en este rango considerando que
\begin{equation}
    \Big[\frac{R}{\mathrm{UA}}\Big] = \Big[\frac{D}{\mathrm{pc}}\Big]\,\Big[\frac{\theta}{\mathrm{arcsec}}\Big],
\end{equation}
donde $\theta$ es la separación que medimos directamente de las observaciones en arcsec y $D$ es la distancia de nosotros a la estrella. En la Figura \ref{P_RAM} vemos como es esta presión RAM del viento como función de la distancia.

\begin{figure}[htb]
    \centering    \includegraphics[width=\textwidth]{Nuevas imagenes finales/PRAMcgs.pdf}
    \caption{Presión RAM del viento estelar de la estrella WR-124 como función de la distancia en parsec}
    \label{P_RAM}
\end{figure}

\chapter{Ajuste del modelo a los perfiles de brillo}\chaptermark{Ajuste del modelo}\label{Chapter : Ajuste}

En este capítulo vamos a utilizar las distintas observaciones para obtener el tamaño de los glóbulos, sus cáscaras chocadas y los anchos de sus cáscaras chocadas. También podemos obtener el brillos en el centro del glóbulo y en la cáscara chocada, con lo que podremos conocer más adelante la densidad del gas ionizado en la cáscara chocada.

La medición de los diferentes parámetros a los 168 glóbulos encontrados lo haremos de la siguiente manera. Una vez que sabemos donde se encuentran los glóbulos vamos a buscar la posición de su máximo de emisión en H$\alpha$, el cual definiremos como el centro de nuestros glóbulos\footnote{En las observaciones de gas ionizado tenemos solo una estimación para el tamaño de los glóbulos, en la Sección \ref{Sec : radio neutro} se realiza una medición mas real para el tamaño de los glóbulos usando gas neutro.}, después tomaremos como el eje de simetría considerado en el modelo (Figura \ref{ejemplo mascara}) a la línea que une el centro del glóbulo con la estrella. Como en nuestro caso nos interesa más describir el glóbulo y su cáscara chocada solo vamos a trabajar con una máscara para poder identificar bien estas dos partes. Al utilizar esta máscara nuestras mediciones no se verán afectadas por la compleja estructura de la nebulosa u otros glóbulos que se encuentren cerca. Esta máscara la vamos a considerar de la siguiente manera : Vamos a tomar solo los píxeles que estén a una distancia menor o igual a \SI{0.2}{\arcsecond} y también los que estén a una distancia máxima de \SI{1.5}{\arcsecond} y a un ángulo máximo de $60^\circ$ con respecto al eje de simetría. Con estos valores los glóbulos y sus cáscaras chocadas están dentro de la máscara en todos los casos además de que se obtuvieron buenos resultados para las distintas mediciones con estos valores.

Después graficamos el brillo de cada píxel dentro de esta máscara como función de la distancia al centro del glóbulo, como vemos en la Figura \ref{ejemplo mascara}. A estos puntos le vamos a ajustar dos gaussianas y una constante.


\begin{figure}[htb]
  \begin{subfigure}[b]{0.5\textwidth}
    \includegraphics[width=\textwidth, height=0.9\textwidth]{Nuevas imagenes finales/F_4_1_A.pdf}
    \caption{Ejemplo de máscara usada para los glóbulos.}
    \label{fig:f1}
  \end{subfigure}
  \hfill
  \begin{subfigure}[b]{0.5\textwidth}
    \includegraphics[width=\textwidth, height=0.9\textwidth]{Nuevas imagenes finales/F_4_1_B.pdf}
    \caption{Brillo de la máscara como función de la distancia, donde el 0 en el eje horizontal representa el centro del glóbulo.}
    \label{fig:f2}
  \end{subfigure}
  \caption{En la imagen de la izquierda vemos un ejemplo de la máscara que vamos a utilizar para hacer el ajuste a los perfiles de brillo (los pixeles en escala de gris), en este caso utilizando las observaciones del HST. El centro del glóbulo se localiza en el centro de la imagen, la línea roja conecta el cetro del glóbulo con la estrella (localizada a la izquierda de esta imagen) y es el eje de simetría que vamos a considerar. Vamos a considerar solo los píxeles que están en un círculo de \SI{0.1}{\arcsecond} alrededor del centro del glóbulo y los que se encuentren a una distancia menor a \SI{1.5}{\arcsecond} y que tengan un ángulo menor de $60^\circ$ con respecto al simetría que estamos considerando. En la imagen de la derecha vemos como es el brillo de estos píxeles considerados en la máscara como función de la distancia al centro del glóbulo y considerando su ángulo con respecto al eje de simetría. En este caso, los puntos verdes son los píxeles que están en la parte inferior de la imagen izquierda, por otro lado los puntos azules se encuentran en la parte superior de la imagen izquierda dentro de la máscara considerada.}
  \label{ejemplo mascara}
\end{figure}

Este ajuste es por la siguiente razón: El ancho de la primer gaussiana centrada en cero nos dirá el tamaño del glóbulo, esto suponiendo que el pico de emisión que encontramos se da justo en el centro del glóbulo. La segunda gaussiana ajustada nos indica la ubicación de la cáscara chocada y su ancho. Esperamos ver un pico de emisión en esta cáscara chocada debido a las recombinaciones que hay en esta parte, es por eso que también le ajustamos una gaussiana a la cáscara chocada. La distancia entre los picos de estas dos gaussianas no dirá el radio de la cáscara chocada. Finalmente, la constante es más un promedio del brillo del fondo en esta región. Un ejemplo más claro de estos ajustes lo podemos ver en la Figura \ref{ejemplo ajuste}.

En este ajuste le dimos menos peso a los píxeles más alejados del centro del glóbulo y también a los que tenían un gran ángulo con respecto al eje de simetría.

\begin{figure}[htbp]
    \centering
    \includegraphics[width=\textwidth]{Nuevas imagenes finales/ejemplo_ajuste_2.pdf}
    \caption{Ejemplo del ajuste de las dos guassianas y una constante a los perfiles de brillo. En este ejemplo vemos como obtuvimos los diferentes parámetros a través del ajuste. Los brillos de la parte interna como de la cáscara son el máximo de la primer y segunda gaussiana, respectivamente. La $\sigma$ de la primer gaussiana centrada en cero nos dice el tamaño de la parte interna. Para el ancho de la cáscara, lo consideramos a partir de la $\sigma$ de la segunda guassiana. Para el radio de la cáscara, lo tomamos como la distancia entre los picos de ambas gaussianas. Este es un ajuste a las observaciones del HST y podemos ver como se ve el ajuste directamente en la imagen.Este ejemplo es el mismo de la Figura \ref{ejemplo mascara}, pero para la visualización del ajuste, la imagen del glóbulo fue rotada $180^\circ$.}
    \label{ejemplo ajuste}
\end{figure}

Este ajuste se hizo a las observaciones con el HST y para el caso del JWST se hizo para el filtro f090w  y la combinación de filtros en la que solo observamos el gas ionizado. De esta manera obtuvimos varias mediciones independientes a los ajustes con distintas resoluciones. Así tenemos una mejor certeza de que realmente estamos detectando una cáscara y del tamaño de los diferentes parámetros ajustados.

\section{Medición del radio en la parte neutra} \label{Sec : radio neutro}

El ajuste anterior se hizo utilizando emisión de gas ionizado por lo que la medición para los radios de la parte neutra no estaría del todo bien. Así que para la medición de la parte neutra se utilizó la combinación de los filtros f150w, f210m y f335m del JWST para ver la emisión neutra. 

En este caso ajustamos solo una gaussiana y una constante al perfil de brillo ya que aquí no podríamos ver la cáscara chocada. El ajuste se realizó de la siguiente manera: Dado que el valor medio de los radios de los glóbulos en el anterior ajuste es de  \SI{0.14}{\arcsecond} con una variación de $\pm\SI{0.04}{\arcsecond}$ decidimos poner una máscara de \SI{0.2}{\arcsecond} alrededor del pico de emisión y unos conos con un pequeño ángulo de apertura, estos conos son perpendiculares al eje de simetría considerado en el modelo y tienen una longitud de \SI{1.5}{\arcsecond}. Con esta máscara considerada esperamos tener toda la emisión neutra en el círculo pequeño que consideramos, los conos nos servirán para calcular mejor la constante ajustada. \\

Con estos ajustes ya conocemos el radio de los glóbulos, $r_0$, el radio de la cáscara chocada, $r_\mathrm{shell}$, y el ancho de la cáscara chocada, $H_\mathrm{s}$.  
Debido a que tanto  $r_0$ como  $H_\mathrm{s}$ se midieron usando la sigma de la gaussiana ajustada, debemos corregir considerando  la anchura a altura media y después por el ancho del telescopio para poder comparar con las observaciones. De esta manera tenemos una medición más realista de estas dos cantidades.

\begin{figure}[htb]
    \centering
    \includegraphics[width=0.8\textwidth]{Nuevas imagenes finales/r_0_.pdf}
    \caption{Ejemplo de la máscara, los pixeles en escala de gris, usada para la medición del radio en la parte neutra del glóbulo. Para ver solo la emisión del gas neutro (PAHs) se usó la combinación de los filtros f150w, f210m y f335m. Estas son observaciones del JWST.}
    \label{Medicion de r_0}
\end{figure}

\section[Errores observacionales]{Estimación de incertidumbres en los parámetros observacionales}

Para el caso de los errores vamos a considerar que estos están dados por fluctuaciones no relacionadas en el brillo de la nebulosa o por efectos  sistemáticos debido a la inadecuación del modelo y no por el ruido de fotones. Así que vamos a estimar estos errores comparando las distintas mediciones hechas, suponiendo que son independientes.

\subsection{\boldmath Incertidumbres en $r_\mathrm{shell}$ y $H_\mathrm{s}$}

En el caso del radio y ancho de la cáscara, $r_\mathrm{shell}$ y $H_\mathrm{s}$, respectivamente, podemos notar que estos siguen una buena tendencia (ver Figura \ref{fgi: Radios de la cascara}) si comparamos las mediciones realizadas con el HST y el JWST. Para estas mediciones vamos a considerar el error estándar, el cual está dado por
\begin{equation}
\epsilon_x = \frac{\sigma(x_\mathrm{HST}-x_\mathrm{JWST})}{\sqrt{2}}     
\end{equation}
donde $x_\mathrm{HST}$ son las mediciones hechas con el HST y $x_\mathrm{JWST}$ las mediciones hechas con el JWST en el gas ionizado, mientras que $\sigma$ es la desviación estándar de la resta de estas dos cantidades anteriores. 

\begin{figure}[htb]
    \centering
    \includegraphics[width=\textwidth]{Nuevas imagenes finales/R_shell_r.pdf}
    \caption{Comparación de los radios de la cáscara (puntos de colores) obtenidos a partir de los ajustes a ambos telescopios. Las mediciones con el HST es en el filtro f656n y en el JWST son mediciones hechas en el gas ionizado. La línea roja representa la identidad, por lo que se puede ver que ambas mediciones son muy similares}
    \label{fgi: Radios de la cascara}
\end{figure}

\subsection{\boldmath Incertidumbre en $r_0$}

Para el caso de la parte interna tenemos una medición para el radio en la parte neutra con un combinación de filtros del JWST y una aproximación con el HST, ya que en este último vemos solo el gas ionizado y el glóbulo contiene principalmente gas neutro.. Aunque no hay una buena correlación entre estas dos mediciones, como se puede ver en la Figura \ref{fig:Rcore dis}, su valor promedio y desviación estándar se parecen mucho. Por lo que vamos a considerar un radio constante que será igual para todos los glóbulos. Este valor será \SI{0.135}{\arcsecond} el cual es el promedio de los valores medios de las dos mediciones y como error vamos a considerar la desviación estándar de ambas mediciones ya que son iguales, $\pm\SI{.03}{\arcsecond}$ (ver Figura \ref{fig:Rcore dis}).

\begin{figure}[htb]
    \centering
    \includegraphics[width=0.8\textwidth]{ultimos/Rcore(1).pdf}
    \caption{En esta imagen vemos la comparación de los radios medidos para la parte neutra con ambos instrumentos. El histograma superior es para las mediciones del radio del glóbulo utilizando el JWST, estas mediciones tienen un valor promedio de \SI{0.15}{\arcsecond} con una desviación estándar de \SI{0.03}{\arcsecond}. El histograma del lado derecho es para las mediciones del radio del glóbulo utilizando el HST, teniendo un valor promedio de \SI{0.12}{\arcsecond} con una desviación estándar de \SI{0.03}{\arcsecond}. }
    \label{fig:Rcore dis}
\end{figure}

\subsection{\boldmath Incertidumbres en $B_\mathrm{s}$ y $B_0$}

Para los errores de los brillos superficiales en la cáscara chocada y en el glóbulo, $B_\mathrm{s}$ y $B_0$, respectivamente, vamos a considerar como error la siguiente desviación estándar
\begin{equation} \epsilon_{B}=\sqrt{\mathrm{Var}\Big(\overline{(y-\overline{y})^2},w_1*w_2\Big)} 
\end{equation} 
donde $y$ son los valores observados directamente del filtro f656n del HST menos el ajuste realizado a los perfiles de brillo, $w_1$ es el peso considerado en el modelo  y $w_2$ es la gaussiana ajustada a la parte correspondiente, es decir, para la parte interna $w_2$ es la primer gaussiana ajustada, la cual tiene su pico en \SI{0}{\arcsecond}, y para la cáscara $w_2$ es la segunda gaussiana ajustada (ver Figura \ref{ejemplo ajuste}).\\

Con la estimación de todas estas incertidumbres, usamos la paquetería \verb|uncertainties|\footnote{La documentación de la paqueteria se encuentra en \url{https://pythonhosted.org/uncertainties/}} para tener nuestras barras de error. Esta paquetería calcula las incertidumbres usando la teoría de propagación de errores lineales tomando en cuenta si los datos son totalmente independientes o están correlacionados.

\section{\boldmath Estimación de la densidad del gas ionizado a partir del brillo superficial de H$\alpha$}\label{Sec : estimacion de densidad}

Para estimar la densidad del gas ionizado, usamos primero la definición de Medida de Emisión (EM por sus siglas en inglés)
\begin{equation}
\mathrm{EM}=\int_z n_\mathrm{i} n_\mathrm{e}dz    
\end{equation}
donde $n_\mathrm{i}$ es la densidad de iones, $n_\mathrm{e}$ la densidad electrónica y en este caso estaremos integrando sobre nuestra línea de visión (dz). Si consideramos un gas totalmente ionizado, donde los electrones provengan solo del H, entonces tenemos que $n_\mathrm{e}=n_\mathrm{i}=n$\footnote{En principio también tenemos contribución de He, pero como esta contribución es pequeña debido a que el espectro ionizante es suave, lo vamos a considerar como neutro.}. Entonces, tenemos que $\mathrm{EM}=n^2l$, donde $n$ es la densidad promedio RMS y $l$ es la profundidad del gas denso en nuestra línea de visión.
 
\begin{figure}[htb]
    \centering    \includegraphics[width=\textwidth]{artesanales/ImgFi01-4.pdf}
    \caption{Diagrama esquemático que muestra una línea de visión, con un máximo en la línea punteada $l$, y si consideramos una simetría esférica vamos a tener esta configuración donde $h$ será el ancho de la cáscara chocada, $r$ el radio del centro del glóbulo hasta donde inicia el flujo fotoevaporativo chocado y $h<r$.}
    \label{fig:EM}
\end{figure}

Si suponemos una cáscara esférica de radio $r$, una anchura $h$ con  $h\ll r$ y una densidad constante, entonces tenemos que
\begin{equation}
\mathrm{EM}=2\sqrt{2rh}n^2.\end{equation}

Esto ya que de la Figura \ref{fig:EM} tenemos que por geometría $r^2+\Big(\frac{l}{2}\Big)^2=(r+h)^2=r^2+2rh+h^2\approx r^2+2rh$, y por tanto, $ l=2\sqrt{2rh}$. De esta manera, usando la EM tenemos que \begin{equation}n=\sqrt{\frac{\mathrm{EM}}{l}}=\sqrt{\frac{\mathrm{EM}}{2\sqrt{2rh}}}.
\end{equation} 

En este trabajo vamos a aplicarlo a las cáscaras detectadas donde tomaremos a $r=r_\mathrm{s}$ y $h=H_\mathrm{s}$.

\subsection{Uso de la EM a partir de las observaciones} \label{Subsec : EM}

En nuestras observaciones del HST el brillo superficial está dado en unidades de \unit{counts.s^{-1}}, por lo que vamos a usar el factor de conversión 0.0137 para tenerlo en unidades de \unit{erg.s^{-1}.cm^{-2}.sr^{-1}}. Si dividimos entre la energía de H$\alpha$ tenemos que

\begin{equation}
    B\frac{0.0137}{(h\nu)_{\mathrm{H}\alpha}}=\int \frac{f_{\mathrm{H}\alpha}\alpha_\mathrm{B} n_\mathrm{e} n_\mathrm{p}}{4\pi} dz
\end{equation}
donde $B$ es el brillo superficial que tenemos directamente de nuestras observaciones, $(h\nu)_{\mathrm{H}\alpha}$ la energía de H$\alpha$, $f_{\mathrm{H}\alpha}$ es la fracción de todas las recombinaciones a los niveles $n\geq 2$, $\alpha_\mathrm{B}$ es el coeficiente de recombinación para el caso B, $n_\mathrm{e}$ la densidad electrónica y $n_\mathrm{p}$ la densidad de iones. La integral de la derecha es a lo largo de nuestra línea de visión.

Si consideramos que tanto la fracción de todas las recombinaciones como el coeficiente de recombinación son constantes, entonces
\begin{equation}
    B\frac{0.0137}{(h\nu)_{\mathrm{H}\alpha}}=\frac{f_{\mathrm{H}\alpha}\alpha_\mathrm{B}}{4\pi}\int n_\mathrm{e} n_\mathrm{p} dz,
\end{equation}
donde la integral es la EM. Por lo que, si consideramos que la fracción de recombinaciones al nivel $n\le 2$ es 0.5 y que $\alpha_\mathrm{B}=\SI{2.3e-13}{cm^{3}.s^{-1}}$ entonces  $f_{\mathrm{H}\alpha}\alpha_\mathrm{B}=\SI{1.17e-13}{cm^3.s^{-1}}$, entonces podemos conocer la EM directamente de las observaciones como
\begin{equation}
    EM = B\frac{0.0137}{(h\nu)_{\mathrm{H}\alpha}}\frac{4\pi}{\SI{1.17e-13}{cm^3.s^{-1}}},
\end{equation}
y como ahora todo está en unidades de cgs, la EM tendrá unidades de \unit{cm^{-5}}.\\


Es importante mencionar que  para conocer la densidad usando la EM y $l$, estamos considerando que $l$ es perpendicular a el eje de simetría considerado en el modelo. 

En la Figura \ref{fig:dens_angl} estamos considerando que la densidad máxima $n_1$ está en el eje de simetría y decae  con el ángulo como $\cos^{1/2} i$ \citep{Tarango:2018}.

\begin{figure}[htb]
    \centering    \includegraphics[width=\textwidth]{Nuevas imagenes finales/densi_angle.pdf}
    \caption{En el eje de simetría tenemos una densidad máxima en la cáscara chocada $n_1$ (color más oscuro) y esta densidad va cayendo con el ángulo que tiene con respecto al eje de simetría (color más claro).}
    \label{fig:dens_angl}
\end{figure}


\section{Buenos ajustes}\label{Good results}

El ajuste descrito al inicio de este capítulo resultó ser muy bueno para los glóbulos en los que podíamos ver una cáscara. Con este ajuste ahora conocemos $r_0$, $r_\mathrm{shell}$ y $H_\mathrm{s}$. Además conocemos el brillo tanto en la parte interna como en la cáscara chocada por lo que podemos obtener más información. En la Sección \ref{Sec : estimacion de densidad} vimos como obtener la densidad en la cáscara chocada a partir de su brillo, y con esto ahora podemos conocer la presión de la cáscara, la cual está dada  como 
\begin{equation}
    P_\mathrm{shell}=\rho c_\mathrm{s}^2.
\end{equation}
 Esta presión la podremos comparar con la presión RAM del viento estelar. De igual manera podremos tener un estimado de la presión interna del glóbulo y comparar directamente con el modelo.

Estos glóbulos a los cuales les encontramos un buen ajuste están en un rango de separación a la estrella muy amplio, por lo que podemos conocer mejor la nebulosa y los glóbulos en general también. En la Figura \ref{Goog G} podemos apreciar algunos ejemplos de los buenos ajustes donde se puede identificar el glóbulo y su cáscara chocada.

\begin{figure}[htb]
    \centering
    \includegraphics[width=\textwidth]{Nuevas imagenes finales/C4_Buenos_ajustes_.pdf}
    \caption{Ejemplos de buenos ajustes. En los paneles izquierdos vemos los ajustes a los perfiles de brillo en el lado izquierdo y como se ven con los datos del HST en la derecha, mientras que los paneles del lado derecho son utilizando los datos del JWST, del lado derecho están  los ajuste a los perfiles de brillo para el gas ionizado (Figura \ref{fig:filters WR124}), pero la visualización de estos ajustes los vemos en el filtro f090w ya que morfológicamente se parece mucho a las imágenes del HST.}
    \label{Goog G}
\end{figure}

\section{Ajustes recuperados}

Gracias a que se hizo el ajuste a los perfiles de brillo en dos telescopios distintos y con distintas resoluciones se pudieron obtener mejores resultados. Esto debido a que en algunos glóbulos a pesar de poder observar sus cáscaras chocadas visualmente nuestro algoritmo no las lograba detectar. 

En la Figura \ref{Recuperados Globulos} vemos como en  ambos casos se puede ver una posible cáscara pero al momento de hacer el ajuste no detecta debido a la dispersión que tenemos en ciertas direcciones o por la falta de puntos, como es el caso de los datos del HST. Otro problema del porqué no se detectaba bien la cáscara, era debido a que esta solía tener un brillo muy bajo.

De esta manera se pudo tener una mayor muestra de la que ya se había obtenido inicialmente. Por lo que tenemos una muestra final de 30 glóbulos en los que conocemos el tamaño de los glóbulos, sus cáscaras chocadas, así como  sus anchos y el brillo superficial en el centro del glóbulo y en la cáscara chocada\footnote{Todos estos ajustes se pueden ver en el Apéndice \ref{App : ajustes}}. Con estas mediciones podemos comparar la presión de las cáscaras chocadas con la presión RAM del viento estelar, por lo que en el Capítulo \ref{Ch : balance de presiones} podremos comparar nuestro modelo con las observaciones.

\begin{figure}[htb]
    \centering
    \includegraphics[width=\textwidth]{Nuevas imagenes finales/C4_Recuperados_ajustes.pdf}
    \caption{Ejemplos de ajustes recuperados. En el glóbulo de arriba vemos que la cáscara que se alcanza a ver tenuemente en ambas imágenes, pero solo se recupera en el ajuste a los datos del JWST. En el glóbulo de abajo vemos a simple vista una cáscara pero debido a la dispersión de los puntos en una dirección esta no se recupera bien en el caso del ajuste a los datos del JWST}
    \label{Recuperados Globulos}
\end{figure}

\section{Glóbulos descartados}\label{Bad globules}

A pesar de tener una buena cantidad de glóbulos para aplicar a este modelo no se usaron todos por diferentes razones. 
Algunos de ellos tenían un mal ajuste debido a la gran estructura de la nebulosa, había estrellas de fondo o se veían afectados por las espigas de difracción de estrellas brillantes.  Debido a esto en algunos casos no se alcanzaba a detectar bien la parte neutra o la cáscara chocada y en algunos casos la detección de estas regiones estaban mal en cuanto a sus tamaños. En la Figura \ref{Bad Globules} vemos algunos ejemplos de los glóbulos a los que no se les pudo detectar bien su cáscara chocada.

\begin{figure}[htb]
    \centering
    \includegraphics[width=\textwidth]{Nuevas imagenes finales/Malos_ajustes_final.pdf}
    \caption{Ejemplos de glóbulos descartados. Del lado izquierdo es la imagen con el HST y del lado derecho con el JWST en el filtro f090w. El primer glóbulo es descartado por que se ve afectado por la difracción del telescopio y por la gran estructura de la nebulosa, esto se puede deber a lo cerca que está de la estrella central. En los otros ejemplos vemos como la presencia de fuentes cercanas afecta nuestras estimaciones de posibles cáscaras chocadas en las observaciones.}
    \label{Bad Globules}
\end{figure}

En la Tabla \ref{tab:mean} tenemos los valores medios de las diferentes mediciones hechas, tomando a la dispersión de los datos entre los diferentes glóbulos como su incertidumbre. Vemos que la separación entre los glóbulos y la estrella tiene un valor medio de $14.96\pm\SI{7.1}{\arcsecond}$. Por otro lado, podemos ver que tanto el radio del glóbulo como el ancho de la cáscara es menor que el radio de la cáscara. La densidad es típicamente del orden de \SI{e3}{cm^{-3}}, mientras que las presiones de las cáscaras son del orden de \SI{e-9}{dyn.cm^{-2}}. Finalmente $n_{i,0}$ es la densidad del gas ionizado en el frente de ionización de los glóbulos,  esto considerando un equilibrio de ionización.

\begin{table}[htb]
    \centering
    \begin{tabular}{c c}
        \toprule
        \multicolumn{2}{c}{Resultados de los buenos ajustes} \\ \midrule
         Separación & 14.96$\pm$\SI{7.1}{\arcsecond}\\
         Radio del glóbulo $r_0$ & 0.135$\pm$\SI{0.03}{\arcsecond} \\
         Radio de la cáscara $r_\mathrm{shell}$& 0.62$\pm$\SI{0.13}{\arcsecond}\\
         Ancho de la cáscara $H_\mathrm{s}$ & 0.24$\pm$\SI{0.1}{\arcsecond}\\
         $n_\mathrm{shell}$ & 1.37$\pm$\SI{.05e3}{cm^{-3}}\\
         $n_{i,0}$ & 5.01$\pm$\SI{.55e3}{cm^{-3}}\\
         $P_\mathrm{shell}$ & 1.89$\pm$\SI{.06e-9}{dyn.cm^{-2}}  \\
         $r_\mathrm{shell}/r_0$ & 4.4$\pm$1 \\\bottomrule
    \end{tabular}
    \caption{Valores medios de los resultados obtenidos a partir de los ajustes. En este caso tomamos las incertidumbres como la dispersión entre los diferentes glóbulos.}
    \label{tab:mean}
\end{table}

\chapter{Balance de presiones en las cáscaras}\label{Ch : balance de presiones}

En este capítulo vamos a utilizar las mediciones obtenidas a partir de las diferentes observaciones a los glóbulos encontrados en la nebulosa M1-67. Así, podemos investigar acerca del equilibrio de presiones que existe entre el flujo fotoevaporativo por parte de los glóbulos y la presión del viento estelar para finalmente compararlo directamente con el modelo hidrodinámico estacionario que proponemos. Si las observaciones concuerdan con nuestro modelo propuesto, entonces podremos conocer mejor la distribución espacial de los glóbulos en la nebulosa M1-67.

Se encontraron 168 nudos glóbulos en la nebulosa, los cuales están distribuidos a una distancia de la estrella central de entre 3--\SI{35}{\arcsecond}.  Estos glóbulos se pueden observar ya sea en grupos o solos, como se puede ver en la Figura \ref{fig:dis_nudos}. 

A pesar de no tener un buen ajuste a todos los glóbulos, tenemos una muestra muy buena ya que estos están distribuidos en un amplio rango de distancia a la estrella y también una gran variedad en la medición de algunos parámetros. 

%\section{Estimación de la presión en la zona neutra}

%Usando el modelo hidrodinámico que proponemos podemos calcular cuál es la presión del flujo fotoevaporativo en la base del glóbulo si conocemos el radio del glóbulo y de la cáscara chocada.  
%Para calcularlo tenemos que
%\[\frac{P_{shell}}{P_0}=\frac{P_{shell}}{P_{RAM}}\frac{P_{RAM}}{P_0} \Rightarrow P_0=\frac{P_{shell}}{P_{RAM}}\frac{P_0}{P_{shell}}P_{RAM}\]
%donde $P_{shell}$ es la presión de la cáscara chocada, $P_0$ es la presión del flujo fotoevaporativo en la base del glóbulo y $P_{RAM}$ es la presión del viento estelar. En la ecuación de la derecha, el primer término $\frac{P_{shell}}{P_{RAM}}$ lo podemos calcular usando las observaciones como ya se había mencionado antes, el segundo término $\frac{P_0}{P_{shell}}$ lo podemos obtener del modelo conociendo la razón entre los radios que encontramos y finalmente la $P_{RAM}$ puede ser calculada de las observaciones conociendo la distancia del glóbulo a la estrella.

\section{Balance interno de presión}\label{Sec : comparacion-modelo}

Ahora vamos a comparar directamente con el modelo. Para esto vamos a comparar como se ve la razón de las presiones que hay en la cáscara chocada del glóbulo y en la superficie del glóbulo contra la razón de los radios encontrados. 

Como estamos considerando un modelo estacionario, tenemos que la presión de la cáscara chocada, en la cual estamos considerando solo presión térmica, debe ser igual a la presión total que hay justo antes de $r_\mathrm{shell}$. En el Capítulo \ref{Chapter : Modelo} hablamos de como el modelo propuesto predice $f$ como función de $r/r_0$, que nos dice como debe ser la presión del flujo fotoevaporativo justo antes del choque interior normalizado con la presión que hay en la base del glóbulo, esto en función del radio.  

En el caso de la razón entre los radios es muy sencillo, pues hemos obtenido estos parámetros de una manera muy directa para cada glóbulo.
Dado que también hemos podido medir los brillos y tamaños, tanto del glóbulo como de la cáscara chocada, podemos calcular sus respectivas densidades.

Por lo que para la razón entre las presiones que hay en la base del glóbulo y en la cáscara chocada del glóbulo, primero, recordando de la sección \ref{Subsec : EM} tenemos que $\mathrm{B}\propto \mathrm{EM}=n^2l$ por lo que
\begin{equation}
\frac{\mathrm{B_\mathrm{s}}}{\mathrm{B_0}}=\frac{n_\mathrm{s}^2l_\mathrm{s}}{n_0^2l_0},
\end{equation}
donde $\mathrm{B_\mathrm{s}}, \; n_\mathrm{s}$ y $l_\mathrm{s}$ son el brillo, la densidad y la longitud donde se tiene el máximo en la EM (Sección \ref{Sec : estimacion de densidad}) y  $\mathrm{B_0}, \; n_0$ y $l_0$ son el brillo, la densidad y la longitud donde tenemos el máximo de la EM en la parte interna. Por otra parte, para las observaciones tenemos que $f_\mathrm{obs} = \frac{P_\mathrm{shell}}{P_0}=\frac{\rho_\mathrm{shell}}{2\rho_0} \Rightarrow \frac{\rho_\mathrm{shell}}{\rho_0}=2f_\mathrm{obs}$, por lo que
\begin{equation}
\frac{\mathrm{B_\mathrm{s}}}{\mathrm{B_0}}=4f_\mathrm{obs}^2\frac{l_\mathrm{s}}{l_0}\Rightarrow f_\mathrm{obs}= \frac{1}{2}\Big(\frac{\mathrm{B_\mathrm{s}}/\mathrm{B_0}}{l_\mathrm{s}/l_0}\Big)^{1/2}.
\end{equation}
De esta manera podemos comparar con el modelo directamente. En el apéndice \ref{App:brillos} damos a detalle las correcciones para los brillos.

Debido a que el tamaño del glóbulo es muy pequeño, vamos a considerar que $r_0\approx l_0$\footnote{De la sección \ref{Subsec : EM} tenemos que $l_0=2\sqrt{2hr}=2\sqrt{2h/r_0}r_0=0.98 r_0\approx r_0$, esto considerando que $h/r_0\approx0.12$ (ver Apéndice \ref{App : tasa de fotoionizacion}).}. Así la comparación con el modelo saldrá directamente de las observaciones. 

\begin{figure}[htb]
    \centering
    \includegraphics[width=\textwidth]{ultimos/R_2(1).pdf}
    \caption{$f_\mathrm{model}$ es la línea teórica que obtuvimos a partir del modelo propuesto, mientras que $f_\mathrm{obs}$ es la razón entre la presión de la cáscara chocada  y la presión en la superficie del glóbulo a partir de las observaciones.}
    \label{Resultados_modelo}
\end{figure}

En la Figura \ref{Resultados_modelo} podemos observar como los resultados a partir de las observaciones se parecen bastante a las predicciones que se obtienen del modelo propuesto. En el caso de las observaciones vemos que la mayoría de los datos caen en un  rango de $r_\mathrm{shell}/r_0=$3--6, el cual es la razón del radio de la cáscara entre el radio del glóbulo, mientras que para la razón entre las presiones caen en un rango de $f_\mathrm{obs}=$0.05--0.15. También podemos notar, para los datos observados, una anti correlación con bastante dispersión.

Dado que el modelo no tiene parámetros libres y que los resultados obtenidos a partir de las observaciones tienen una gran concordancia con las predicciones del modelo, podemos decir que este modelo funciona para intentar explicar la interacción que hay entre los flujos supersónicos. En la Figura \ref{Resultados_modelo} vemos como las mediciones de nuestras observaciones, $f_\mathrm{obs}$ se ajustan a las que predice nuestro modelo $f_\mathrm{model}=\frac{P}{P_0}$ (Ecuación \ref{eq : 1}) si consideramos las barras de error.

\section{Balance externo de presión}\label{Sec:proyeccion}

En la Figura \ref{graf_presion}vemos como la presión de las cáscaras chocada de los glóbulos como función de la separación que hay entre la estrella y cada glóbulo. Podemos apreciar como la presión de las cáscaras de los glóbulos, puntos morados, cae por debajo de la presión RAM del viento estelar, la línea roja. En nuestro modelo consideramos que la presión de la cáscara debe estar en equilibrio con la presión RAM del viento estelar, por lo que podemos pensar que la separación que estamos viendo, en realidad, no es más que una separación proyectada. 

\begin{figure}[htb]
    \centering
    \includegraphics[width=\textwidth]{Nuevas imagenes finales/R_inf.pdf}
    \caption{En esta gráfica vemos como las presiones de la cáscara de los glóbulos (círculos morados) estarían por debajo de la presión RAM (línea roja) si la separación proyectada fuera igual a la separación real.}
    \label{graf_presion}
\end{figure}

Si suponemos que el glóbulo está proyectado por un ángulo $i$ como vemos en la Figura \ref{Ang proyeccion}, entonces tenemos que 
\begin{equation}
R\cos i=R_\mathrm{p}
\end{equation}\label{eq:sep_real}
\noindent donde $R$ es la separación real del glóbulo a la estrella, $i$ el ángulo de inclinación y $R_\mathrm{p}$ la distancia proyecta, que es la que observamos. Debemos tomar en cuenta que la presión de la cáscara es igual a la presión RAM del viento estelar solamente en el eje de simetría, y como vimos en la sección \ref{Subsec : EM}, fuera del eje de simetría, la presión de la cáscara disminuye por un factor de $\cos^{1/2}i$ \citep{Tarango:2018}. Por lo que considerando este ángulo de inclinación la presión de la cáscara debería estar dada como 
\begin{equation}
P_\mathrm{shell}(i)=\frac{\dot{M}v_\infty}{4\pi R_\mathrm{p}^2}\cos^{5/2}i.
\end{equation}\label{eq:cos 5_2}

\begin{figure}[htb]
    \centering
    \includegraphics[width=\textwidth]{artesanales/ImgFi01-6.pdf}
    \caption{Representación de como algunos glóbulos se ven afectados por un ángulo de inclinación con respecto a nuestra línea de visión. Para el glóbulo de abajo tenemos que la distancia real a la estrella es $R$, mientras que nosotros vemos $R_\mathrm{p}$ que es la distancia proyectada a un ángulo $i$. Por otro lado el glóbulo de arriba no se ve afectado por alguna proyección.}
    \label{Ang proyeccion}
\end{figure}

Con esta ecuación, podemos conocer el coseno del ángulo, ya que conocemos tanto la presión de la cáscara como su separación proyectada. Usando este coseno del ángulo en la ecuación \ref{eq:sep_real} podemos conocer cuál es la separación real entre la estrella y el glóbulo, por lo que podemos conocer cuál es la distribución real de los glóbulos en cuanto a separación.  

En la Figura \ref{graf_presion_ang} vemos como la presión de los glóbulos se puede ajustar a la presión RAM si consideramos cierto ángulo de inclinación $i$. De aquí podemos obtener el ángulo de inclinación de cada glóbulo.% Este ángulo de inclinación lo podemos ver en la Figura \ref{fig:ang_Will}, en el podemos ver claramente una anti correlación entre el ángulo de inclinación y la separación proyectada. Los glóbulos más cercanos a la estrella tienen un mayor ángulo de inclinación. La medición de este ángulo de inclinación es considerando un equilibrio de presiones entre la presión de la cáscara y la presión RAM del viento estelar, los puntos rojos en la Figura \ref{fig:ang_Will}. Estos ángulos se obtuvieron para cada glóbulo individualmente, pero concuerdan con la medición realizada a grupos de glóbulo, datos azules. En el siguiente capítulo hablaremos más acerca de esta Figura.

\begin{figure}[htb]
    \centering
    \includegraphics[width=\textwidth]{ultimos/R_1_.pdf}
    \caption{Ejemplos de como podemos conocer el ángulo de inclinación de cada glóbulo si ajustamos una línea continua a la presión de la cáscara chocada de cada glóbulo (círculos morados) de tal manera que esta presión sea similar a la presión RAM del viento estelar (líneas continuas).}
    \label{graf_presion_ang}
\end{figure}

En la Figura \ref{fig:ncos_2} vemos en líneas continuas como sería el ángulo de inclinación si consideramos que la densidad de la cáscara no decae con el ángulo, es decir, $P_\mathrm{shell}(i)=\frac{\dot{M}v_\infty}{4\pi R_\mathrm{p}^2}\cos^2i$. Y vemos que a ángulos pequeños las líneas continuas se pegan mucho a las líneas punteadas (cuando la densidad decae con el ángulo), por lo que la diferencia de ángulos no es muy grande, sin embargo, a grandes ángulos esto comienza a tomar relevancia. De igual manera, podemos apreciar que a grandes distancias las líneas continuas y discontinuas se pegan mucho, esto se puede deber a que en los glóbulos mas lejanos a la estrella su separación real y proyectada sean muy similares. En el siguiente capítulo vamos a discutirlo más a detalle.

\begin{figure}[htb]
    \centering
    \includegraphics[width=\textwidth]{ultimos/R_1_cos2(1).pdf}
    \caption{La línea continua roja es la presión RAM del viento estelar como función de la distancia real, las demás líneas continuas son las presiones de las cáscaras considerando cierto ángulo de inclinación $i$ y que la densidad no decae con el ángulo. Las líneas discontinuas son las presiones de las cáscaras considerando cierto ángulo $i$ y que la densidad también decae con el ángulo como $\cos^{-1/2}(i)$.}
    \label{fig:ncos_2}
\end{figure}

En la Tabla \ref{tab:mean_i} vemos los valores medios considerando este ángulo de inclinación que hemos encontrado, considerando que la densidad decae con el ángulo. Como el valor medio del $\cos(i)$ es 0.65, podemos notar que en realidad sí estamos viendo la separación proyectada de los glóbulos, de hecho de casi todos. Por lo que en realidad están más lejanos de lo que apreciamos en las imágenes. En la Figura \ref{fig:hist_sep_ryp} las barras naranjas son el histograma de la separación proyectada entre los glóbulos y la estrella, y podemos ver que muchos de estos glóbulos parecían estar cerca de la estrella. Las barras de colo azul es la separación real, y vemos como todos están a más de \SI{10}{\arcsecond} de distancia, así como también tenemos una concentración de glóbulos a $\sim$\SI{17}{\arcsecond}.

\begin{table}[htb]
    \centering
    \begin{tabular}{c c}
        \toprule
        \multicolumn{2}{c}{Resultados con ángulo de inclinación} \\ \midrule
         cos(i) & 0.65$\pm$0.01 \\
         $R$ & 22.08$\pm$\SI{0.34}{\arcsecond}\\
         $n_\mathrm{shell}(i)$ & 1.81$\pm$\SI{0.05}{cm^{-3}}\\
         $P_\mathrm{shell}(i)$ & 1.89$\pm$\SI{0.06e-9}{dyn.cm^{-2}} \\
         \bottomrule
    \end{tabular}
    \caption{Valores típicos de los resultados obtenidos considerando el ángulo de inclinación $i$.}
    \label{tab:mean_i}
\end{table}

\begin{figure}[htb]
    \centering
    \includegraphics[width=\textwidth]{ultimos/Hist_seprarcion(1).pdf}
    \caption{Histogramas de la separación real (color azul) y la separación proyectada (color naranja) medida en arcsec.}
    \label{fig:hist_sep_ryp}
\end{figure}


\chapter{Discusión}\label{Chp:conclusiones}

Ahora vamos a discutir algunos problemas que que se presentaron al momento de hacer los ajustes a los perfiles de brillos, así como también algunas propiedades de los glóbulos, entre otras cosas. Debido a que son varias cosas a discutir, vamos a dividir la discusión en secciones.

Considerando que \cite{Zavala:2022} estima que la nebulosa circunestelar se formó hace unos \SI{11.8}{kyr}, podríamos suponer que las fases mencionadas en la sección \ref{Sec:fluijos fotoevaporativos} ya han pasado y esto es importante porque si estuviéramos en otra fase como en la de implosión, los radios cambiarían además que las presiones que consideramos podrían aún no estar en equilibrio. En el caso de no estar aún en el equilibrio de presiones indicaría que esta interacción entre el flujo fotoevaporativo y el viento estelar se formó recientemente.\\

\section{Identificación de los glóbulos y sus cáscaras}\label{dis:casaras}

Primero vamos a discutir los problemas que se presentaron al momento de identificar tanto los glóbulos como sus cáscaras chocadas.

Si bien la presencia de estos glóbulos y su interacción con el viento estelar se puede apreciar en las imágenes del JWST, no en todos los casos es claro ya que hay algunos glóbulos que se encuentran en grupo tal como se ve en la Figura \ref{globule_group}. En estos casos la detección de la cáscara chocada es un poco difícil por las siguientes razones. En la imagen de arriba de la Figura \ref{globule_group} vemos un claro ejemplo de que cuando dos glóbulos estén muy cercanos se pueden confundir con que sea un solo glóbulo, como es el caso de los que están marcados con círculos azules. De esta manera nos podríamos confundir en el tamaño de la parte neutra del glóbulo y sobrestimarla. Detrás de estos glóbulos se encuentran otros dos glóbulos cercanos (marcados con círculos negros), debido a la proyección en el cielo, estos parecen estar en la estela de los glóbulos marcados con círculos azules, por lo que en principio no podríamos detectar bien una cáscara chocada. Por otro lado tenemos el glóbulo marcado con un círculo rojo. Este glóbulo es más pequeño que los otros que están en el grupo y aparentemente su cáscara chocada está muy cerca del glóbulo pero en realidad esta cáscara parece ser del par de glóbulos marcados con círculos azules. Algo similar pasa con el glóbulo marcado con el círculo verde, el cual aparentemente no tiene una cáscara pero pareciera estar en la cáscara chocada de algún otro glóbulo.

En la imagen inferior de la Figura \ref{globule_group} de igual manera vemos un grupo de glóbulos, pero en esta ocasión están lo suficientemente lejos como para no confundir sus respectivas cáscaras. El problema aquí es que ahora las cáscaras están cerca la una de la otra, por lo que la emisión de una afecta a otras. En este ejemplo en particular, vemos que las cáscaras más grandes contaminan a las más pequeñas en cuanto a su emisión.\\

\begin{figure}[htb]
    \centering
    \includegraphics[width=\textwidth]{discusion/groups_globules.pdf}
    \caption{Estos son algunos ejemplos de los grupos de glóbulos que se encontraron en la nebulosa. En la imagen de arriba tenemos varios problemas, entre ellos identificar si es uno o más glóbulos para el caso de los que están muy cerca y saber de qué glóbulo viene cada cáscara chocada, en caso de detectarla. En la imagen de abajo vemos como la cercanía entre glóbulos hace que la emisión de una cáscara chocada se vea afectada por otra cercana.}
    \label{globule_group}
\end{figure}

\section{Masa de los glóbulos}\label{app:masa_glo}

Para la densidad del gas ionizado tenemos que \cite{Grosdidier:1998} en su análisis a la nebulosa M1-67, encuentra unos puntos brillantes de 0.2--\SI{0.3}{\arcsecond}, a los cuales les estima una densidad de gas ionizado de 4800--\SI{12 000}{cm^{-3}}. Estos puntos brillantes parecen ser en su mayoría nuestros glóbulos encontrados. Su estimación de la densidad del gas ionizado de los nudos es congruente con nuestras estimaciones, donde nosotros encontramos una densidad típica de \SI{5e3}{cm^{-3}} (Tabla \ref{tab:mean}).  

Para estimar la densidad en la parte neutra vamos a suponer un equilibrio de presiones entre la parte neutra y la ionizada, por lo que tendríamos que
\begin{equation}
    2 n_{i,0}c_{s,i}^2 = n_n(c_{s,n}^2+\frac{1}{2}v_A^2),
\end{equation}
donde $n_{i,0}, c_{s,i}$ es la densidad y la velocidad, respectivamente, en la parte ionizada, $n_n, c_{s,n}$ la densidad y la velocidad del sonido, respectivamente, en la parte neutra y $v_A$ la velocidad de Alfvén. En la parte neutra vamos a suponer que esta dominada por el campo magnético, por lo que tendrá valores en el rango de 1--\SI{3}{km.s^{-1}}, que son valores típicos en el medio interestelar \citep{Bertoldi_1989}. Mientras que la velocidad del sonido en la parte neutra es de \SI{0.5}{km.s^{-1}}. Así tenemos que la razón entre la densidad neutra e ionizada 
\begin{equation}
    \frac{n_n}{n_{i,0}}=\frac{2c_{s,i}^2}{c_{s,n}^2+\frac{1}{2}v_A^2},
\end{equation}
está en el rango de 42--266. Para estimar la masa de los glóbulos vamos a considerar que $n_n=100n_{i,0}$, aunque claro, tendríamos una gran incertidumbre por un factor de 2 aproximadamente. Tomando esto en cuenta y que en el modelo estamos considerando que los glóbulos son esféricos, entonces, la masa de cada glóbulo estaría dada por
\begin{equation}
    M_g = \frac{4}{3}\pi r_0^3 100n_{i,0} m_\mathrm{H}.
\end{equation}

Con estas densidades obtenidas para la parte neutra, tenemos un valor promedio en la masa de \SI{4.5e-3}{\msun}.
Si consideramos que todos los glóbulos tienen esta masa promedio, entonces tendríamos que la masa total de todos los glóbulos es $168\times\SI{4.5e-3}{\msun}=$\SI{0.75}{\msun}\footnote{En la nebulosa encontramos 168 glóbulos (Sección \ref{Chapter : Ajuste}).}, la cual es menor que la masa de gas ionizado que calcula \cite{Grosdidier:1998}, que es de \SI{1.33}{\msun}.

%Masas y tasa de pérdida de masas 
\section{Complicaciones }

En este modelo asumimos que el viento estelar viaja de manera libre hacia el glóbulo, por lo que no podría aplicarse muy bien a los casos en los que la presión externa es modificada antes de llegar al glóbulo, es decir, que la presión externa que llega a un glóbulo haya sido desacelerada por una primera interacción con algún otro glóbulo. Por ejemplo, el caso de la Figura \ref{fig:multiples_glo}, dentro del círculo rojo, la presión externa pudo haber interactuado con el glóbulo que se encuentra en la parte superior y entonces, la presión externa que actúa sobre el glóbulo que está debajo ya ha sido desacelerado.

La corrección por ángulo en principio se puede aplicar solo a los glóbulos cuya presión de la cáscara  es menor que la presión RAM. Si la presión de la cáscara fuera mayor que la presión RAM del viento entonces la cáscara se tendría que expandir hasta encontrar un equilibrio con la presión RAM. Esto en puede pasar cuando tenemos un escenario como se muestra en la Figura \ref{fig:multiples_glo}. En esta figura se ve como la presión RAM primero actúa sobre el glóbulo más cercano a la estrella, en cambio para el otro glóbulo esta presión externa se ve afectada por la interacción anterior. Por lo que la presión RAM no actúa directamente sobre el glóbulo más lejano ya que ha sido desacelerado. Estos glóbulos fueron descartados debido a que en la cáscara podría haber material del otro glóbulo y que tendríamos que calcular la presión que actúa sobre ellos ya que la presión RAM se vio afectada por una primer interacción.\\

\begin{figure}[htb]
    \centering
    \includegraphics[width=\textwidth]{Nuevas imagenes finales/multiples_glo.pdf}
    \caption{Dentro del círculo rojo un caso en el que dos glóbulos están unidos. En la imagen de la izquierda (gas ionizado) no se puede ver claramente si los glóbulos están juntos o no.  En cambio, en la imagen de la derecha (f1130w) vemos como el glóbulo más lejano está interactuando con la estela del otro glóbulo cercano, y en la imagen central (gas neutro) vemos como sus partes neutras parecen estar incluso juntas. En situaciones como esta, es muy probable que el glóbulo no este interactuando con el viento estelar directamente, por lo que, en caso de tener una cáscara chocada, no podríamos comparar muy bien con el modelo debido a que la presión con la que está interactuando es distinta a la presión RAM del viento.}
    \label{fig:multiples_glo}
\end{figure}

En la Figura \ref{graf_presion} vemos como un glóbulo tiene una presión mayor la presión RAM. Para este glóbulo en particular hay dos cosas que notar, considerando la barra de error este aún podría tener una presión menor o igual que la presión RAM del viento estelar y también puede que esté contaminado por la emisión de la nebulosa, esto debido a que su brillos, tanto en la parte interna como en la cáscara son muy grandes, por lo que podríamos estar sobre estimando la presión de la cáscara.\\

En esta corrección por ángulo no esperamos ver glóbulos con un ángulo de inclinación de $\pm 90^\circ$ ya que estos estarían totalmente de espaldas o de frente, en cualquier caso estos no los podríamos ver porque se verían afectados por la emisión de la estrella WR ya que en proyección estarían muy cerca de la estrella.

\section{Distribución tridimensional de los glóbulos} \label{sec: distrtibucion}

\cite{Zavala:2022} utilizando espectroscopia de rendijas en ciertas partes de la nebulosa midió las velocidades en estas regiones (ver Figura \ref{fig:zavala_rendijas_nebula}). Una de las observaciones se realizó en H$\alpha$ y además se realizó por donde están algunos glóbulos que se encontraron en este trabajo. 

En la Figura \ref{fig:zavala_nudos} vemos como los grupos de glóbulos están en una elipse, lo que indica que están en una cáscara esférica. En la rendija H es donde podemos ver el radio de estas cáscaras en el eje vertical, que es la posición. En la misma rendija, usando el eje horizontal que mide la velocidad, podemos ver que todas estas cáscaras tienen una velocidad de \SI{40}{km.s^{-1}}. De esta manera podemos obtener la separación real que hay entre los glóbulos y la estrella. Por lo que conociendo su separación proyectada, que es la que medimos directamente de las imágenes, y su separación real a partir de estos espectro de rendija, podemos saber cuál debería ser su ángulo de inclinación.

En la Figura \ref{fig:ang_Will} vemos como nuestras mediciones (puntos rojos), obtenidos a partir de comparar la presión de la cáscara de los glóbulos con la presión exterior del viento (Sección \ref{Sec:proyeccion}) concuerdan con la mediciones realizada a partir de espectros bidimensionales (puntos azules) (Henney, W., in prep.). Estas dos mediciones son totalmente independientes y podemos ver una clara anti correlación entre el ángulo de inclinación y la separación que observamos.

El grupo de glóbulos más cercanos a la estrellas parece estar a unos \SI{15}{\arcsecond} según la Figura \ref{fig:zavala_nudos}, los cuales coinciden con nuestras mediciones que vemos en la Figura \ref{fig:hist_sep_ryp}. También en la Figura \ref{fig:hist_sep_ryp} podemos ver que la separación real de muchos glóbulos está cerca de \SI{18}{\arcsecond}.

\begin{figure}[htb]
    \centering
    \includegraphics[width=\textwidth]{Nuevas imagenes finales/rendijas_zavala.pdf}
    \caption{Las lineas punteadas muestran donde \cite{Zavala:2022} realizaron espectroscopia de rendija larga en la nebulosa M1-67. Estas observaciones se realizaron en San Pedro Mártir. La línea continua en la esquina superior derecha representa \SI{20}{\arcsecond}.}
    \label{fig:zavala_rendijas_nebula}
\end{figure}

En la Figura \ref{fig:zavala_nudos} podemos apreciar que la mayoría de los grupos de glóbulos están corridos al rojo y muy pocos están corridos al azul, por lo que la mayoría de los grupos de glóbulos localizados en estas rendijas se están alejando de nosotros y solo unos pocos se están acercando.

\begin{figure}[htb]
    \centering
    \includegraphics[width=\textwidth]{Nuevas imagenes finales/zavala-slit-spectra-select-annotated.pdf}
    \caption{Aquí vemos la espectroscopia de rendijas de \cite{Zavala:2022}, en la parte superior vemos la emisión de H$\alpha$ y en la parte inferior la emisión de NII. Los círculos amarillos representan donde se encuentran los grupos de glóbulos en cada rendija. Las elipses representan modelos cáscaras esféricas con radios de \SI{15}{\arcsecond} (elipses moradas), \SI{20}{\arcsecond} (elipses rosas), \SI{25}{\arcsecond} (elipses amarillas), \SI{30}{\arcsecond} (elipses verde claro) y \SI{35}{\arcsecond} (elipses verdes), todos con una velocidad de \SI{40}{km.s^{-1}}.}
    \label{fig:zavala_nudos}
\end{figure}

\begin{figure}[htb]
    \centering
    \includegraphics[width=\textwidth]{Nuevas imagenes finales/W_1.pdf}
    \caption{Los datos en color rojo son nuestras estimaciones de los ángulos de inclinación realizados en la sección \ref{Sec:proyeccion}, en la que suponemos que la presión de la cáscara está en equilibrio de presión con la presión externa del viento. Los datos en color azul son las estimaciones de los ángulos de inclinación usando los espectros bidimensionales de \cite{Zavala:2022}}
    \label{fig:ang_Will}
\end{figure}

\chapter{Conclusiones}

En esta tesis se ha propuesto un modelo hidrodinámico estacionario para explicar la interacción que hay entre el flujo fotoevaporativo de los glóbulos encontrados en la nebulosa M1-67 y el viento estelar por parte de la estrella WR-124. 

Estos glóbulos fueron posibles de detectar, así como su cáscara chocada, gracias a las diferentes observaciones. De estas observaciones fue posible encontrar diferentes parámetros físicos haciendo un ajuste de dos gaussianas y una constante como vimos en el capítulo \ref{Chapter : Ajuste}. Estos parámetros físicos fueron:

\begin{itemize}
    \item Separación proyectada entre el glóbulo y la estrella
    \item Radio de los glóbulos
    \item Radio de la cáscara
    \item Ancho de la cáscara chocada
    \item Brillo del glóbulo
    \item Brillo de la cáscara
\end{itemize}

Usando la EM calculamos la densidad en la cáscara chocada (Sección \ref{Sec : estimacion de densidad}), y suponiendo que la cáscara chocada está dominada por presión térmica, pudimos conocer la presión de las cáscaras de los glóbulos.

En la Figura \ref{graf_presion} vemos que la presión de los glóbulos es menor que la presión RAM del viento estelar, sin embargo, podemos encontrar un ángulo de inclinación respecto al plano del cielo con el cuál podemos tener un equilibrio de presiones entre la presión de la cáscara y la presión RAM del viento estelar (Figura \ref{graf_presion_ang}). Estas estimaciones de los ángulos de inclinación en los que suponemos un equilibrio de presiones (Sección \ref{Sec:proyeccion}), son consistentes con la mediciones hechas a partir de espectros bidimensionales (Sección \ref{sec: distrtibucion}).

Además, usando las observaciones del HST, comparamos la razón de la presión térmica de la cáscara entre la presión del flujo fotoevaporativo en la base del glóbulo (Sección \ref{Sec : comparacion-modelo}). Afortunadamente, estas mediciones concuerdan con nuestro modelo teórico como podemos ver en la Figura \ref{Resultados_modelo}.

Por lo que podemos decir que este modelo sencillo es un buen modelo para explicar la interacción del flujo fotoevaporativo de los glóbulos y el viento estelar por parte de la estrella WR 124. Además, no tenemos parámetros libres, por lo que la consistencia con otros trabajos apoya nuestro modelo propuesto. Por lo que nuestras suposiciones de que estos glóbulos ya han pasado por la fases mencionadas en el Capítulo \ref{Capitulo 1:introduccion} y que estamos en un equilibrio de ionización son ciertas. 

Este modelo propuesto en un principio está puesto para un escenario sencillo, un glóbulo que es radiado por una fuente, pero esto podría ampliarse un poco más si tenemos varias fuentes que radian al glóbulo y una de ellas es la que domina en cuanto al flujo de fotones ionizantes. 

\appendix

\chapter{Filtros de las observaciones}\label{App: Filtros}

En este apéndice vamos a dar más detalles acerca de los diferentes filtros que se utilizaron en las observaciones.
\section{Filtro del HST}
Para el caso del HST solo se utilizaron las observaciones del filtro f656n, en el cual podemos ver la emisión de H$\alpha$. Este filtro observa desde \SI{6548.77}{A} hasta \SI{6674.27}{A} y está centrada en \SI{6563.8}{A} como se puede ver en la Figura \ref{fig:filtro f656n}.

\begin{figure}
    \centering
    \includegraphics[width=\textwidth]{Appendices/HST_WFC3_UVIS1.F656N.png}
    \caption{Transmisión del filtro f656n del HST}
    \label{fig:filtro f656n}
\end{figure}

\section{Filtros del JWST}\label{AP S:JWST}

Para el caso de las observaciones del JWST se utilizaron diferentes filtros, tanto del NIRcam como del MIRI. El filtro f1130w es el único filtro del MIRI y observa desde 10.953--\SI{11.667}{\mum} y está centrado en \SI{11.3}{\mum}. En la Figura \ref{fig:filtos MIRI} vemos la transmisión de este filtro.

\begin{figure}
    \centering
    \includegraphics[width=\textwidth]{Appendices/miri_img_pces_ETC4.0.png}
    \caption{Transmisión de los filtros de MIRI. El filtro f1130w está en azul.}
    \label{fig:filtos MIRI}
\end{figure}

Para los filtros del NIRcam  tenemos a los filtros f090w, f150w, f444w, f210m y f335m, en la tabla \ref{tab:filtros} podemos ver en que longitudes de onda observa cada filtro. En la Figura \ref{fig:filtros JWST} vemos la transmisión de cada filtro. 

\begin{table}[htb]
    \centering
    \begin{tabular}{c c c c}
        \toprule
        filtro & $\lambda_0$($\mu$m) & $\lambda_{min}$($\mu$m) & $\lambda_{max}$($\mu$m) \\ 
        \midrule
         f090w & .903 & .795 & 1.005\\
         f150w &1.501 &1.331 & 1.668\\
         f210m &2.096 &1.992 & 2.201\\
         f335m &3.362 &3.177 & 3.537\\
         f444w &4.401 &3.880 & 4.981\\
         \bottomrule
    \end{tabular}
    \caption{Rango en el que observa cada filtro utilizado para las observaciones utilizadas. $\lambda_0$ es la longitud de onda a la que está centrado cada filtro, $\lambda_{min}$ es la longitud de onda mínima a la que observa y $\lambda
    _{max}$ es la longitud de onda más grande a la que observa cada filtro.}
    \label{tab:filtros}
\end{table}

\begin{figure}
    \centering
    \includegraphics[width=\textwidth]{Appendices/multi_filter_plot_per_detector_May2024.png}
    \caption{Transmisión de los filtros de NIRcam. Los filtros f090w, f150w y f444w se encuentra en la segunda imagen, mientras que los filtros f210m y f335m se encuentran en la tercera imagen.}
    \label{fig:filtros JWST}
\end{figure}

\chapter{Estimación de fuerzas en el flujo fotoevaporativo ionizado}\chaptermark{Fuerzas en el flujo fotoevaporativo} \label{App:fuerzas}

En estas estimaciones de las diferentes fuerzas solo haremos aproximaciones, por lo que para los cálculos vamos a usar los valores típicos de los ajustes (tabla \ref{tab:mean} y \ref{tab:mean_i}).

Para comparar las distintas fuerzas es más conveniente comparar las presiones o aceleraciones ya que estas son fuerza por unidad de área o fuerza por unidad de masa, respectivamente. 


Primero vamos a considerar la aceleración provocada por el gradiente de presión, para esto tomamos la fuerza por unidad de masa la cual está dada por 
\begin{equation}
\rho a = \frac{dP}{dr}\Rightarrow a= \frac{1}{\rho}\frac{dP}{dr}=\frac{1}{\rho}\frac{\partial \rho}{\partial r}\frac{\partial P}{\partial\rho}=\frac{c_s^2}{h}
\end{equation}
como estamos considerando un gas isotérmico, vemos que del lado derecho tenemos los factores de la velocidad del sonido cuadrada y la escala de altura $h$, la cual está definida por 
\begin{equation}
h^{-1}=\Big|\frac{1}{\rho_0}\frac{\partial\rho}{\partial r}\Big|=\Big|\frac{d \ln \rho}{dr}\Big|
\end{equation}
usando las ecuaciones \ref{eq : 2} y \ref{eq ; 3} tenemos que 
\begin{equation}
\frac{\rho}{\rho_0}=e^{\frac{1-M^2}{2}}\Rightarrow\ln\frac{\rho}{\rho_0}=\frac{1-M^2}{2}
\end{equation}
\begin{equation}
\Rightarrow d\ln\frac{\rho}{\rho_0}=-M dM
\end{equation}
por otro lado usando que 
\begin{equation}
\frac{r}{r_0}=M^{-1/2}e^{\frac{M^2-1}{2}}
\end{equation}
tenemos que 
\begin{equation}
d\frac{r}{r_0}=e^{\frac{M^2-1}{4}}\Big(-\frac{M^{-3/2}}{2}+\frac{M}{2M^{1/2}}\Big)dM = \frac{e^{\frac{M^2-1}{4}}}{2}\Big(M^{1/2}(1-1/M^2) \Big)dM
\end{equation}
por lo que 
\begin{equation}
h^{-1}=\Big|\frac{2M}{e^{\frac{M^2-1}{4}}\Big(M^{1/2}-M^{-3/2}\Big)}\Big|
\end{equation}
que como podemos ver en la Figura \ref{fig:h} cuando $r/r_0\sim 1$ tenemos un valor del orden de 0.2\footnote{En la Figura \ref{fig:h} vemos que cuando $r=r_0$ $h\to 0$, por lo que tendríamos una aceleración infinita.}. Por lo que el gradiente de presión nos da una aceleración de 
\begin{equation}
a_p \approx \frac{c_s^2}{0.2 r_0} = \SI{3.8e-4}{cm.s^{-2}}
\end{equation}
a un radio típico $r_0\sim \SI{0.135}{\arcsecond}$. Como el modelo se resolvió en el eje de simetría, tenemos que aquí los gradientes transversales son cero, mientras que si consideramos el modelo a cierto ángulo debemos considerar que el gradiente de densidad transversal es más pequeño, por un factor de 10 aproximadamente. 

\begin{figure}
    \centering
    \includegraphics{Appendices/h_3.jpg}
    \caption{Gráfica de $h$ con respecto al radio normalizado. Vemos que cerca de donde tenemos la emisión en $r/r_0$ el valor de $h$ es del orden de 0.1}
    \label{fig:h}
\end{figure}

\section{Fuerzas de gravedad} \label{F gravedad}

\cite{Hamann:2019} estima una masa  de 20-$\SI{22}{\msun}$ para la estrella WR 124, por lo que la fuerza de gravedad por parte de la estrella nos da un aceleración de 
\begin{equation}
a_*=\frac{GM_*}{R^2}\approx \SI{1.97e-9}{cm/s^2}
\end{equation}
con $R$ una distancia típica entre la estrella y el glóbulo de \SI{14.96}{\arcsecond}. Si tomamos la distancia típica considerando el ángulo de inclinación $i$ (Sección \ref{Sec:proyeccion}), vemos que esta aceleración es todavía más pequeña.

Ahora vamos a considerar la aceleración por parte de la gravedad del mismo glóbulo. Para esto, vamos a considerar la masa neutra del glóbulo y la masa ionizada. En la sección \ref{app:masa_glo} hablamos de como obtener la masa neutra de los glóbulos, para la masa ionizada vamos a considerar la masa que se encuentra en la mitad de la cáscara que hay entre $r_0$ y $r_\mathrm{shell}$, por lo que a masa ionizada es
\begin{equation}
    M_i = \rho_1 \frac{2\pi}{3}(r_\mathrm{shell}^3-r_0^3)
\end{equation}
donde $\rho_1$ es la densidad por unidad de masa en la parte ionizada. De esta manera, tenemos que la masa del glóbulo a es $M_g = M_i + M_n \approx \SI{7.87e-4}{\msun}$ (considerando una $v_A=\SI{1}{km.s^{-1}}$ para la parte neutra) y la aceleración por parte de la fuerza de gravedad del mismo glóbulo es de
\begin{equation}
a_g=\frac{G M_g}{r_\mathrm{shell}^2}\approx \SI{4.55e-11}{cm/s^2}.
\end{equation}
Si en esta estimación tomamos una $v_A$ mayor, tendríamos una aceleración un poco menor. De igual, si consideramos la densidad ionizada de la tabla \ref{tab:mean_i}, esta aceleración no cambia mucho. Para el caso de lo glóbulos que están en grupo, de igual manera podemos despreciar la aceleración por parte de los demás glóbulos, ya que por muy cercanos que estén, podemos considerar una distancia mínima de $r_\mathrm{shell}$. 

De esta manera tenemos que
\begin{equation}
a_g<a_*\ll a_p
\end{equation}
donde $a_g$ es la aceleración provocada por la gravedad del glóbulo, $a_*$ la aceleración provocada por la estrella WR 124 y $a_p$ la aceleración provocada por la diferencia de presiones en la superficie del glóbulo.

\section{Presión de radiación}

Vamos a considerar la presión de radiación ya que podemos suponer que todo el momento de los fotones ionizantes se va al flujo fotoevaporativo, por lo que si consideramos que todo la radiación ionizante es absorbida en el flujo fotoevaporativo entonces tenemos que para la radiación ionizante $Q = \SI{1.25e49}{s^{-1}}$, según la tabla \ref{tab:parametros WR-124}, tendríamos una intensidad de 
\begin{equation}
\frac{Q h\nu}{4\pi R^2} = \frac{\SI{2.74e38}{erg.s^{-1}}}{4\pi R^2} \approx \SI{6.8}{erg.s^{-1}.cm^{-2}}
\end{equation}
para la frecuencia que corresponde a  $\SI{1}{Ry}$, el cual es un límite inferior para los fotones que son capaces de ionizar el gas neutro, esto a una distancia típica de los glóbulos. Por lo que tendríamos una presión de radiación de $P_r\approx \SI{2.26e-10}{dyn.cm^{-2}} < P_\mathrm{shell}$.

\chapter{Combos de filtros}\label{AP: combos}

En este apéndice vamos a explicar como se obtuvieron los combos de gas ionizado y gas neutro utilizando las diferentes observaciones

\chapter{Escalas de tiempo}

\section{Tiempo dinámico}

Usando los valores de la tabla \ref{tab:mean}, para el flujo fotoevaporativo tenemos un tiempo dinámico 
\begin{equation}
t_{DF} = \frac{r_\mathrm{shell}}{v} \approx \SI{5.27e10}{s}  = \SI{1.67e3}{yr}.
\end{equation}

\cite{Mancherko:2010} estima una velocidad de expansión para la nebulosa de 42--\SI{46}{km.s^{-1}} por lo que para la nebulosa tenemos un tiempo dinámico de 
\begin{equation}
t_{DN}= \frac{R_{nebula}}{v_{exp}}\approx\frac{\SI{1.5}{pc}}{\SI{46}{km.s^{-1}}}= \SI{1e12}{s}=\SI{3.18e4}{yr}
\end{equation}

\section{Tiempo de recombinación}

Usando la densidad promedio de la tabla \ref{tab:mean} tenemos un tiempo de recombinación 
\begin{equation}
t_r = \frac{1}{\alpha_B n} \approx \SI{3.64e9}{s}= \SI{115.64}{yr}
\end{equation}

Para el caso del tiempo de calentamiento-enfriamiento, vamos a considerar que es de 3--5 veces menor que el tiempo de recombinación. Esto considerando que $t_c=\frac{3P}{2\mathcal{L}}=\frac{3k_\mathrm{B}T}{\Lambda n}$ donde $P=2nk_\mathrm{B}T$ es la presión y $\mathcal{L}=\Lambda n^2$, así tenemos que $\frac{t_c}{t_r}=\frac{3k_\mathrm{B}T\alpha_\mathrm{B}}{\Lambda}$ y considerando que $3k_\mathrm{B}T\approx\SI{4e-12}{erg},\; \alpha_\mathrm{B}=\SI{2.6e-13}{cm^3.s^{-1}}$ y $\Lambda\approx\SI{3e-24}{erg.cm^3.s^{-1}}$ entonces
\begin{equation}
    \frac{t_c}{t_r}\approx\frac{\SI{e-24}{erg.cm^3.s^{-1}}}{\SI{3e-24}{erg.cm^3.s^{-1}}}=\frac{1}{3}
\end{equation}

\section{Tiempo de vida de los glóbulos}

Para calcular el tiempo de vida de los glóbulos, primero vamos a considerar que su tasa de pérdida de masa está dada por 
\begin{equation}
    \dot{M} =\pi r_0^2n_{i,0}c_{s,i}m_\mathrm{H}
\end{equation}
usando los valores típicos de la tabla \ref{tab:mean}, tenemos que $\dot{M} = \SI{4.91e-8}{\msun.yr^{-1}}$. Por lo que el tiempo de vida de los glóbulos es
\begin{equation}
    t_{glo}=\frac{M_g}{\dot{M}}=\SI{9.16e4}{yr}
\end{equation}

Por lo que tenemos que
\begin{equation}
    t_{cool} < t_r < t_{DF} < t_{DN} \sim t_{glo}
\end{equation}
Por lo que los tiempos de vida de los glóbulos y el tiempo dinámico de la nebulosa son similares. 

\chapter{Predicciones del modelo fotoevaporativo para la densidad ionizada en el frente de ionización }\chaptermark{Densidad ionizada en el frente de ionización}\label{App : tasa de fotoionizacion}

En el modelo suponemos un estado estacionario y además, también suponemos que no hay absorción por polvo, por lo que el flujo incidente de fotones, $F_0$, debe ser igual a la suma de dos términos, las recombinaciones por unidad de área y las nuevas ionizaciones
\begin{equation}\label{eq:flujo_io}
F_0 = n_{i,0} u_{i,0} +\int n^2\alpha_\mathrm{B}dr = n_{i,0}u_{i,0}+n_{i,0}^2h_1\alpha_\mathrm{B}
\end{equation}
donde $F_0$ es la tasa de fotones ionizantes por unida de área, $n_{i,0}$ la densidad del gas ionizado, $u_0$ la velocidad del gas ionizado y $h_1$ es la anchura efectiva de la capa ionizada que se define como
\begin{equation}
n_0^2h_1=\int n^2dr,
\end{equation}
el cual se puede estimar usando las ecuaciones  \ref{eq : 1}, \ref{eq : 2} y \ref{eq ; 3}. Por lo que tendríamos que 
\begin{equation}
h_1=\int_0^\infty \Big(\frac{n(r)}{n_0}\Big)^2dr=r_0\int_1^\infty\frac{exp(\frac{3}{4}(1-M^2))}{2}(M^{1/2}-M^{3/})dM\approx0.12r_0.\end{equation}

Tomando los valores de $F_0$ y $r_0$ de nuestros glóbulos, tenemos que, el primer término del lado derecho de la ecuación \ref{eq:flujo_io} es muy pequeño en comparación con el segundo término. Por lo que podemos despreciar este término.

\chapter{Corrección en la estimación de los brillos}\label{App:brillos}

En la Sección \ref{Sec : comparacion-modelo} usamos los resultados obtenidos a partir de las observaciones para comparar con el modelo. Por lo que ahora vamos a calcular las correcciones a los brillos estimados, tanto de la parte interna como de la cáscara, por efectos instrumentales.

En el caso de la cáscara, vamos a ignorar estas correcciones debido a que está bien resuelta y  el brillo no se ve reducido por el PSF del telescopio.

Por otro lado, para la parte interna tenemos un radio muy pequeño. De hecho es casi un píxel en las observaciones del HST. Así que para esta corrección vamos a considerar dos efectos instrumentales, uno por el efecto del PSF y el otro por el tamaño del píxel.

Si asumimos que el perfil de brillo real tiene un perfil gaussiano como función de la distancia $r$ 
\begin{equation}
B(r)= B_0 e^{-r^2/2\sigma_0}
\end{equation} 
tenemos que el flujo total está dado como 
\begin{equation}
F_0=\iint_{-\infty}^\infty B(r)dxdy=B_0 \pi r_{eff}^2=B_02\pi \sigma_0^2.
\end{equation}

Para considerar estas correcciones por los dos efectos instrumentales, vamos a asumir que estos también están dados por perfiles gaussianos. De este modo, al convolucionar dos perfiles gaussianos con parámetros $\sigma_1$ y $\sigma_2$, tenemos que 
\begin{equation}
\sigma^2=\sigma_1^2+\sigma_2^2
\end{equation}
donde $\sigma$ sería el parámetro de los dos perfiles convolucionados.

En nuestro caso, lo que observamos es el perfil del brillo real convolucionado con el perfil del PSF y el perfil de las rendijas de los píxeles. Entonces 
\begin{equation}
\sigma_{obs}^2=\sigma_0^2+\sigma_{PSF}^2+\sigma_{pix}^2
\end{equation}
donde $\sigma_{PSF}=\frac{W_{PSF}}{2\sqrt{2\ln{2}}}$, siendo $W_{PSF}$ el ancho del PSF a la altura media, y $\sigma_{pix}=\frac{\Delta X_{pix}}{\sqrt{2\pi}}$, siendo $\Delta X_{pix}$ el tamaño del píxel. Entonces, para comparar el los brillos reales y los observados tenemos que
\begin{equation}
\frac{F_0}{F_{obs}}=\frac{B_0\sigma_0^2}{B_{obs}\sigma_{obs}^2}=\frac{B_0}{B_{obs}}\frac{\sigma_0^2}{\sigma_0^2+\sigma_{PSF}^2+\sigma_{pix}^2}
\end{equation} 
\begin{equation}
\Rightarrow \frac{B_0}{B_{obs}}=1+\frac{\sigma_{PSF}^2+\sigma_{pix}^2}{\sigma_0^2}.
\end{equation}

Debido a que usamos un solo radio para todos los glóbulos, esta corrección es solo una constante, y gracias a este valor, los datos obtenidos a partir de las observaciones se ajustan muy bien al modelo propuesto.


\chapter{Imágenes de ajustes}\label{App : ajustes}

En este Apéndice mostramos los ajustes realizados a los glóbulos. En la parte izquierda mostramos el ajuste a los perfiles de brillo para las observaciones del HST y a su lado la visualización de estos ajustes en el filtro f656n del HST. Del lado derecho vemos el ajuste a los perfiles de brillo para el gas ionizado usando los datos del JWST, y a su lado la representación de estos ajustes en el filtro f090w del JWST, ya que morfológicamente se parece a las imágenes del HST.

\begin{figure}[htb]
    \centering
    % Crear un nodo en la ubicación deseada
    \begin{tikzpicture}[overlay, remember picture]
        \node[anchor=south west, xshift=.2cm, yshift=3cm] at (current page.south west) {
            \includegraphics[width=1.7\textwidth]{Nuevas imagenes finales/Aj_01.pdf}
        };
    \end{tikzpicture}
    \caption{Visualización de los ajustes realizados a los glóbulos.}
    \label{fig:ajuestes_apendice}
\end{figure}

\newpage

\begin{tikzpicture}[overlay, remember picture]
  % La imagen se coloca en una posición específica
  \node[anchor=south west, xshift=1.3cm, yshift=.2cm] at (current page.south west) {
    \includegraphics[width=7.5in]{Nuevas imagenes finales/Aj_02.pdf}
  };
\end{tikzpicture}

\newpage

\begin{tikzpicture}[overlay, remember picture]
  % La imagen se coloca en una posición específica
  \node[anchor=south west, xshift=1.3cm, yshift=.2cm] at (current page.south west) {
    \includegraphics[width=7.5in]{Nuevas imagenes finales/Aj_03.pdf}
  };
\end{tikzpicture}

\newpage

\begin{tikzpicture}[overlay, remember picture]
  % La imagen se coloca en una posición específica
  \node[anchor=south west, xshift=1.3cm, yshift=.2cm] at (current page.south west) {
    \includegraphics[width=7.5in]{Nuevas imagenes finales/Aj_04.pdf}
  };
\end{tikzpicture}

\newpage

\begin{tikzpicture}[overlay, remember picture]
  % La imagen se coloca en una posición específica
  \node[anchor=south west, xshift=1.3cm, yshift=.2cm] at (current page.south west) {
    \includegraphics[width=7.5in]{Nuevas imagenes finales/Aj_05.pdf}
  };
\end{tikzpicture}

\newpage

\begin{tikzpicture}[overlay, remember picture]
  % La imagen se coloca en una posición específica
  \node[anchor=south west, xshift=1.3cm, yshift=.2cm] at (current page.south west) {
    \includegraphics[width=7.5in]{Nuevas imagenes finales/Aj_06.pdf}
  };
\end{tikzpicture}

\newpage

\begin{tikzpicture}[overlay, remember picture]
  % La imagen se coloca en una posición específica
  \node[anchor=south west, xshift=.5cm, yshift=5cm] at (current page.south west) {
    \includegraphics{Nuevas imagenes finales/Aj_07.pdf}
  };
\end{tikzpicture}


\bibliography{references}

\end{document}

%%% Local Variables:
%%% mode: LaTeX
%%% TeX-master: t
%%% End:
