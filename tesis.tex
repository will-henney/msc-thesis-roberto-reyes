\documentclass{book}
\usepackage{graphicx} % Required for inserting images

\usepackage[utf8]{inputenc}
\usepackage[T1]{fontenc}
\usepackage{listings}
\renewcommand{\lstlistingname}{Algoritmus}

\title{Avances de tesis}
\author{r.reyes }
\date{December 2023}

\begin{document}

\maketitle

\tableofcontents

\newpage

\chapter{Introducción}

En el medio interestelar hay muchas cosas maravillosas que nos permiten entender mucho acerca de la física en las diferentes regiones que observamos.

Las estrellas se forman a partir del colapso gravitacional en nubes moleculares frías, teniendo sus diferentes fases durante este proceso, algunas regiones en las que se pueden formar son en lo que llamamos \textit{glóbulos de Bok}, que son nubes oscuras, relativamente pequeñas si las comparamos con otras regiones de formación estelar, que tienen una gran cantidad de gas y polvo. Estas nubes fueron observadas por primera vez por el astrónomo Bart Bok en 1940. 

Estos glóbulos contienen principalmente hidrógeno molecular en su interior, así como también puede tener otras moléculas, metales e incluso algunos silicatos. Si bien pueden tener formación estelar en su interior, no podemos ver la radiación UV ya que esta es absorbida por el hidrógeno atómico y el polvo, es por eso que se ven oscuras. Sin embargo, estos pueden ser radiados externamente, en regiones de formación estelar, por estrellas jóvenes que se están formando cerca, y si la radiación es lo suficientemente fuerte, en algunos casos podemos ver el frente de ionización. Gracias a estas interacciones podemos ver que se forman estructuras como columnas, dedos o pilares en estas nubes

Esta interacción se puede dar a diferentes escalas, por lo que se les puede conocer con diferentes nombres según la escala. Por ejemplo en Cygnus OB2, que es una de las regiones se mayor formación estelar, vemos que la interacción de estos glóbulos puede alcanzar tamaños de hasta 1 pc. También se ha observado a escalas más pequeñas, como en regiones H II. 

Esto también se puede dar en nebulosas planetarias, donde se les conoce también como nudos. Aquí vemos que la escala física es mucho más pequeña que en los glóbulos de Bok. 

\section{Flujos de foto evaporación ionizada}

Como ya mencionamos hay muchos tipos de glóbulos de diferentes tamaños, en diferentes regiones. Algunos más pequeños que otros. 

En nuestro caso nos concentraremos en lo que llamaremos \textit{nudos}, los cuales se encuentran principalmente en nebulosas planetarias, estos son glóbulos de un tamaño relativamente pequeño y tiene una gran concentración molecular neutra. Debido a la radiación UV de la estrella central esta comienza a ser ionizada en la superficie más cercana a la estrella. ...
%Lyman continuo, proceso intermedio
Después de cierto tiempo, podemos llegar a un equilibrio de ionización, donde esta superficie cercana a la estrella es ionizada por el continuo de Lyman, mientras que también tenemos procesos de recombinación. Estas recombinaciones de en la superficie de los nudos pueden emitir intensamente ...

\section{Estrellas Wolf-Rayet y sus vientos}

\section{Nebulosa M1-67}

Gracias a las nuevas imágenes de JWST podemos saber mejor como es la nebulosa M1-67 que rodea la estrella WR-124. En setas imágenes vemos como la nebulosa es muy simétrica en algunos filtros, mientas que en algunos otros la emisión de los diferentes mecanismo se ven más intensos de un lado que del otro.

Algo muy notorio a simple vista es que tiene una simetría que...

%Poner imagen de M1-67

Podemos ver como muchos detalles en cada uno de los filtros, en especial de como podemos ver en color rosa lo que parece ser choques de una gran acumulación de gas neutro que está interactuando con el viento de la estrella, y al rededor de ellos vemos una morfología similar a la de los proplyds que hay en otras nubes moleculares. También podemos ver como estos tienen una estela en la parte más lejana de la interacción, pero para nuestro estudio solo nos concentraremos solo en la parte más central.

\chapter{Modelos analíticos de flujos foto evaporativos interactuando con una presión externa}

\section{Estimación de la densidad ionizada a partir del brillo superficial de $H_\alpha$}

Para estimar la densidad ionizada, usamos primero la definición de medida de emisión (EM por sus siglas en inglés)
\[EM=\int_z n_in_edz\] donde en este caso estaremos integrando sobre nuestra línea de visión. Esta EM depende tanto de la densidad de lectrones como de iones, pero en nuestro caso vamos a considerar un equilibrio de ionización entre el flujo foto evaporativo y el viento estelar.

Si suponemos una simetría esférica entre esta interacción del flujo fotoevaporativo y el viento estelar, podemos tomar, por geometría, que
\[EM=2\sqrt{rh}n^2.\]

%Insertar imagen de los radios

Esto ya que de la fig(?) tenemos que por geometría $r^2+\frac{1}{2}l^2=(r+h)^2=r^2+2rh+h^2\approx r^2+2rh\Rightarrow l=2\sqrt{rh}$, por lo que podemos estimar la densidad ionizada a través de la EM como \[n=\sqrt{\frac{EM}{l}}\]

\section{Modelo hidrodinámico estacionario}

Para el caso de nuestro modelo analítico consideraremos un modelo analítico hidrodinámico estacionario, esto ya que consideramos que la interacción entre el flujo foto evaporativo y el viento estelar ha llegado al equilibrio de ionización.

%inseratr imagen de las regiones durante la interaccion

Para nuestro modelo analítico tenemos las regiones:
* región neutra: En esta parte consideraremos que está constituida principalmente por el hidrógeno atómico, el cual tendrá parámetros : $r_0,\rho_0,P_0$. Es la parte más central de nuestro glóbulo.

* Región interna: Es la región que esta seguida de la región neutra, en la cual el flujo foto evaporativo que sale de la superficie de la región neutra tiene un número de Mach $M_0$. Esta región tiene como parámetros $r_1,\rho_1,P_1$.

* Cáscara chocada: Es la parte donde vemos el choque que hay entre el flujo foto evaporativo y el viento estelar, en realidad este también tendrá un radio $r_2$, pero que será muy parecido a $r_1$ en tamaño.  

* Parte externa: Es donde está viajando el viento estelar con dirección al glóbulo molecular.

Para este modelo consideramos glóbulos relativamente pequeños, de unos mili arco segundos comparado con su entorno que puede ser unas pocas decenas de arco segundos, además a unos 10-30 arco segundos de la estrella masiva como para que su radiación UV llegue desde un pequeño ángulo e interactúe con el glóbulo, de esta manera podemos decir que la radiación incide perpendicularmente a la superficie del glóbulo.

En nuestro caso tomamos una simetría cilíndrica en la cual podamos tomar un radio perpendicular de un tamaño muy pequeño como para tomar este problema solo de manera radial, pero notemos que debido a la forma de estos glóbulos, en realidad tendremos una familia de cilindros a cada ángulo, por lo que tomando el radio perpendicular a la radiación muy pequeño, entonces podemos considerar este problema solo de manera radial, un problema unidimensional, en la dirección en la que incide la radiación de la estrella.

En general en el medio interestelar hay muchas fuerzas que afectan el cambio en la materia, pero en este caso vamos a despreciar la fuerza de gravedad, tanto del mismo glóbulo como la gravedad impuesta por la estrella, tampoco vamos a considerar fuerza por campos magnéticos por simplicidad. Por lo que solo vamos a considerar las fuerzas dada por el gradiente de presión por parte del flujo fotoevaporativo, donde tomaremos el viento solar como condición de frontera.

Para el flujo fotoevaporativo por parte del glóbulo vamos a considerar las ecuaciones de hidrodinámica, la ecuación de conservación de masa ya que como dijimos antes, el tiempo en el que un bulto sale de esta interacción le toma bastante tiempo. Usaremos también la ecuación de Bernuoulli para un gas isotermo, donde la velocidad del sonido en este gas dependerá de la temperatura.

% Balance between ionizations and recombinations

% Assumption of isothermal equation of state

% General solution for the internal structure of model


\chapter{Aplicación a M1-67}

\section{Observaciones con HST}

\section{Observaciones con JWST}

\section{Ajustando el modelo a los perfiles de brillo radial}

\section{Estimando la presión RAM del viento estelar}

\section{Estimando la tasa de foto ionización}

\chapter{Conclusiones}
\end{document}
